In diesem Abschnitt werden wir uns drei Fragen stellen:
	\begin{itemize}
		\item Was können wir messen?
		\item Wie können wir messen?
		\item Wie können wir Maße ökonomisch definieren?
	\end{itemize}
	
	\section{Mengensysteme}
		\begin{defi}
			Sei $\Omega$ eine beliebige Menge. Dann heißt $\mf{C}\subseteq 2^\Omega$ ein Mengensystem (über $\Omega$).
		\end{defi}
		
		\begin{defi} [Semiring]
			Sei $\mf{T}$ ein nichtleeres Mengensystem über $\Omega$. Dann heißt $\mf{T}$ Semiring (im weiteren Sinn), falls
			\begin{enumerate}
				\item Durchschnittsstabilität:
				\[ A,B\in\mf{T}\Rightarrow A\cap B\in\mf{T} \]
				\item Leiterbildung:
				\[ A,B\in\mf{T}, A\subseteq B\Rightarrow \exists n\in\N: C_1,...,C_n\in\mf{T}: \forall i\neq j: C_i\cap C_j=\varnothing, A\setminus B=\bigcup_{i=1}^n C_i \]
			\end{enumerate}
			gilt zusätzlich für die Leiter
			\[ \forall k=1,...,n: A\cup\bigcup_{i=1}^k C_i\in\mf{T}, \]
			so spricht man von einem Semiring im engeren Sinn.
		\end{defi}
		
		\begin{defi}[Ring]
			Sei $\mf{R}$ ein nichtleeres Mengensystem über $\Omega$. $\mf{R}$ heißt Ring, falls
			\begin{enumerate}
				\item Differenzenstabilität:
				\[ A,B\in\mf{R}\Rightarrow B\setminus A\in\mf{R} \]
				\item Vereinigungsstabilität:
				\[ A,B\in\mf{R}\Rightarrow A\cup B\in\mf{R} \]
			\end{enumerate}
		\end{defi}
		
		\begin{defi}[Sigmaring]
			Sei $\mf{R}_\sigma$ ein nichtleeres Mengensystem über $\Omega$. $\mf{R}_\sigma$ heißt Sigmaring, falls
			\begin{enumerate}
				\item Differenzenstabilität:
				\[ A,B\in\mf{R}_\sigma\Rightarrow B\setminus A\in\mf{R}_\sigma \]
				\item Sigma-Vereinigungsstabilität:
				\[ A_n\in\mf{R}_\sigma\Rightarrow \bigcup_{n\in\N}A_n\in\mf{R}_\sigma \]
			\end{enumerate}
		\end{defi}
		
		\begin{defi}[Algebra]
			Sei $\mf{A}$ ein nichtleeres Mengensystem über $\Omega$. $\mf{A}$ heißt Ring, falls
			\begin{enumerate}
				\item Abgeschlossenheit bzgl. Komplementbildung:
				\[ A\in\mf{A}\Rightarrow A^c\in\mf{A} \]
				\item Vereinigungsstabilität:
				\[ A,B\in\mf{A}\Rightarrow A\cup B\in\mf{A} \]
			\end{enumerate}
		\end{defi}
		
		\begin{defi}[Dynkin System]
			Sei $\mf{D}$ ein nichtleeres Mengensystem über $\Omega$. $\mf{D}$ heißt Dynkin-System (im weiteren Sinn), falls
			\begin{enumerate}
				\item Sigmaadditivität:
					\[A_i\in\mf{D}: A_i \text{ disjunkt}\Rightarrow \bigcup_{i\in\N} A_i\in\mf{D}\] 
				\item Differenzenstabilität:
					\[ \forall A,B\subseteq \Omega: A,B\in\mf{D}\Rightarrow B\setminus A\in\mf{D}\]
			\end{enumerate}
			Ist zusätzlich noch
			\[ \Omega\in\mf{D} \]
			erfüllt, so spricht man von einem Dynkin-System im engeren Sinn.
		\end{defi}
	
	\section{Maße und Inhalte}
		\begin{defi}
			Ein Inhalt $\mu$ auf einem Mengensystem $\mathfrak{C}$ heißt endlich, wenn für alle $A\in C$:
			\[ \mu(A)<\infty \]
		\end{defi}
		\begin{defi}
			Ein Maß $\mu$ auf $C$ heißt sigmaendlich, wenn für jedes $A\in C$ Mengen $A_n\in C, n\in \N$ existieren mit $\mu(A_n)<\infty, A\subseteq \bigcup_{n\in\N} A_n$.
		\end{defi}
		\begin{defi}
			Ein Inhalt $\mu$ auf $C$ heißt totalendlich, wenn
			\[ \Omega\in C\land \mu(\Omega)<\infty \]
		\end{defi}
		\begin{defi}
			Ein Inhalt $\mu$ auf $C$ heißt total sigmaendlich, wenn es $A_n\in C, n\in \N$ gibt mit $\mu(A_n)<\infty$ und $\Omega\subseteq \bigcup_{n\in \N} A_n$. 
		\end{defi}
		\begin{defi}
			$A\in C$ hat sigmaendliches Maß ($A$ ist sigmaendlich), wen es $A_n\in C, n\in\N: \mu(A_n)<\infty$ und $A\subseteq \bigcup A_n$. 
		\end{defi}
		\begin{defi}
			$\mu$ heißt Wahrscheinlichkeitsmaß, wenn $\mu(\Omega)=1$.
		\end{defi}
		\begin{bsp}
			Sei $\Omega\neq\varnothing$ endlich, $C=2^\Omega, \mu(A)=\frac{|A|}{|\Omega|}$.
		\end{bsp}
		\begin{bsp}
			Sei $\Omega=\{1,2,3,4,5,6\}$, also ein "`fairer Würfel"'.
			\end{bsp}
		\begin{bsp}
			Sei $\Omega=\{(1,1), (1,2),...,(2,1),(2,2),...,(6,6)\}$, also würfeln mit zwei Würfeln, Würfel sind unterscheidbar. 
		\end{bsp}
		\begin{defi}
			Sei $\Omega\neq\varnothing$ beliebige Menge und $\ms{S}$ eine Sigmaalgebra über $\Omega$. Dann heißt $(\Omega, \ms{S})$ Messraum. 
		\end{defi}
		\begin{defi}
			Sei $\mu$ ein Maß auf $\ms{S}$ und $(\Omega, \ms{S})$ Messraum. Dann heißt $(\Omega, \ms{S}, \mu)$ Maßraum. 
		\end{defi}
		\begin{bsp}
			$(\Omega, 2^\Omega, \mu)$, $\Omega\neq\varnothing$ endlich, $C=2^\Omega, \mu(A)=\frac{|A|}{|\Omega|}$ ist der Laplace-Wahrscheinlichkeitsraum.
		\end{bsp}
		\begin{satz}
			Seien $\mu_n$ Inhalte auf $\ms{C}$, und existiere $\mu(A)=\lim_{n\to\infty} \mu_n(A)$. Dann ist $\mu$ ein Inhalt. 
		\end{satz}
		\begin{bew}
			$A=\sum_{i=1}^k A_i$, $\mu(A)=\sum_{i=1}^k \mu_n(A_i)$, für $n\to\infty$ gehen beide Seiten gegen $\mu$, stimmt also. 
		\end{bew}
		\begin{satz}[Satz von Vitali-Hahn Saks:]
			Wenn $\ms{C}$ ein Sigmaring ist und $\mu_n$ endliche Maße und für alle $A\in\ms{C}: \mu(A)=\lim_{n\to\infty} \mu_n(A)$, dann ist $\mu$ auch ein Maß. 
		\end{satz}
		\begin{bew}
			noch nicht, Eigenschaften fehlen noch. 
		\end{bew}
		\begin{satz}
			Sei $\mu$ ein Inhalt/Maß auf einem Ring. Dann gilt:
			\begin{enumerate}
				\item Monotonie: 
				\[ A,B\in\ms{R}, A\subseteq B\Rightarrow \mu(A)\le \mu(B) \]
				\item Additionstheorem:
				\[ \mu(A\cup B)=\mu(A)+\mu(B)-\mu(A\cap B) \]
				\item Allgemeineres Additionstheorem:
				\begin{align*}
				\mu\left(\bigcup_{i=1}^n A_i\right)&=\sum_{J\subseteq\{1,...,n\}, J\neq \varnothing} (-1)^{|J|-1}\mu\left(\bigcap_{i\in J} A_i\right)\\
				&=\sum_{k=1}^n (-1)^{k-1} S_k \:\:\:\text{für } S_k=\sum_{i\le i_1<...<i_k\le n} \mu\left(\bigcap_{k=1}^n A_{i_k}\right)
				\end{align*}
				\item Subadditivität:
				\[ \mu\left(\bigcup_{i=1}^n A_i\right)\le \sum_{i=1}^n \mu(A_i) \]
			\end{enumerate}
			
		\end{satz}
		\begin{bew}
			\begin{enumerate}
				\item Es gilt:
				\[ B=A\cup (B\setminus A)\Rightarrow \mu(B)=\mu(A)+\mu(B\setminus A)\ge \mu(A) \]
				Nun ist außerdem mit $\mu(A)<\infty$:
				\[ \mu(B\setminus A)=\mu(B)-\mu(A) \]
				\item
				Für $A,B\in\ms{R}$:
				\[ \mu(B\setminus A)=\mu(B\setminus (A\cap B))=\mu(B)-\mu(A\cap B) (\text{ wenn }\mu(A\cap B)<\infty) \]
				\[ \Rightarrow \mu(A\cup B)=\mu(A)+\mu(B)-\mu(A\cap B) \]
				Außerdem (Zusatz für zwei Mengen):
				\[ \mu(A\cup B\cup C)=\mu((A\cup B)\cup C)=\mu(A)+\mu(B)+\mu(C)-\mu(A\cap B)-\mu(A\cap C)-\mu(B\cap C)+\mu(A\cap B\cap C) \]
				\item Es gilt:
				\[ A,B\in\ms{R}: \mu(A\cup B)=\mu(A\cup (B\setminus A))=mu(A)+\mu(B\setminus A)\le \mu(A)+\mu(B) \]
				\[ \Rightarrow \mu\left(\bigcup_{i=1}^n A_i\right)\le \sum_{i=1}^n \mu(A_i) \]
				\item Induktion (wahrscheinlich)
			\end{enumerate}
		\end{bew}
		\begin{satz}
			Sei $\mu$ Inhalt auf $\ms{R}$, $A_n, n\in\N, A\subseteq\ms{R}$, dann gilt:
			\[ \sum_{n\in\N} A_n\subseteq A\Rightarrow \sum_{n\in\N}\mu(A_n)\le \mu(A) \]
		\end{satz}
		\begin{bew}
			Es gilt:
			\[ \sum_{n=1}^N A_n\subseteq A\Rightarrow \mu\left(\sum_{n=1}^N A_n\right) \le \mu(A) \]
			\[ \Rightarrow \sum_{n=1}^N \mu(A_n)\le \mu(A) \]
			Für $n\to\infty$:
			\[ \sum_{n\in\N} \mu(A_n)\le \mu(A) \]
		\end{bew}
		
		\subsection{Folgerungen für Maße}
		
		\begin{satz}
			Sei $\mu$ ein Maß auf $\ms{R}$:
			\begin{enumerate}
				\item Stetigkeit von unten:
				\[ A_n\uparrow A, A_n, A\in\ms{R} \]
				\[ \Rightarrow \mu(A)=\lim_{n\to\infty} \mu(A_n) \]
				\item Stetigkeit von oben:
				\[ A_n\downarrow A, A_n, A\in\ms{R}\land \mu(A_1)<\infty \]
				\[ \Rightarrow \mu(A)=\lim_{n\to\infty} \mu(A_n) \]
			\end{enumerate}
		\end{satz}
			
		\begin{bew}
			\begin{enumerate}
				\item Sei $B_1=A_1$ und $B_n=A_n\setminus A_{n-1}$. Nun sind $B_n$ disjunkt und $A_n=\sum_{i=1}^n B_i$. Nun gilt:
				\[ \mu(A_n)\sum_{i=1}^n \mu(B_i) \]
				und:
				\[ A=\sum_{i=1}^\infty B_i \]
				\[ \Rightarrow \mu(A)=\sum_{i=1}^\infty \mu(B_i)=\lim_{n\to\infty}\sum_{i=1}^n \mu(B_i)=\lim_{n\to\infty} \mu(A_n) \]
				\item 
				\[ \mu(A)=\lim_{n\to\infty}\mu(A_n) \]
				\[ \mu(A_1\setminus A)=\lim_{n\to\infty} \mu(A_1\setminus A_n)=\lim_{n\to\infty} \mu(A_1)-\lim_{n\to\infty} \mu(A_n) \]
			\end{enumerate}
		\end{bew}
			
		\subsection{Eigenschaften von Maßen (Inhalten) auf Ringen(Semiringen)}
			\begin{satz}
				Sei $\mu$ ein Maß auf dem Ring $\ms{R}$, $A_n\uparrow A$, $A_n, A\in \ms{R}$. Dann gilt
				\[ \mu(A)=\lim_{n\to\infty} \mu(A_n) \]
				Entsprechendes für $A_n\downarrow A$.
			\end{satz}
			\begin{satz}
				Sei $\mu$ Inhalt auf Ring $\ms{R}$ ist genau dann ein Maß, wenn $\mu$ stetig von unten ist.
			\end{satz}
			\begin{bew}
				Seien $A_n, A\in\ms{R}$, $A=\sum_{n\in\N} A_n$, $A_n$ paarweise disjunkt. Sei 
				\[ B_n=\sum_{i=1}^n A_i \]
				Nun gilt $B_n\uparrow A$. $\mu$ ist nun stetig von unten, also
				\[ \mu(A)=\lim_{n\to\infty}\mu(B_n)=\lim_{n\to\infty}\mu(\sum_{i=1}^n A_i)=\lim_{n\to\infty} \sum_{i=1}^n \mu(A_i)=\sum_{i=1}^\infty \mu(A_i) \]\arge
			\end{bew}
			\begin{satz}
				Sei $\mu$ ein endlicher Inhalt auf einem Ring $\ms{R}$. Dann ist $\mu$ genau dann ein Maß, wenn er stetig von oben bei $\varnothing$ ist, also
				\[ A_n\downarrow \varnothing\Rightarrow \mu(A_n)\to 0. \]
			\end{satz}
			
			\begin{bew}
				Sei $A_n,A\in\ms{R}, A=\sum_{n=1}^\infty A_n$. \newline
				\zz: $\mu(A)=\sum_{n=1}^\infty \mu(A_n)$. \newline
				Nämlich:
				\[ A=\sum_{i=1}^n A_i\cup \sum_{i=n+1}^\infty \]
				\[ B_n:=\sum_{i=n+1}^\infty \Rightarrow B_n=A\setminus(\sum_{i=1}^n A_i)\in\ms{R} \]
				Nun gilt:
				\[ \mu(A)=\sum_{i=1}^n \mu(A_i)+\mu(B_n) \]
				Nun gilt:
				\[ \lim_{n\to\infty} B_n=\bigcap_{n\in\N} B_n=\bigcap_{n\in\N} A\setminus \left(\bigcup_{i=1}^n A_i\right)=A\setminus\left(\bigcup_{n\in\N}\bigcup_{i=1}^n A_i\right)=\varnothing \]
				Also $B_n\downarrow \varnothing$. Also:
				\[ \mu(A)=\lim_{n\to\infty} \left( \sum_{i=1}^n A_i + \mu(B_n)\right)=\sum_{i=1}^\infty + 0 \]
				\arge
			\end{bew}
			
			\begin{bem}
				Dieses Argument kann auch umgedreht werden. Dies werden wir später zumindest einmal benutzen.
			\end{bem}
			
			\begin{satz}
				Sei $\mu$ ein Maß auf dem Ring(Semiring) $\ms{R}$, $A_n, A\in\ms{R}$ mit
				\[ A\subseteq \bigcup_{n\in\N} A_n \]
				so gilt
				\[ \mu(A)\le \sum_{n\in\N}\mu(A_n).\:\:(\mu\text{ ist abzählbar-, bzw sigmasubadditiv}) \]
			\end{satz}
			
			\begin{bew}
				Sei $B_n=A\cap\bigcup_{i=1}^n A_i=\bigcup_{i=1}^n A\cap A_i$. Es gilt also $B_n\uparrow A$. Aus der endlichen Subadditivität erhalten wir:
				\[ \mu(B_n)\le\sum_{i=1}^n \mu(A_i\cap A)\le \sum_{i=1}^n\mu(A_i)\le \sum_{i=1}^\infty \mu(A_i) \]
				\[ \Rightarrow \mu(A)=\lim_{n\to\infty} \mu(B_n)\le\sum_{i=1}^\infty \mu(A_i) \]
			\end{bew}
			
			\begin{satz}
				Sei $\mu$ ein Maß auf dem Sigmaring $\ms{R}$ und $A_n$ eine Folge von Mengen aus $\ms{R}$. Dann gilt:
				\[ \limsup_{n\to\infty} A_n = \bigcap_{n\in\N}\bigcup_{k\ge n} A_k \]
			\end{satz}
			
			\begin{satz}
				Lemma von Borel Cantelli:\newline
				Sei $\mu$ ein Maß auf einem Sigamring $\ms{R}$. Ist $\sum_{n\in\N} \mu(A_n)<\infty$ für $A_n\in\ms{R}$, so gilt:
				\[ \mu(\limsup_{n\to\infty} A_n)=0 \]
			\end{satz}
			
			\begin{bew}
				Sei $\epsilon>0$ beliebig. Es gilt:
				\[ \mu(\limsup A_n)\le \mu\left(\bigcup_{k\ge n_0} A_k\right)\le \sum_{k\ge n_0} \mu(A_k)\le \epsilon \]
				\arge
			\end{bew}
			
			\begin{bem}
				Als Hausübung: Ist $\mu$ endliches Maß auf einem Sigmaring, so gilt
				\[ \mu(\limsup_{n\to\infty} A_n)\ge \limsup_{n\to\infty} \mu(A_n). \]
			\end{bem}
			
			\begin{bsp}[Additionstheorem]
				Die Anzahl der Permutationen von $n$ Elementen ohne Fixpunkt. 
				\[ \mathbb{P}(\text{kein Fixpunkt})=1-\mathbb{P}(\text{Fixpunkt})=1-\mathbb{P}\left(\bigcup A_i\right) \]
				mit $A_i=[i$ ist Fixpunkt $]$. 
				\[ \mathbb{P}\left(\bigcup A_i\right)=\sum_{i=1}^n\mathbb{P}(A_i)-\sum_{1\le i_1\le i_2\le n} \mathbb{P}(A_{i_1}\cap A_{i_2})+\sum \mathbb{P}(A_{i_1}\cap A_{i_2}\cap A_{i_2})-... \]
				Es gilt:
				\[ \mathbb{P}(A_i)=\frac{(n-1)!}{n!} \]
				\[ \mathbb{P}(A_i\cap A_0)=\frac{(n-2)!}{n!} \]
				\[ \mathbb{P}(A_{i_1}\cap ... \cap A_{i_k})=\frac{(n-k)!}{n!} \]
				Jetzt: (was auch immer $S_k$ ist...)
				\[ S_k=\frac{(n-k)!}{n!} \left(\begin{array}{c} n \\ k \end{array}\right)=\frac{1}{k!} \]
				Damit:
				\[ \mathbb{P}\left(\bigcup A_i\right)=\sum_{k=1}^n (-1)^{k-1} \frac{1}{k!} \]
				\[ \Rightarrow \mathbb{P}(\text{kein Fixpunkt})=1-\sum_{k=1}^n(-1)^{k-1} \frac{1}{k!}=\sum_{k=0}^n (-1)^k \frac{1}{k!}\stackrel{\rightarrow}{n\to\infty} \frac{1}{e} \]
			\end{bsp}
			
			\subsection{Bedingte Wahrscheinlichkeit}
			\begin{defi}
				Sei $(\Omega, \ms{S}, \P)$ ein Wahrscheinlichkeitsraum. Nun heißt $A,B\in\ms{S}$ Ereignisse. Gilt $\P(B)\neq 0$ so heißt 
				\[ \P(A|B):=\frac{\P(A\cap B)}{\P(B)} \]
				die bedingte Wahrscheinlichkeit. 
			\end{defi}
			
			\begin{defi}
				Ereignisse $A$ und $B$ heißen unabhängig, wenn
				\[ P(A\cap B)=\P(A)\P(B). \]
			\end{defi}
			
			\begin{defi}
				Allgemeiner heißen Ereignisse $A_1,...,A_n$ unabhängig, wenn
				\[ \P\left(\bigcap_{i=1}^n A_i\right) = \prod_{i=1}^n \P(A_i). \]
			\end{defi}
			
			\begin{defi}
				Ereignisse $A_1,...,A_n$ heißen paarweise unabhängig, wenn:
				\[ \forall i,j\in\{1,...,n\}: i\neq j\Rightarrow \P(A_i\cap A_j)=\P(A_i)\P(A_j). \]
			\end{defi}
			
			\begin{bem}
				Es gilt:
				\[ \P(A\cap B)=\P(B)\P(A|B)=\P(A)\P(B|A) \]
				und:
				\[ \P(A_i\cap...\cap A_n)=\P(A_1)\P(A_2|A_1)\P(A_3|A_1\cap A_2)...P(A_n|A_1\cap...\cap A_n) \]
				Dies ist das Multiplikationstheorem für Wahrscheinlichkeiten. 
			\end{bem}
			
			\begin{bsp}[Bedingte Wahrscheinlichkeiten (Multiplikationstheorem)]
				In einer Urne liegen zwei schwarze und drei weiße Kugeln. Es wird 3-mal ohne Zurücklegen gezogen, wobei das Ziehen der Laplace-Wahrscheinlichkeit folgt. Nun ist
				\[ \P(\text{Alle 3 Kugeln weiß})=\P(A_1\cap A_2\cap A_3) \]
				wobei $A_i=$ "`$i$-te Kugel ist weiß"'. Also
				\[ \P(A_1\cap A_2\cap A_3)=\P(A_1)\P(A_2| A_1)\P(A_3|A_1\cap A_2) \]
				mit
				\[ P(A_1)=\frac{3}{5} \]
				\[ P(A_2| A_1)=\frac{2}{4}=\frac{1}{2} \]
				\[ P(A_3|A_2\cap A_1)=\frac{1}{3} \]
				und damit 
				\[ \P(\text{Alle 3 Kugeln weiß})=\frac{1}{10} \]
			\end{bsp}
			
			\begin{bsp}
				Selbe Voraussetzungen wie im vorigen Beispiel. Nun ist
				\begin{align*}
				\P(\text{genau 2 Kugeln weiß})&=\P(\text{wws})+\P(\text{wsw})+\P(\text{sww})\\
				&=\P(A_1\cap A_2\cap A_3^c)+\P(A_1\cap A_2^c+A_3)+\P(A_1^c\cap A_2 \cap A_3)\\
				&=\frac{3}{5}\frac{2}{4}\frac{2}{3}+\frac{3}{5}\frac{2}{4}\frac{2}{3}+\frac{2}{5}\frac{3}{4}\frac{2}{3}=3\cdot\frac{12}{60}=\frac{3}{5}.
				\end{align*}
				Dieses Beispiel kann analog auf jede Anzahl an Kugeln fortgesetzt werden.
			\end{bsp}
			
			\begin{satz}[Borel-Cantelli II]
				Sei $(\Omega, \ms{S}, \P)$ ein Wahrscheinlichkeitsraum. Sei $A_n\in\ms{S}$ eine Folge unabhängiger Ereignisse. \newline
				Ist nun 
				\[ \sum_{n=0}^\infty \P(A_n)=\infty \]
				so folgt
				\[ \P(\limsup_{n\to\infty} A_n)=1 \]
			\end{satz}
			
			\begin{bew}
				Definition des $\limsup$ war:
				\[ \limsup_{n\to\infty} A_n=\bigcap_{n\in\N}\bigcup_{k\ge n} A_k \]
				und damit nach den de Morgan'schen Regeln:
				\[ (\limsup_{n\to\infty} A_n)^c=\bigcup_{n\in\N}\bigcap_{k\ge n} A_k^c \]
				Betrachten wir nun $\bigcap_{k\ge n} A_k^c$. Die $A_k^c$ sind nun auch unabhängig. (siehe Übung) Also:
				\[ \bigcap_{k\ge n} A_k^c = \lim_{N\to\infty} \bigcap_{k=n}^N A_k^c \]
				\[ \Rightarrow \P\left(\bigcap_{k\ge n} A_k^c\right)=\lim_{N\to\infty}\prod_{k=n}^\infty \P(A_k^c)=\prod_{k=n}^\infty \P(A_k^c)=\prod_{k=n}^\infty (1-\P(A_k)) \]
				mit $1+x\le e^x$ folgt
				\[ \prod_{k=n}^\infty (1-\P(A_k)) \le\prod_{k=n}^\infty e^{-\P(A_k)}=e^{-\sum_{k\ge n}^\infty \P(A_k)}=\lim_{n\to \infty} -e^{-n}=0 \]
				Damit:
				\[ \P\left(\bigcup_{n=1}^\infty\bigcap_{k=n}^\infty A_k\right)\le \sum_{n=1}^\infty \P\left(\bigcap_{k=n}^\infty A_k\right)=\sum_{n=1}^\infty 0=0 \]\arge
			\end{bew}
			
			\subsection{Der Fortsetzungssatz für Maßfunktionen}
			
			In diesem Abschnitt werden wir den folgenden Satz beweisen:
			\begin{satz}[Fortsetzungssatz für Maßfunktionen]
				Sei $\mu$ ein Maß auf einem Ring $\mf{R}$. Dann gilt:
				\begin{enumerate}
					\item $\mu$ kann zu einem Maß $\widetilde{\mu}$ auf dem erzeugten Sigmaring fortgesetzt werden. 
					\item Wenn $\mu$ sigmaendlich ist, dann ist $\widetilde{\mu}$ eindeutig bestimmt. 
				\end{enumerate}
				
			\end{satz}
			
			\begin{bem}
				Wir werden $\widetilde{\mu}$ im Folgenden immer mit $\mu$ bezeichnen, da es nicht wichtig ist, ob wir auf einem Ring oder auf dem erzeugten Sigmaring arbeiten. 
			\end{bem}
			
			\begin{bem}
				Die Motivation für diesen Satz ist das klassische Ausschöpfungs-, bzw Exhaustionsprinzip, das z.B. Archimedes und Eudoxos bearbeitet haben. Dabei wurde die Fläche eines Kreises durch Rechtecke approximiert. Damit ist ($A$ ist die Fläche des Kreises, $B$ die Fläche der Vierecke)
				\[ \mu^+(A)=\inf\{\mu(B):A\subseteq B, B\in\mf{R}\} \]
				\[ \mu^-(A)=\sup\{\mu(B):B\subseteq A, B\in\mf{R}\} \]
				wenn $\mu^+(A)=\mu^-(A)$, dann ist $A$ messbar (im Sinn von Jordan). Dann $\mu^*$ das Jordon-Maß. 
				\begin{align*}
				\mu^*(A) &= \inf\left(\sum_{n\in\N}\mu(B_n)\right), B_n\in\mf{R}, A\subseteq \bigcup_{n\in\N} B_n\\
				&=\inf\{\sum_{n\in\N}\mu(B_n): B_n\in\mf{R}, A\subseteq \sum_{n\in\N} B_n\}
				\end{align*}
				Die letzte Gleichheit folgt durch Zeigen von $\le$ und $\ge$. 
				
			\end{bem}
			
			\begin{defi}
				Das Maß von einem Maß $\mu$ erzeugte Maß
				\[ \mu^*(A) =\inf\{\sum_{n\in\N}\mu(B_n): B_n\in\mf{R}, A\subseteq \sum_{n\in\N} B_n\} \]
				heißt äußeres Maß oder Jordan-Maß. Hierbei wird 
				\[ \inf\varnothing=\infty \]
				gesetzt. 
			\end{defi}
			
			\begin{defi}
				Ist $\mu(\Omega)<\infty$, so ist 
				\[ \mu_*(A)=\mu(\Omega)-\mu^*(A^c) \]
				das innere Maß. 
			\end{defi}
			
			\begin{defi}[vorläufige Definition]
				$A$ heißt messbar, falls
				\[ \forall E\in\mf{R}: \mu(E)=\mu^*(E\cap A)+\mu^*(E\setminus A). \]
			\end{defi}
			
			\begin{defi}
				$A$ heißt messbar, wenn
				\[ \forall B\subseteq \Omega: \mu^*(B)=\mu^*(B\cap A)+\mu^*(B\setminus A). \]
			\end{defi}	
			
			\begin{satz}[Eigenschaften von äußeren Maßfunktionen]
				Sei $\mu$ ein Maß und $\mu^*$ das von $\mu$ erzeugte äußere Maß. Dann gilt:
				\begin{enumerate}
					\item $\mu^*(A)\ge 0$
					\item $\mu^*(\varnothing)=0$
					\item Monotonie: 
					\[ A\subseteq B\subseteq \Omega \Rightarrow \mu^*(A)\le \mu^*(B) \]
					\item Sigmasubadditivität: 
					\[ A\subseteq \bigcup_{n\in\N} A_n\subseteq \Omega \]
					\[ \Rightarrow \mu^*(A)\le \sum_{n\in\N} \mu^*(A_n) \]
				\end{enumerate}
			\end{satz}
			
			\begin{defi}
				Eine Funktion $\mu^*: 2^\Omega \to [0,\infty]$ heißt eine äußere Maßfunktion, wenn sie die Eigenschaften 1.-4. besitzt.
			\end{defi}	
			
			\begin{bem}
				Will man zeigen, dass $\mu^*$ ein äußeres Maß ist, so muss man nur 1.,2. und 4. zeigen, 3. folgt dann automatisch. 
			\end{bem}
			
			\begin{bew}
				Eigenschaften 1. und 2. sind klar. Bleibt also noch 4. zu zeigen, 3. folgt ja automatisch. \newline
				Sei also $A\subseteq \bigcup_{n\in\N} A_n$. Zu zeigen ist nun, dass 
				\[ \mu^* (A)\le \sum_{n\in\N} \mu^*(A_n) \]
				wenn $\sum_{n\in\N} \mu^*(A_n)=\infty$, so sind wir fertig.\newline
				Sei also $\sum_{n\in\N} \mu^*(A_n)<\infty$. Dann ist
				\[ \mu^*(A_n)=\inf\{\sum_{k\in\N}\mu(B_k): A_n\subseteq \bigcup B_k, B_k\in\mf{R}\} \]
				Sei $\epsilon >0$. Für $B_{nk}\in\mf{R}: A_n\subseteq\bigcup_{k\in\N} B_{nk}$ und $\sum_{k\in\N} \mu(B_{nk})\le \mu^*(A_n)+\frac{\epsilon}{2}$. Nun ist
				\[ \bigcup_{n\in\N} A_n\subseteq \bigcup_{n\in\N}\bigcup_{k\in\N} B_{nk} \]
				und damit
				\[ \mu^*\left(\bigcup_{n\in\N} A_n\right)\le \sum_{n\in\N}\sum_{k\in\N} \mu(B_nk) \le \sum_{n\in\N} (\mu^*(A_n)+\frac{\epsilon}{2^n})=\sum_{n\in\N} \mu^* (A_n)+\epsilon \]
				\[ \Rightarrow \mu\left(\bigcup_{n\in\N}A_N\right)\le\sum_{n\in\N} \mu^* (A_n) \]
				\arge
			\end{bew}
			
			\begin{bsp}
				Sei $|\Omega|\ge 3$ und 
				$$\mu^*(A)=\left\{\begin{array}{l}
					0: A=\varnothing\\
					1: A\notin\{\varnothing, \Omega\}, A\subseteq\Omega\\
					2: A=\Omega
				\end{array}\right. $$
			\end{bsp}
			
			\begin{defi}
				$A\subseteq \Omega$ heißt messbar ($\mu^*$-messbar), wenn 
				\[ \forall B\subseteq \Omega: \mu^*(B)=\mu^*(B\cap A)+\mu^*(B\cap A^c). \]
			\end{defi}
			
			\begin{bem}
				Um die Messbarkeit von $A$ zu zeigen, genügt es zu zeigen, dass
				\[ \mu^*(B)\ge \mu^*(B\cap A)+\mu^*(B\cap A^c), \]
				da die Ungleichung "`$\le$"' trivialerweise immer erfüllt ist. 
			\end{bem}
			
			\begin{defi}
				$m_{\mu^*}$ bezeichnet das System aller $\mu^*$-messbaren Mengen. Ist klar, um welches Maß $\mu^*$ es sich handelt (oder das egal ist), so schreiben wir einfach $m$.
			\end{defi}
			
			\begin{satz}
				\begin{enumerate}
					\item $m$ ist eine Sigmaalgebra, $\mu^*|_m$ ein Maß.
					\item Wenn $\mu^*$ von einem Maß $\mu$ auf einem Ring $\mf{R}$ erzeugt wird und $\mu^*(B)=\mu(B)$, so folgt $\mf{R}\subseteq m$.
				\end{enumerate}  
			\end{satz}
			
			\begin{bew}
				Wir beweisen zunächst 2.:\newline
				Sei $B\subset\Omega, A\in\mf{R} B_n\in\mf{R}, B\subseteq \bigcup_{n\in\N} B_n, \mu^*(B)<\infty$. Dann ist
				\begin{align*}
				\sum_{n\in\N}\mu(B_n)&=\sum_{n\in\N}\mu\left((B_n\cap A)\cup(B_n\cap A^c)\right)\\
				&=\sum_{n\in\N} \left(\mu(B_n\cap A)+\mu(B_n\setminus A)\right)\\
				&=\sum_{n\in\N}\mu(B_n\cap A)+\sum_{n\in\N} \mu(B_n\setminus A)\\
				&\ge \mu^*(B\cap A)+\mu^*(B\cap A^c)
				\end{align*}
				\[ \Rightarrow \mu^*(B)\ge \mu^*(B\cap A)+\mu^*(B\setminus A) \]
				\arge
				Sei nun $A\in \mf{R}, A\subseteq \bigcup A_n, A_n\in\mf{R}$. 
				\[ \mu(A)\le \sum \mu(A_n) \]
				wurde schon gezeigt. Sei jetzt $A_1=A$, $A_n=\varnothing$ für $n>1$. Dann folgt 
				\[ \mu^*(A)\ge\mu(A), \]
				A ist also messbar.\newline\newline
				Für 1. erste Behauptung: $m$ ist Algebra und $\mu^*|_m$ ist additiv. Wir wollen zeigen:
				\[ A_1,A_2\text{ messbar}\Rightarrow A_1\cup A_2 \text{ messbar} \]
				\[ A \text{ messbar}\Rightarrow A^c \text{ messbar} \]
				Das zweite folgt direkt daraus, dass ${A^c}^c=A$ und die Definition von "`messbar"' diesbezüglich symmetrisch ist. \newline
				Für das erste sei $B\subseteq\Omega$. Nun ist $A_1$ messbar, also
				\[ \mu^*(B)=\mu^*(B\cap A_1)+\mu^*(B\cap A_1^c) \]
				und mit
				\[ \mu^*(B\cap A_1)=\mu^*(B\cap A_1\cap A_2)+\mu^*(B\cap A_1\cap A_2^c) \]
				\[ \mu^*(B\cap A_1^c)=\mu^* (B\cap A_1^c\cap A_2)+\mu^*(B\cap A_1^c\cap A_2^c) \]
				ergibt sich:
				\begin{align*}
				\mu^*(B)&=\mu^*(B\cap A_1\cap A_2)+\mu^*(B\cap A_1\cap A_2^c)+\mu^* (B\cap A_1^c\cap A_2)+\mu^*(B\cap A_1^c\cap A_2^c)\\
				&\ge \mu^*\left((B\cap A_1\cap A_2)\cup(B\cap A_1\cap A_2^c)\cup (B\cap A_1^c\cap A_2)\right)+ \mu^*(B\cap (A_1\cup A_2)^c)\\
				&=\mu^*(B\cap(A_1\cup A_2))+\mu^*(B\cap (A_1\cup A_2)^c)
				\end{align*}
				Damit ist $m$ tatsächlich eine Algebra.\newline
				Um nachzuweisen, dass $\mu^*|_m$ additiv ist, seien $A_1, A_2\in m$, $A_1\cup A_2=\varnothing$. Über die Messbarkeit von $A_1$ erhalten wir:
				\[ \mu^*(A_1\cup A_2)=\mu^*((A_1\cup A_2)\cap A_1)+\mu^*((A_1\cup A_2)\cap A_1^c)=\mu^*(A_1)+\mu^*(A_2) \]
				\arge
				Nun bleibt noch zu zeigen, dass $m$ Sigmaalgebra ist, seien also $A_n\in m, A_n\text{ disjunkt}, B\subseteq \Omega$.\newline
				\zz:
				\[ \mu^*(B)\ge \mu^*\left(B\cap \bigcup_{n\in \N} A_n\right)+\mu^*\left(B\setminus \bigcup_{n\in\N} A_n\right) \]
				Wir wissen schon:
				\begin{align*}
				\mu^*(B)=\mu^*\left(B\cap \bigcup_{n=1}^N A_n\right)+\mu^*\left(B\setminus\bigcup_{n=1}^N A_n\right)\\
				&\ge \mu^*\left(B\cap \bigcup_{n=0}^N A_n\right)+\mu^*\left(B\setminus\bigcup_{n\in\N} A_n\right)\\
				&=\sum_{n=1}^N\mu^*(B\cap A_n)+\mu^*\left(B\setminus\bigcup_{n\in\N} A_n\right)\\
				\end{align*}
				Für $n\to\infty$ erhalten wir also
				\[ \mu^*(B)\ge \sum_{n\in\N} \mu^*(B\cap A_n)+\mu^*\left(B \setminus \bigcup_{n\in\N} A_n\right)\ge\mu^*\left(\bigcup_{n\in\N} (B\cap A_n)\right)+\mu^*\left(B\setminus\bigcup_{n\in\N} A_n\right) \]
				\arge
			\end{bew}
			
			\begin{bem}
				Der erste Teil des Fortsetzungssatzes ist damit bewiesen. Bleibt also noch der folgende Satz zu zeigen:
			\end{bem}
			
			\begin{satz}
				Ist $\widetilde{\mu}$ eine Fortsetzung von $\mu$ auf $\mf{R}_\sigma(\mf{R})$ ist, dann gilt 
				\[ \widetilde{\mu}=\mu^*|_{\mf{R}_\sigma} \]
			\end{satz}
			
			\begin{satz}
				Ist $\mu$ auf $\mf{R}$ sigmaendlich, dann auch auf dem erzeugten Sigmaring.
			\end{satz}
			
			\begin{bew}
				\zz:
				\[ \mf{R}^*=\{A\subseteq \Omega: A \text{ ist sigmaendlich}\}=\{A\subseteq \exists B_1\in\mf{R}, n\in\N: \mu(B_n)<\infty \land A\subseteq\bigcup B_n\} \]
				ist Sigmaring. \newline
				\begin{itemize}
					\item $A,B\in\mf{R}^*\Rightarrow A\setminus B\in\mf{R}^*$: trivial
					\item $A_n\in\mf{R}^*, n\in\N\Rightarrow \bigcup A_n\in\mf{R}^*$:\newline
					Sei $A_n\subseteq \bigcup B_{nk}, B_{k}\in\mf{R}, \mu(B_{nk})<\infty$. Dann ist
					\[ \bigcup A_n\subseteq \bigcup_{n}\bigcup_k B_{nk} \]
					und damit folgt die Behauptung.
				\end{itemize}
			\end{bew}
			
			\begin{satz}
				Für $A\in\mf{R}_\sigma(\mf{R}): \widetilde{A}\le \mu^*(A)$
			\end{satz}
			
			\begin{bew}
				Sei $A\in\mf{R}, A\subseteq \bigcup B_n, B_n\in\mf{R}$. Nun gilt:
				\[ \sum_{n\in\N}\mu(B_n)=\sum_{n\in\N} \widetilde{\mu}(B_n)\ge \widetilde{\mu}\left(\bigcup B_N\right)\ge \widetilde{\mu}(A) \]
				Nimmt man das $\inf$ über alle $(B_n)_{n\in\N}$, so erhält man $\mu^*(A)$.
			\end{bew}
			
			\begin{satz}
				$\widetilde{\mu}(A)=\mu^*(A)$ (siehe oben)
			\end{satz}
			
			\begin{bew}
				Ist $A$ sigmaendlich, so folgt:
				\[ \exists B_n\in\mf{R}, n\in\N, \mu(B_n)<\infty, a\subseteq \bigcup B_n, \]
				wobei wir die $B_n$ oBdA als disjunkt annehmen können, da wir sie notfalls disjunkt machen können.\newline
				Nun ist
				\[ \widetilde{\mu}(A)=\widetilde{\mu}\left(A\cap\bigcup_{n\in\N}B_n\right)=\widetilde{\mu}\left(\bigcup_{n\in\N}A\cap B_n\right)=\sum_{n\in\N} \widetilde{\mu}(A\cap B_n) \]
				Nun zeigen wir:
				\[ \widetilde{\mu}(A\cap B_n)=\mu^*(A\cap B_n), \]
				dann können wir die obere Gleichungskette nach hinten durchlaufen und sind fertig.\newline
				Also: Wir wissen: 
				\[ \widetilde{\mu}(A\cap B_n)\le \mu^*(A\cap B_n) \]
				\[ \widetilde{\mu}(A^c\cap B_n)\le\mu^*(A^c\cap B_n) \]
				Außerdem, da $A$ messbar:
				\[ \mu(B_n)=\mu^*(B_n)=\mu^*(A^c\cap B_n)+\mu(A\cap B_n)\ge \widetilde{\mu}(A^c\cap B_n)+\widetilde{\mu}(A^c\cap B_n)=\widetilde{\mu}(B_n)=\mu(B_n), \]
				womit für $\ge$ auch $=$ folgt und $\widetilde{\mu}(A\cap B_n)=\mu^*(A\cap B_n)$ bewiesen ist. Damit folgt also auch:\[ \widetilde{\mu}=\mu^*|_{\mf{R}_\sigma(\mf{R})} \]
			\end{bew}
			
			\begin{bem}
				Nun ist der Fortsetzungssatz für Maßfunktionen vollständig bewiesen.
			\end{bem}
			
			\subsection{Zusammenhang zwischen dem Maß auf dem Ring und dem Maß auf dem Sigmaring}
			\begin{satz} [Approximationstheorem I]
				Sei $\mu$ ein sigmaendliches Maß auf einem Ring $\mf{R}$. Sei $A\in\mf{R}_\sigma(\mf{R}), \mu(A)<\infty$. Dann gilt
				\[ \forall \epsilon>0: \exists B\in\mf{R}: \mu(A\Delta B)<\epsilon \]
			\end{satz}
			
			\begin{bew}
				Mit $\mu(A)<\infty$ und
				\[ \mu(A)=\mu^*(A)=\inf\left\{\sum_{n\in\N}\mu(B_n): B_n\in\mf{R}, A\subseteq \bigcup_{n\in\N} B_n\right\} \]
				folgt: Wir wählen ein $(B_n)$, sodass $\sum_{n\in\N} \mu(B_n)\le \mu(A)+\frac{\epsilon}{2}$. Das geht, weil wir ja beliebig nahe an das Infimum herankommen können. Wir wählen nun $N$ so, dass $\sum_{n>N}\mu(B_n)<\frac{\epsilon}{2}$. Sei weiters
				\[ B:=\bigcup_{n=1}^N B_n\in\mf{R}. \]
				Dann folgt:
				\[ \mu(A\Delta B)=\mu (A\setminus B)+\mu(B\setminus A) \]
				Außerdem:
				\[ A\setminus B\subseteq\bigcup_{n\in\N} B_n\setminus \bigcup_{n=1}^N B_n\subseteq \bigcup_{n>N}B_n. \]
				Damit gilt:
				\[ \mu(A\setminus B)\le\sum_{n>N} \mu(B_n)<\frac{\epsilon}{2} \]
				\[ \mu(B\setminus A)\le \mu\left(\left(\bigcup_{n\in\N} B_n\right)\setminus A\right)=\mu\left(\bigcup_{n\in\N} B_n\right)-\mu(A)\le \sum_{n\in\N} \mu(B_n)-\mu(A)<\frac{\epsilon}{2} \]
				und wir sind fertig. 
			\end{bew}
			
			\begin{bem}
				Es gilt auch 
				\[ |\mu(A)-\mu(B)|\le \mu(A\Delta B) \]
			\end{bem}
			
			\begin{bem}
				Wir nehmen nun an, dass $\Omega\in\mf{R}_\sigma(\mf{R})$, der erzeugte Sigmaring ist also schon eine Sigmaalgebra. 
			\end{bem}
			
			
			\begin{defi}
				Sei $(\Omega, \mf{S}, \mu)$ Maßraum. Ist $\mu(A)=0$, so heißt $A$ Nullmenge.
				
			\end{defi}
			
			\begin{satz}
				Ist $A$ messbar, so kann man $A$ schreiben als Vereinigung einer Menge aus dem Sigmaring und einer Nullmenge, also
				\[ A=F\cup N, F\in\mf{R}_\sigma, N\subseteq M\in\mf{R}_\sigma:\mu(M)=0 \]
			\end{satz}
			
			\begin{bew}
				Sei $A\subseteq \Omega$. Mit 
				\[ \mu^*(A)=\inf\left\{\sum_{n\in\N} \mu(B_n), B_n\in\mf{R}, A\subseteq \bigcup_{n\in\N}B_n \right\} \]
				erhalten wir über
				\[ \sum_{n\in\N} \mu(B_n)\ge \mu\left(\bigcup_{n\in\N}B_n\right) \]
				das folgende:
				\[ \mu^*(A)\ge\inf\left\{\mu(B):B\in\mf{R}_\sigma, A\subseteq B\right\} \]
				und damit folgt
				\[ \mu^*(A)=\inf\left\{\mu(B):B\in\mf{R}_\sigma, A\subseteq B\right\} \]
				Wir nehmen nun ein $(C_n)$ mit $C_n\in\mf{R}, \bigcup_{n\in\N}=\Omega, C_n$ disjunkt und $\forall n\in\N: \mu(C_n)<\infty$. Ist $A\in m_{\mu^*}$ messbar, $\mu^*(A\cap C_n)<\infty$, $\mu^*(A\cap C_n)=\inf\{\mu(B):B\in\mf{R}_\sigma, A\cap C_n\subseteq B\}$, dann wählen wir ein $B_k\in\mf{R}_\sigma$, sodass
				\[ A\cap C_n\subseteq B_k, \mu(B_k)\le \mu^*(A\cap C_n)+\frac{1}{k}. \]
				Sei 
				\[ D_n=\bigcap_{k\in\N} B_k\]
				also 
				\[ A\cap C_n\subseteq D_n, \mu(D_n)=\mu^*(A\cap C_n) \]
				Analog: $E_n\in\mf{R}_\sigma, A^c\cap C_n\subseteq E_n, \mu(E_n)=\mu^*(A^c\cap C_n)$:
				\[ \mu(C_n)=\mu^*(C_n\cap A)+\mu(C_n\cap A^c)=\mu(E_n)+\mu(D_n) \]
				oBdA: $E_n,D_n\subseteq C_n$. Nun ist 
				\[ \mu(D_n)=\mu(C_n)-\mu(D_n)=\mu(C_n\subseteq D_n) \]
				Über
				\[ D_n\supseteq A\cap C_n \land F_n:=C_n\setminus E_n\subseteq A\cap C_n\]
				\[ \mu(D_n\setminus F_n)=\mu(D_n)-\mu(F_n)=0 \]
				\[ F_n\subseteq A\cap C_n\subseteq F_n\cup (D_n\setminus F_n) \]
				erhalten wir
				\[ \bigcup_{n\in\N} F_n\subseteq A\subseteq \bigcup_{n\in\N} F_n\cup \bigcup_{n\in\N}(D_n\setminus F_n) \]
				Wir betrachten
				\[ \mu\left(\bigcup_{n\in\N} (D_n\setminus F_n)\right)\le \sum_{n\in\N} \mu(D_n\setminus F_n)=0 \]
				Nun können wir $A$ schreiben als
				\[ A=F\cup N, F\in\mf{R}_\sigma, N\subseteq M\in\mf{R}_\sigma:\mu(M)=0 \]
			\end{bew}
			
			\begin{defi}
				Ein Maßraum $(\Omega,\mf{S},\mu)$ heißt vollständig, wenn 
				\[ A\in\mf{S}, \mu(A)=0, B\subseteq A\Rightarrow B\in\mf{S} \]
			\end{defi}
			
			\begin{defi}
				Sei $(\Omega,\mf{S}, \mu)$ Maßraum. Mit
				\[ \overline{\mf{S}}:=\{A\cup N, A\in\mf{S}, \exists M\in\mf{S}: N\subseteq M, \mu(B)=0 \} \]
				und
				\[ \overline{\mu}(A\cup N)=\mu(A) \]
				heißt der vollständige Maßraum $(\Omega, \overline{\mf{S}}, \overline{\mu})$ die Vervollständigung von $(\Omega, \mf{S}, \mu)$.
			\end{defi}
			
			\begin{satz}
				Ist $\mu*$ das von einem Maß $\mu$ auf dem Ring $\mf{R}$ erzeugte äußere Maß, so ist ein $A\subseteq \Omega$ messbar genau dann, wenn
				\[ \forall B\in\mf{R}: (\mu^*(B)=)\mu(B)=\mu^*(B\cap A)+\mu^*(B\setminus A). \]
				Ist zusätzlich $\mu(\Omega)<\infty$ ($\mu^*(\Omega)<\infty$), dann ist $A$ messbar, wenn 
				\[ \mu(\Omega)=\mu^*(A)+\mu^*(A^c). \]
			\end{satz}
			
			\begin{bew}
				Eine Richtung ist klar. \newline
				Für die andere sei $C\subseteq \Omega$. \newline
				\zz:
				\[ \mu^*(C)=\mu^*(C\cap A)+\mu^*(C\setminus A), \]
				also nur die Richtung $\ge$, $\le$ wird durch die Subadditivität schon garantiert. Sei nun $(B_n)$ eine Überdeckung von $C$, $C\subseteq \bigcup_{n\in\N} B_n$. Nun gilt:
				\[ \sum_{n\in\N}\mu(B_n)=\sum_{n\in\N} \mu^*(B_n\cap A) + \sum_{n\in\N}\mu^*(B_n\setminus A)\ge \mu^*(C\cap A)+\mu^*(C\setminus A) \]
				Durch Infimumbildung ergibt sich die Behauptung.   \newline
				Sei nun $\mu(\Omega)<\infty$. Sei $E\in\mf{R}$, dann ist
				\[ \mu^*(A)=\mu^*(A\cap E)+\mu^*(A\cap E^c) \]
				\[ \mu^*(A^c)=\mu^*(A^c\cap E)+\mu^*(A^c\cap E^c) \]
				und somit 
				\[ \mu(\Omega)=\mu^*(A)+\mu^*(A^c)=\mu^*(A\cap E)+\mu^*(A\cap E^c)+\mu^*(A^c\cap E)+\mu^*(A^c\cap E^c)\ge \mu(E)+\mu(E^c)=\mu(\Omega) \]
				Damit folgt statt $\ge$ Gleichheit. 
			\end{bew}
			
		\subsection{Maße auf $(\R, \mf{B})$}
			
			Die Frage, die sich stellt ist: Ist $\mu^*$ auf $\R$ frei definiert, wann gilt $\mf{B}\subseteq \mf{M}_{\mu^*}$?
			
			\begin{defi}
				Seien $A,B\in\R$. Dann ist der Abstand 
				\[ d(A,B):=\inf\{|x-y|, x\in A, y\in B\}. \]
				Ein äußeres Maß $\mu^*$ heißt arithmetisch, wenn 
				\[ \forall A,B\in\R: d(A,B)>0\Rightarrow \mu^*(A\cup B)=\mu^*(A)+\mu^*(B) \]
			\end{defi}
			
			\begin{satz} [Satz von Carathéodory]
				$\mf{B}\subseteq \mf{M}_{\mu^*}$ genau dann, wenn $\mu^*$ arithmetisch ist.
			\end{satz} 
			
			\begin{bew}
				Dieser Satz erfordert einige Trickserei und wird hier nicht bewiesen.
			\end{bew}
			
		\subsection{Maße auf $(\R, \mf{B})$, zweiter Anlauf}
			Im folgenden ist immer $\mf{T}:=\{(a,b], a\le b, a,b\in\R\}$.
			
			\begin{satz}
				$\mu$ ist genau dann endliches Maß auf $\mf{T}$, wenn
				\[ \forall x\in\R\exists\delta(x)>0: \mu((x-\delta(x),x])<\infty \]
			\end{satz}	
			
			\begin{bew}
				In der Übung.
			\end{bew}
			
			\begin{defi}
				$\mu$ auf $(\R, \mf{B})$ heißt Lebesgue-Stieltjes Maß, oder lokalendlich, wenn jede beschränkte Borelmenge endliches Maß hat. 
			\end{defi}
			
			\begin{bem}
				Dazu muss man ein Maß finden, dass für alle $a<b$ $\mu((a,b])$ festlegt. Dies ist nicht ganz frei möglich, die Additivität muss erfüllt werden, also 
				\[ \mu((a,c])=\mu((a,b])+\mu((b,c]). \]
				
				Wir beginnen dazu mit einem Spezialfall, dass $\mu(\R)<\infty$:
			\end{bem}
			
			\begin{bsp}
				Sei
				\[ F(x)=\mu((-\infty,x])<\infty \]
				dann ist für $a<b$:
				\[ (-\infty,a]\cup (a,b]=(-\infty,b] \]
				\[ \mu((-\infty,a])+\mu((a,b])=\mu((-\infty,b]) \]
				\[ \Rightarrow \mu((a,b])=F(b)-F(a) \]
			\end{bsp} 
			
			\begin{defi}
				$F:\R\to\R$ heißt Verteilungsfunktion von $\mu$, wenn $\mu((a,b])=F(b)-F(a)$. 
			\end{defi}
			
			\begin{bem}
				\[ F(x)=\mu((0,x]), x\ge 0, \]
				damit:
				\[ \mu((a,b])=F(b)-F(a) \]
				und $F(0)=0$. \newline
				für
				\[ \mu((x,0])=F(0)-F(x) \]
				\[ \Rightarrow F(x)=-\mu((x,0]), \]
				eine Verteilungsfunktion muss also die Form
				\[ F(x)=\left\{\begin{array}{l}
				\mu((0,x]): x\ge 0\\
				-\mu((x,0]): x<0
				\end{array}\right. \]
				dies funktioniert, siehe Aufgaben (\zz $\mu((a,b])=F(b)-F(a)$). Dies fassen wir im folgenden Satz zusammen:
			\end{bem}
			
			\begin{satz}
				Zu jeder Lebesgue-Stieltjes Maßfunktion gibt es eine Verteilungsfunktion. Diese ist bis auf eine additive Konstante eindeutig bestimmt. 
			\end{satz}
			
		\subsection{Ergänzungen zu bedingten Wahrscheinlichkeiten}
			\begin{satz}[Satz von der vollständigen Wahrscheinlichkeit]
				Sei $(\Omega, \mf{S}, \P)$ ein Wahrscheinlichkeitsraum.
				Sei dann $(B_i, i\in I)$ eine Partition, $I$ höchstens abzählbar mit $B_i\in\mf{S}, \P(B_i)>0, \sum_{i\in I}B_i=\Omega$ und $A\in\mf{S}$. Dann ist
				\[ \P(A)=\sum_{i\in I} \P(B_i)\P(A|B_i). \]
			\end{satz}
			
			\begin{bew}
				Es gilt:
				\[ \P(A)=\P(A\cap \Omega)=\P\left( A\cap\bigcup_{i\in I}B_i\right)=\P\left(\bigcup_{i\in I} A\cap B_i\right)=\sum_{i\in I} \P(A\cap B_i)=\sum_{i\in I} \P(B_i)\P(A|B_i) \]
			\end{bew}
			
			\begin{satz}[Satz von Bayes]
				Sei $(\Omega, \mf{S}, \P)$ ein Wahrscheinlichkeitsraum.
				Sei wieder $(B_i, i\in I)$ eine Partition, $I$ höchstens abzählbar mit $B_i\in\mf{S}, \P(B_i)>0, \sum_{i\in I}B_i=\Omega$ und $A\in\mf{S}$. Zusätzlich zu vorher gelte $\P(A)>0$. Dann gilt:
				\[ \P(B_i|A)=\frac{\P(A\cap B_i)}{\P(A)}=\frac{\P(B_i)\P(A|B_i)}{\P(A)}=\frac{\P(B_i)\P(A|B_i)}{\sum_{j\in I}\P(B_j)\P(A|B_j)} \]
			\end{satz}
			
			\begin{defi}
				Die $\P(B_i)$ in den Sätzen vorher heißen a-priori Wahrscheinlichkeiten, $\P(B_i|A)$ die a-posteriori Wahrscheinlichkeiten. 
			\end{defi}
			
			\begin{bsp}
				Es gibt vier Blutgruppen, A, B, AB, 0. Die Blutgruppe der Frau ist A, die des Sohnes ist 0. Wie ist die Wahrscheinlichkeit für die Blutgruppe des Mannes?\newline
				Mit zusätzlichem Wissen über Genetik, kann man über die Wahrscheinlichkeiten $p_a,p_b,p_0$ für das Auftreten der Allele $a,b,0$ die Wahrscheinlichkeit der Blutgruppen ausrechnen. In der Bevölkerung haben Blutgruppe 0 40\% der Bevölkerung, Blutgruppe A 47\%, B 9\% und AB 4\%. Damit erhalten wir:
				\[ 0.4=p_0^2 \]
				\[ 0.47=p_a^2+2p_ap_0 \]
				\[ 0.09=p_b^2+2p_bp_0 \]
				\[ 0.04=2p_ap_b \]
				Eine gute Approximation ist
				\[ p_a\approx \frac{9}{30}, \:\:\:\:\:p_b\approx \frac{2}{30}, \:\:\:\:\: p_0\approx\frac{19}{30}. \]
				Damit erhält man:
				\[ \P(\text{Sohn 0|0})=1 \]
				\[ \P(\text{Sohn 0|A})=\frac{1}{2} \]
				\[ \P(\text{Sohn 0|B})=\frac{1}{2} \]
				und 
				\[ \P(\text{0|Sohn 0})=\frac{p_0^2}{p_0^2+p_ap_0+p_bp_0}=p_0, \]
				genauso für A und B, also ist die Wahrscheinlichkeit für Blutgruppe 0 am größten. 
			\end{bsp}
			
		\subsection{Eigenschaften von Verteilungsfunktionen}
			\begin{satz}
				Sei $F:\R\to\R$ eine Verteilungsfunktion. Dann gilt:
				\begin{enumerate}
					\item Monotonie:
					\[ a\le b\Rightarrow F(a)\le F(b) \]
					\item Rechtsstetigkeit:
					\[ b_n\downarrow b\Rightarrow F(b_n)\downarrow F(b) \]
				\end{enumerate}
			\end{satz}
			
			\begin{bew} $\text{  }$ %newline
				
				\begin{enumerate}
					\item trivial
					\item Sei $a\le b$, $b_n\downarrow b$, also
					\[ (a,b_n]\downarrow (a,b] \]
					und über die Stetigkeit von oben von Maßen
					\[ \mu((a,b_n])\to \mu((a,b]) \]
					\[ \Rightarrow F(b_n)-F(a)\to F(b)-F(a) \]
				\end{enumerate}
			\end{bew}
			
			\begin{satz}
				Sei $F:\R\to\R$ nichtfallend und rechtsstetig. Dann ist durch 
				\[ \mu_F((A,b]):=F(b)-F(a) \]
				ein Maß auf $\mf{T}=\{(a,b], a\le b, a,b,\in\R\}$ definiert.
			\end{satz}
			
			\begin{bew}$\text{  }$
				\begin{enumerate}
					\item $\mu_F$ ist Inhalt:\newline
					\[ \mu_F(\varnothing)=\mu_F((a,a])=F(a)-F(a)=0 \]
					und
					\[ \mu_F((a,b])\le 0 \]
					folgt sofort. Für die Additivität benutzen wir, dass $\mf{T}$ ein Semiring im engeren Sinn ist. Ist außerdem die Vereinigung zweier Intervalle wieder ein Intervall, so hat die Vereinigung die Form:
					\[ (a,b]\cup (b,c]=(a,c], \]
					also
					\[ \mu_F((a,b])+\mu_F((b,c])=F(b)-F(a)+F(c)-F(b)=F(c)-F(a)=\mu_F((a,c]). \]
					\item Sigmaadditivität:\newline
					Sei also 
					\[ (a,b]=\sum_{n\in\N} (a_n,b_n]. \]
					Dann wissen wir schon, dass, da $\mu_F$ Inhalt ist, gilt:
					\[ \mu_F((a,b])\ge\sum_{n\in\N} \mu_F((a_n,b_n]). \]
					\zz:
					\[ \mu_F((a,b])\le\sum_{n\in\N} \mu_F((a_n,b_n]) \]
					Sei $a'>a$, $F(a')\le F(a)+\epsilon$ und $b_n'>b_n$, $F(b_n')\le F(b_n)+\frac{\epsilon}{2^n}$. Dann ist
					\[ [a',b]\subseteq (a,b]=\bigcup_{n\in\N}(a_n,b_n]\subseteq \bigcup_{n\in\N} (a_n,b_n') \]
					Nach dem Satz von Heine-Borel gibt es also eine endliche Teilüberdeckung, also
					\[ \exists N\in\N: [a',b]\subseteq\bigcup_{n=1}^N(a_n,b_n') \]
					\[ \Rightarrow (a',b]\subseteq\bigcup_{n=1}^N (a_n,b_n']. \]
					Nun haben wir eine endliche Vereinigung und wir erhalten aus der endlichen Subadditivität des Inhalts $\mu_F$:
					\[ \mu_F((a',b])\le\sum_{n=1}^{N}\mu_F((a_n,b_n'])\le\sum_{n=1}^{\infty}\mu_F((a_n,b_n']) \]
					\[ F(b)-F(a)-\epsilon\le F(b)-F(a')\le\sum_{n\in\N}(F(b_n')-F(a_n) \]
					\[ \Rightarrow F(b)-F(a)-\epsilon\le\sum_{n\in\N} (F(b_n)+\frac{\epsilon}{2^n}-F(a_n))=\sum_{n\in\N}(F(b_n)-F(a_n))+\epsilon \]
					\[ \Rightarrow F(b)-F(a)\le \sum_{n\in\N}(F(b_n)-F(a_n))+2\epsilon, \]
					da $\epsilon$ beliebig, erhalten wir die Behauptung. 
					
				\end{enumerate}
			\end{bew}
			
			\begin{bem}
				Für Wahrscheinlichkeitsmaße $\mu$ hat die Verteilungsfunktion 
				\[ F(x):=\mu((-\infty,x]) \]
				die zusätzlichen Eigenschaften:
				\begin{itemize}
					\item
					\[ 0\le F\le 1 \]
					\item
					\[ \lim_{x\to-\infty}F(x)=0 \]
					\item
					\[ \lim_{x\to +\infty} F(X)=1. \]
				\end{itemize}
				Eine Verteilungsfunktion, die das erfüllt, heißt Verteilungsfunktion im engeren Sinn. 
			\end{bem}
			
		\subsection{Maße von Mengen mit Verteilungsfunktionen}
			Ab diesem Kapitel werden wir offene Intervallgrenzen auch mit eckigen Klammern schreiben.\newline
			Wir wissen schon:
			\[ \mu(]a,b])=F(b)-F(a). \]
			Was passiert, für$\mu([a,b]), \mu(]a,b[), \mu([a,b[)$? 
			\[ \mu([a,b])=\mu(\bigcap_{n\in\N}]a-\frac{1}{n},b])=\lim_{n\to\infty}\mu(F(b)-F(a-\frac{1}{n}))=F(b)-F(a-0) \]
			\[ \mu(]a,b[)=\mu(\bigcup_{n\in\N}]a,b-\frac{1}{n}])=\lim_{n\to\infty}(F(b-\frac{1}{n})-F(a))=F(b-0)-F(a) \]
			\[ \mu([a,b[)=F(b-0)-F(a-0) \]
			Und damit auch 
			\[ \mu(\{x\})=\mu([x,x])=F(x)-F(x-0) (=\text{Sprunghöhe von $F\text{ in $x$}$}) \]
			\begin{satz}
				Jedes (sigma-)endliche Maß $\mu$ auf $(\Omega, \mf{S})$ lässt sich darstellen als Summe eines stetigen Maßes $\mu_c$ und eines diskreten Maßes $\mu_d$, wobei
				\begin{itemize}
					\item $\mu_d$ diskret, wenn es eine Menge $D$ gib, die höchstens abzählbar ist, sodass
					\[ \mu(D^c)=0. \]
					\item $\mu_c$ stetig, wenn 
					\[ \forall w\in\Omega: \mu_c(\{w\})=0. \]
				\end{itemize}
				Nämlich
				\[ \mu(A)=\mu(A\cap D^c)+\mu(A\cap D)=0+\mu(\bigcup_{x\in A\cap D} \{x\})=\sum_{x\in A\cap D}\mu(\{x\})=\sum_{x\in A}\mu(\{x\}) \]
			\end{satz}
			
			\begin{bew}
				In der Übung.
			\end{bew}
			
			\begin{bsp}
				Sei $\mu$ ein endliches Lebesgue-Stieltjes Maß auf $(\R,\mf{B})$. Für die Verteilungsfunktion $F(x)=\mu(]-\infty,x[)$ kann man nun, da $\mu$ dargestellt werden kann als
				\[ \mu=\mu_c+\mu_d \]
				auch zerlegen in 
				\[ F=F_c+F_d, F_d(x)=\sum_{y\le x} \mu_d(\{y\}).\]
				Wir erhalten den folgenden Satz:
			\end{bsp}
			
			\begin{satz}
				Jede diskrete Verteilungsfunktion (Verteilungsfunktion eines diskreten, endlichen Maßes) auf $\R$ lässt sich anschreiben als
				\[ F(x)=\sum_{y\le x} p(y). \]
				Ist $\sum_{y\in\R} p(y)=1$, so nennen wir $p$ Wahrscheinlichkeitsfunktion. Umgekehrt gibt es zu jeder Funktion $p$ mit $p(y)\ge 0$ eine diskrete Verteilungsfunktion. 
			\end{satz}
			
			\begin{defi}
				Ein Wahrscheinlichkeitsmaß $\P$ auf $(\R,\mf{B})$ heißt Verteilung. 
			\end{defi}
			
			\begin{satz}
				Ist eine Verteilungsfunktion $F(x)$ (stückweise) stetig differenzierbar, $f(x):=F'(x)\ge0$, so ist
				\[ \mu_F(]a,b])=\int_{a}^{b} f(x)dx. \]
				$f(x)$ heißt dann Dichtefunktion. 
			\end{satz}
			
			\begin{bem}
				Ist $\mu_F(\R)=1$, so ist
				\[ 1=\int_{-\infty}^{+\infty} f(x)dx. \]
			\end{bem}
			
			\begin{bem}
				Wir werden anstatt des Riemann-Integrals bald ein Lebesgue-Integral schreiben.
			\end{bem}
			
			\begin{bsp} [Standardnormalverteilung]
				\[ \varphi(x)=\frac{1}{\sqrt{2\pi}}e^{-\frac{x^2}{2}}. \]
				Aus der Analysis ist schon bekannt
				\[ \int_{-\infty}^{+\infty}e^{-\frac{x^2}{2}}dx=\sqrt{2\pi}, \]
				also
				\[ \int_{-\infty}^{+\infty}\varphi(x)dx=1. \]
				Wir erhalten die Verteilungsfunktion
				\[ \Phi(x):=\int_{-\infty}^{x}\varphi(x)dx. \]
				Dann ist
				\[ \P_\Phi(]a,b])=\Phi(b)-\Phi(a). \]
				Zum Beispiel also 
				\[ \P_\Phi(]-1,2])=\Phi(2)-\Phi(-1)=0.9772-0.1587=0.8185, \]
				\[ \Phi(1.67)=0.9525 \]
			\end{bsp}
			
			\begin{bsp}
				Allgemeiner nimmt man 
				\[ \mathcal{N}(\mu,\sigma^2,x)=\frac{1}{\sqrt{2\pi \sigma^2}}e^\frac{-(x-\mu)^2}{2\sigma^2} \]
			\end{bsp}
			
		\subsection{Mehrdimensionale Lebesgue-Stieltjes Maße und Verteilungsfunktionen}
			Der hier verwendete Maßraum ist $(\R^d, \mf{B}_d)$ mit 
			\[ \mf{B}_d:=\mf{A}_\sigma\left(\{]a,b]: a,b\in\R, a\le b\}\right), \]
			wobei die Ungleichung $a\le b$ komponentenweise zu verstehen ist, also
			\[ a\le b:\Leftrightarrow\forall i\in\{1,...,d\}: a_i\le b_i \]
			und 
			\[ ]a,b]:=]a_1,b_1]\times]a_2,b_2]\times...\times]a_d,b_d] \]
			
			\begin{defi}
				Sei $\mu$ ein Lebesgue-Stieltjes Maß auf $(\R^d, \mf{B}_d)$, wenn für beschränkte Mengen $A\in\mf{B}_d$
				\[ \mu(a)<\infty. \]
			\end{defi}
			
			\begin{bem}
				Sei $\mu$ ein endliches Maß. Dann können wir die Verteilungsfunktion wieder anschreiben als
				\[ F(x)=\mu(]-\infty,x])=\mu(]-\infty,x_1])\times...\times\mu(]-\infty,x_d]). \]
				Genügt dies, um $\mu$ festzulegen?
			\end{bem}
			
			\begin{bsp}
				Für $d=2$ erhalten wir:
				\[ \mu(]a,b])=F(b_1,b_2)-F(a_1,b_2)-F(b_1,a_2)+F(a_1,a_1). \]
				Wir können den Satz von oben also zmd. für den 2-dimensionalen Raum erweitern:
			\end{bsp}
			
			\begin{satz}
				$F$ ist eine Verteilungsfunktion von einem Lebesgue-Stieltjes Maß $\mu$, wenn 
				\begin{itemize}
					\item $F$ rechtsstetig ist, also
					\[ x_n\downarrow x\Rightarrow F(x_n)\downarrow F(x) \]
					\item $F$ monoton ist, also
					\[ a\le b\Rightarrow F(b_1,b_2)-F(a_1,b_2)-F(b_1,a_2)+F(a_1,a_2)\ge 0 \]
				\end{itemize}
			\end{satz}
			
			\begin{bew}
				Analog zum 1-dimensionalen Fall.
			\end{bew}
			
			\begin{bsp}[random shit]
				Für $d\ge 2$ erhalten wir:
				\[ \mu(]a,b])=\mu(]a_1,b_1]\times...\times]a_d,b_d])=\sum_{e\in\{0,1\}^d} F(ae+b(1-e)), \]
				wobei 
				\[ ae+b(1-e)=\big(a_1e_1+b_1(1-e_1),...,a_de_d+b_d(1-e_d)\big). \]
			\end{bsp}
			
			\begin{defi}[Differenzoperatoren]
				\begin{align*} \Delta_i(a_i,b_i): &\R^{\R^d}\to\R^{\R^d};\\&f\mapsto \Delta_i (a,b)f(x_1,...,x_d):=f(x_1,...,x_{i-1},b_i,x_{i+1},...x_d)-f(x_1,...,x_{i-1},a_i,x_{i+1},...,x_d)
				\end{align*}
			\end{defi}
			
			\begin{bsp}
				$d=2$. $f(x_1,x_2)=x_1x_2$. 
				\[ \Delta_1(4,17)f(x_1,x_2)=17x_2-4x_2-13x_2-13x_2 \]
				bzw
				\[ \Delta_1(a_1,b_1)f(x_1,x_2)=(b_1-a_1)x_2 \]
				\[ \Delta_1(a_1,b_1)\Delta_2(a_2,b_2)f(x_1,x_2)=(b_1-a_1)b_2-(b_1-a_1)a_2=(b_1-a_1)(b_2-a_2) \]
			
			\end{bsp}
			\begin{bem}
				Damit ist (für $d\in\N$)
				\[ \mu_F(a,b)=\Delta_1(a_1,b_1)\Delta_2(a_2,b_2)...\Delta_d(a_d,b_d)F \]
				Und
				\[ \Delta_i(a_i,b_i) F(x_1,...,x_d)=\int_{a_i}^{b_i}\frac{\partial}{\partial x_i} F(x_1,...,x_d)dx_i \]
			\end{bem}
			
			\begin{bsp}
				Endliche Maße:
				\[ F(x)=\mu(]-\infty,x]) \]
				Wir betrachten den Spezialfall für $d=2$. Dann ist
				\[ \mu(]0,x])=F(x_1,x_2)-F(x_1,0)-F(0,x_2)+F(0,0). \]
				Setze  $F(x_1,0)=F(0,x_2)=0$. Dann ist für $x>0$
				\[ F(x_1,x_2)=\mu(]0,x_1]\times]0,x_2]) \]
				und für $x_1\ge 0, x_2 <0$
				\[ \mu(]0,x_1]\times]x_2,0])=F(x_1,0)-F(0,0)-F(x_1,x_2)+F(0,x_2)=-F(x_1,x_2). \]
				Dies lässt sich quadrantenweise durchführen. \newline
				Allgemein: 
				\[ F(x)=\mu(]\min(x,0),\max(x,0)])\sgn(x), \]
				wobei das Minimum und Maximum koordinatenweise zu verstehen ist und 
				\[ \sgn(x)=\prod_{i=1}^{n}\sgn(x_i) \]
			\end{bsp}
			
			\begin{defi}
				Das $d$-dimensionale Lebesguemaß $\lambda_d$ ist
				\[ \lambda_d(]a,b])=\prod_{i=1}^d(b_i-a_i) \]
				und
				\[ F(x_1,...,x_d)=x_1\cdots x_d. \]
				Mit Hilfe des Fortsetzungssatzes erhalten wir das Maß $\lambda_d$ auf $\mathfrak{B}_d$.\newline
				Die $\lambda_d^*$-messbaren Mengen werden mit $\mathfrak{L}_d$ ($d$-dimensionale Lebesguemengen) bezeichnet, wobei
				\[ A\in \mathfrak{L}_d\Leftrightarrow A=B\cup N, B\in\mathfrak{B}_d, \exists M\in\mathfrak{B}_d: N\subseteq M, \lambda_d(M)=0 \] 
			\end{defi}
			
			\begin{satz}
				Sei $\lambda_d$ das Lebesguemaß auf $\mathfrak{B}_d$. Dann gilt:
				\begin{itemize}
					\item $\lambda_d$ ist translationsinvariant:
					\[ A\oplus c:=\{x+c:x\in A\}, \]
					\[ A\in\mathfrak{L}_d, c\in\R^d\Rightarrow A\oplus c\in \mathfrak{L}_d, \lambda_d(A\oplus c)=\lambda_d(A) \]
				\end{itemize}
			\end{satz}
			
			\begin{bew}
				$\text{   }$
				\begin{itemize}
					\item Wir zeigen zunächst
					\[  A\in\mathfrak{L}_d, c\in\R^d\Rightarrow A\oplus c\in \mathfrak{L}_d. \]
					Sei $A=]a,b]$ und $\mathfrak{S}:=\{A\in\mathfrak{B}_d: A\oplus c\in \mathfrak{B}_d\}$. $\mathfrak{S}$ ist Sigmaalgebra, damit gilt die Behauptung. $\lambda_d(A\oplus c)=\lambda_d(A)$ für $A\in\mathfrak{B}$ ist klar, da sie auf dem Semiring gilt, damit auch auf den Borelmengen. 
				\end{itemize}
			\end{bew}
