In diesem Abschnitt werden wir uns drei Fragen stellen:
	\begin{itemize}
		\item Was können wir messen?
		\item Wie können wir messen?
		\item Wie können wir Maße ökonomisch definieren?
	\end{itemize}
	
	\section{Mengensysteme}
		\begin{defi}
			Sei $\Omega$ eine beliebige Menge. Dann heißt $\mf{C}\subseteq 2^\Omega$ ein Mengensystem (über $\Omega$).
		\end{defi}
		
		\begin{defi}
			Eine Mengenfunktion $\mu$ auf dem Mengensystem $\mf{C}$ heißt additiv, falls
			\[ \mu \left ( \biguplus_{i \in I} A_i \right ) = \sum_\In \mu (A_i)\]
		\end{defi}
		
		\begin{defi} [Semiring]
			Sei $\mf{T}$ ein nichtleeres Mengensystem über $\Omega$. Dann heißt $\mf{T}$ Semiring (im weiteren Sinn), falls
			\begin{enumerate}
				\item Durchschnittsstabilität:
				\[ A,B\in\mf{T}\Rightarrow A\cap B\in\mf{T} \]
				\item Leiterbildung:
				\[ A,B\in\mf{T}, A\subseteq B\Rightarrow \exists n\in\N: C_1,...,C_n\in\mf{T}: \forall i\neq j: C_i\cap C_j=\varnothing, A\setminus B=\bigcup_{i=1}^n C_i \]
			\end{enumerate}
			gilt zusätzlich für die Leiter
			\[ \forall k=1,...,n: A\cup\bigcup_{i=1}^k C_i\in\mf{T}, \]
			so spricht man von einem Semiring im engeren Sinn.
		\end{defi}
		
		\begin{defi}[Ring]
			Sei $\mf{R}$ ein nichtleeres Mengensystem über $\Omega$. $\mf{R}$ heißt Ring, falls
			\begin{enumerate}
				\item Differenzenstabilität:
				\[ A,B\in\mf{R}\Rightarrow B\setminus A\in\mf{R} \]
				\item Vereinigungsstabilität:
				\[ A,B\in\mf{R}\Rightarrow A\cup B\in\mf{R} \]
			\end{enumerate}
		\end{defi}
		
		\begin{defi}[Sigmaring]
			Sei $\mf{R}_\sigma$ ein nichtleeres Mengensystem über $\Omega$. $\mf{R}_\sigma$ heißt Sigmaring, falls
			\begin{enumerate}
				\item Differenzenstabilität:
				\[ A,B\in\mf{R}_\sigma\Rightarrow B\setminus A\in\mf{R}_\sigma \]
				\item Sigma-Vereinigungsstabilität:
				\[ A_n\in\mf{R}_\sigma\Rightarrow \bigcup_{n\in\N}A_n\in\mf{R}_\sigma \]
			\end{enumerate}
		\end{defi}
		
		\begin{defi}[Algebra]
			Sei $\mf{A}$ ein nichtleeres Mengensystem über $\Omega$. $\mf{A}$ heißt Algebra, falls
			\begin{enumerate}
				\item Abgeschlossenheit bzgl. Komplementbildung:
				\[ A\in\mf{A}\Rightarrow A^c\in\mf{A} \]
				\item Vereinigungsstabilität:
				\[ A,B\in\mf{A}\Rightarrow A\cup B\in\mf{A} \]
			\end{enumerate}
		\end{defi}
		
		\begin{defi}[Dynkin System]
			Sei $\mf{D}$ ein nichtleeres Mengensystem über $\Omega$. $\mf{D}$ heißt Dynkin-System (im weiteren Sinn), falls
			\begin{enumerate}
				\item Sigmaadditivität:
					\[A_i\in\mf{D}: A_i \text{ disjunkt}\Rightarrow \bigcup_{i\in\N} A_i\in\mf{D}\] 
				\item Differenzenstabilität:
					\[ \forall A,B\subseteq \Omega: A,B\in\mf{D}\Rightarrow B\setminus A\in\mf{D}\]
			\end{enumerate}
			Ist zusätzlich noch
			\[ \Omega\in\mf{D} \]
			erfüllt, so spricht man von einem Dynkin-System im engeren Sinn.
		\end{defi}
		
		\begin{lemma}
			\begin{enumerate}[(i)]
				\item Wenn ein Dynkinsystem abgeschlossen bezüglich $\cap$ ist, so ist es eine Sigmaalgebra.
				\item Sei $\mf{C}$ ein Mengensystem, welches abgeschlossen bezüglich $\cap$ ist, so gilt:
				\[ \mf D(\mf C) = \mf A_\sigma(\mf C) \]
				\item Für endliche Maße $\mu,\nu$ auf einem Ring $\mf R$ ist
				\[ \{a \in \mf R:\mu (A)=\nu(A) \} \]
				ein Dynkinsystem im weiteren Sinn.
			\end{enumerate}
		\end{lemma}
		
		\begin{satz}
			Eine Mengenfunktion $\mu$ auf einem Semiring im engeren Sinn $\mf T$ ist genau dann additiv, wenn für disjunkte Mengen $A,B \in \mf T$ mit $A \cup B \in \mf T$ gilt:
			\[ \mu(A \cup B)=\mu(A) + \mu(B) \]
		\end{satz}
		\begin{bew}
			Es gilt $ A \cap B = \emptyset $ und $ \mu(A \cup B)=\mu(A) + \mu(B) $ \\
			Dann folgt mittels vollständiger Induktion, dass für disjunkte $(A_i)_\In$ aus $\mf T$ gilt:
			\[ \mu \left ( \biguplus_\In A_i \right ) = \sum_\In \mu(A_i) \]
		\end{bew}
		
		\begin{bsp} $ $ 
			\begin{enumerate}[-]
				\item Für ein beliebiges $\Omega$ ist $\mf C := \{ A \subset \Omega: |A|< \infty \} $ ein Ring und damit auch ein Semiring.
				\item Sei $a \in \N$, so ist $ \mf C:=\{ A \subset \Omega: |A|<a \}$ für $|\Omega|>a$ nur ein Semiring
				\item $ \mf C := \{ A \subset \Omega: card(A) \leq \aleph_0 \} $ eine Sigmaalgebra
			\end{enumerate}
		\end{bsp}
		
		\begin{satz}
			$\mf T$ sei ein Semiring ( in weiterem Sinne) und $I:=\{1,...,n\}$. Dann gilt:
			\[ \mf R(\mf T) = \{ \bigcup_{i=1}^n A_i,n \in \N, A_i \in \mf T \} = \{ \sum_{i=1}^n, n \in \N, A \in \mf T \} \]
		\end{satz}
		\begin{bew}
			Da $\mf R(\mf T)$ abgeschlossen bezüglich der Vereinigung ist, gilt
			\[ \mf R(\mf T) \supset \{ \bigcup_{i=1}^n A_i,n \in \N, A_i \in \mf T \} \]
			Nun muss man nur noch die andere Inklusion zeigen.\\
			Es gilt $M:=\{ \bigcup_{i=1}^n A_i,n \in \N, A_i \in \mf T \}$ ist abgeschlossen bezüglich der Vereinigung. Nun wollen wir die Abgeschlossenheit bezüglich der Mengendifferenz zeigen. Dazu seien $A,B \in M$. Es gilt:
			\begin{align*}
				A \backslash B = A \cap B^c = A \cap (\Omega \cap B) \in M
			\end{align*}
			da $M$ abgeschlossen bezüglich des Durchschnittes ist. Somit ist die Gleichheit gezeigt.	
		\end{bew}
		
		\begin{satz}
			Sei $\mf C$ ein nicht leeres Mengensystem. Dann ist
			\[ \{ \bigcap_{i=1}^n A_i | n \in \N , A_1 \in \mf C, A_i \in \mf C \vee A_i^c \in \mf C ,i \geq 1 \} \]
			ein Semiring.
		\end{satz}
		
		\begin{bsp}Intervalle\\
				$ \mf T:=\{ (a,b] | a \leq b \wedge a,b \in \R \} $
				"westlich" $ $ und 
				$ \{ [a,b) | a \leq b \wedge a,b \in \R \} $ 
				"russisch"
			bilden Semiringe.\\
			Im $\R^n:(a,b]=(a_1,b_1] \times (a_2,b_2] \times ... \times (a_n,b_n]$\\
			Der erzeugte Sigmaring von $\mf T$ sind die Borelmengen $ \mf B (\mf B_n)$\\
			Für zwei Semiringe $\mf T_1, \mf T_2$ ist 
			\[ \mf T_1 \times \mf T_2 = \{ A_1 \times A_2 | A_1 \in \mf T_1 \wedge A_2 \in \mf T_2 \} \]
			ein Semiring.			
		\end{bsp}
		
		\begin{defi}
			Ein Mengensystem $\mf C$ heißt monoton, wenn
			\[ A_n \in \mf C, A_n \subset A_{n+1} , n\in \N \Rightarrow \bigcup_{n \in \N} A_n \in \mf C \]
			oder 
			\[ A_n \in \mf C, A_n \supset A_{n+1} , n\in \N \Rightarrow \bigcap_{n \in \N} A_n \in \mf C \]
		\end{defi}
		
		\begin{satz}[monotone class theorem]
			Der von einem Ring erzeugte Sigmaring stimmt mit dem erzeugten monotonen System überein (Jeder monotone Ring ist Sigmaring)
		\end{satz}
		
		\begin{defi}
			Für Zahlenfolgen:
			\begin{align*}
				\limsup_{n \in \N} x_n &= \inf_{n \in \N} \sup_{k \geq n} x_n\\
				\liminf_{n \in \N} x_n &= \sup_{n \in \N} \inf_{k \geq n} x_n
			\end{align*}
			Für Mengenfolgen:
			\begin{align*}
			\limsup_{n \in \N} A_n &= \bigcap_{n \in \N} \bigcup_{k \geq n} A_n=\{ x:x \in A_n \text{ für unenlich viele n} \}\\
			\liminf_{n \in \N} A_n &= \bigcup_{n \in \N} \bigcap_{k \geq n} A_n=\{ x:x \in A_n \text{ für fast alle n} \}
			\end{align*}
		\end{defi}
		\begin{bem} Es gibt einige Tricks, wenn man mit Mengen arbeitet:
			\begin{enumerate}[1.]
				\item Folgen monoton machen:\\
					Sei $(A_n)_{n \in \N}$ eine Mengenfolge, so ist $B_n=\bigcup_{i=1}^n A_i$ eine monoton wachsende Folge.
				\item Folgen disjunkt machen:\\
					Sei $C_1 =B_1=A_1$
					\[ C_n=B_n \backslash B_{n-1} = \left (\bigcup_{j=1}^{n}A_j \right ) \backslash \bigcup_{i=1}^{n-1} A_i=
					\left ( \bigcup_{j=1}^{n-1}A_j \cup A_n \right ) \backslash \bigcup_{i=1}^{n-1} A_i = \emptyset \cup A_n \backslash \bigcup_{i=1}^{n-1} A_i  \]
					mit $ \bigcup_{n \in \N} C_n = \bigcup_{n \in \N} A_n$
				\item Klassische Tauschgeschäft:\\
					Wenn du eine Gleichung willst, musst du 2 Ungleichungen zeigen
					\begin{align*}
						&x=y \Leftrightarrow (x \leq y) \wedge (x \geq y)\\
						&A=B \Leftrightarrow (A \subset B) \wedge (A \supset B)
					\end{align*}
				\item Prinzip der guten Menge:
					Wenn du zeigen willst, dass alle Elemente x aus einer Menge $X$ eine Eigenschaft haben, dann zeigt man $ Y \supset X $ für:
					\[ Y:=\{ x \in X | x \text{ hat die Eigenschaft} \} \]
			\end{enumerate}
		\end{bem}
		
		\begin{defi}
			Seien $\mf S_1, \mf S_2$ Sigmaalgebren über $\Omega$, so heißt $ \mf S_1 \times \mf S_2 = \mf A_\sigma(\mf S_1 \otimes \mf S_2) $ die Produktalgebra
		\end{defi}
		
		\begin{defi}
			Sei $f:\Omega_1 \to \Omega_2$ und $A \in \Omega_2$, so heißt
			\[ f^{-1}(A):=\{ x \in \Omega_1 |f(x) \in A \} \]
			das Urbild von $A$.
		\end{defi}
		
		\begin{satz}
			Sei $f:\Omega_1 \to \Omega_2$ , $\mf S_2$ Sigmaalgebra über $\Omega_2$ dann ist $ f^{-1}(\mf S_2) := \{ f^{-1}(A) :A \in \mf S_2 \} $ eine Sigmaalgebra über $\Omega_1$
		\end{satz}
		\begin{bew}
			Es gilt zu zeigen:
			\begin{enumerate}[1.]
				\item $ f^{-1}(\mf S_2) \neq \emptyset $\newline
					da $\emptyset \in \mf S_2$ und stets $f(\emptyset) = \emptyset$.
				\item $A \in f^{-1}(\mf S_2) \Rightarrow A^c \in f^{-1}(\mf S_2)$ \newline
					\[ A \in f^{-1}(\mf S_2) \Rightarrow \exists B \in \mf S_2:f^{-1}(B)=A \Rightarrow A^c=f^{-1}(B)^c=f^{-1}(B^c) \in f^{-1}(\mf S_2) \]
				\item $ (A_n)_{n \in \N}, A_n \in f^{-1}(\mf S_2) \Rightarrow \bigcup_{n \in \N} A_n \in f^{-1}(\mf S_2) $
					\[ \Rightarrow \exists (B_n)_{n \in \N} : A_n = f^{-1}(B_n) \Rightarrow \bigcup_{n \in \N} A_n = \bigcup_{n \in \N} f^{-1}(B_n) =f^{-1} \left ( \bigcup_{n \in \N} B_n \right ) \in f^{-1}(\mf S_2) \]
			\end{enumerate}
			Also ist $f^{-1}(\mf S_2)$ eine Sigmaalgebra.
		\end{bew}
		
		\begin{bem}
			Dieser Satz funktioniert auch für:
			\begin{enumerate}[-]
				\item Semiringe
			\end{enumerate}
			Jedoch nicht für
			\begin{enumerate}[-]
				\item Dynkin-Systeme
				\item monotone Systeme
			\end{enumerate}
		\end{bem}
		
		\begin{satz}
			Sei $ f:\Omega_1 \to \Omega_2 $ und $\mf C$ ein beliebiges Mengensystem über $\Omega_2$
			\[ \Rightarrow \mf A_\sigma(f^{-1}(\mf C)) = f^{-1}(\mf A_\sigma(\mf C)) \]
		\end{satz}
		\begin{bew}
			Wir zeigen zwei Inklusionen:\\
			$ A_1:= \mf A_\sigma(f^{-1}(\mf C)) \subset f^{-1}(\mf A_\sigma(\mf C)) =: A_2$ laut dem vorherigen Satz.\\
			"$ A_1 \supset A_2 $" vermittels dem Prinzip der gutem Menge\\
			Sei $G:=\{ A \in \mf A_\sigma(\mf C)| f^{-1}(A) \in \mf A_\sigma(f^{-1}(\mf C)) \}$
			\begin{enumerate}[1.]
				\item Es ist klar, dass $\mf C \subset G$
				\item Zu zeigen: G ist Sigmaalgebra \\
					Sei $A \in G: f^{-1}(A) \in \mf A_\sigma(f^{-1}(\mf C)) $
					\[\Rightarrow f^{-1}(A^c) = f^{-1}(A)^c \in  \mf A_\sigma(f^{-1}(\mf C))\]
					Also ist $A^c \in G$. \\
					Sei $A_n \in G, n \in \N$
					\[ \Rightarrow f^{-1} (A_n) \in  \mf A_\sigma(f^{-1}(\mf C)) \Rightarrow f^{-1} \left ( \bigcup_{n \in \N} A_n \right ) = \bigcup_ {n \in \N} f^{-1}(A_n) \in  \mf A_\sigma(f^{-1}(\mf C)) \]
					Also ist $G=\mf A_\sigma(\mf C)$.
			\end{enumerate}
			Damit ist die Aussage bewiesen.
		\end{bew}
				
	\section{Maße und Inhalte}
		\begin{defi}
			Ein Inhalt $\mu$ auf einem Mengensystem $C$ heißt endlich, wenn für alle $A\in C$:
			\[ \mu(A)<\infty \]
		\end{defi}
		\begin{defi}
			Ein Maß $\mu$ auf $C$ heißt sigmaendlich, wenn für jedes $A\in C$ Mengen $A_n\in C, n\in \N$ existieren mit $\mu(A_n)<\infty, A\subseteq \bigcup_{n\in\N} A_n$.
		\end{defi}
		\begin{defi}
			Ein Inhalt $\mu$ auf $C$ heißt totalendlich, wenn
			\[ \Omega\in C\land \mu(\Omega)<\infty \]
		\end{defi}
		\begin{defi}
			Ein Inhalt $\mu$ auf $C$ heißt total sigmaendlich, wenn es $A_n\in C, n\in \N$ gibt mit $\mu(A_n)<\infty$ und $\Omega\subseteq \bigcup_{n\in \N} A_n$. 
		\end{defi}
		\begin{defi}
			$A\in C$ hat sigmaendliches Maß ($A$ ist sigmaendlich), wen es $A_n\in C, n\in\N: \mu(A_n)<\infty$ und $A\subseteq \bigcup A_n$. 
		\end{defi}
		\begin{defi}
			$\mu$ heißt Wahrscheinlichkeitsmaß, wenn $\mu(\Omega)=1$.
		\end{defi}
		\begin{bsp}
			Sei $\Omega\neq\varnothing$ endlich, $C=2^\Omega, \mu(A)=\frac{|A|}{|\Omega|}$.
		\end{bsp}
		\begin{bsp}
			Sei $\Omega=\{1,2,3,4,5,6\}$, also ein "`fairer Würfel"'.
			\end{bsp}
		\begin{bsp}
			Sei $\Omega=\{(1,1), (1,2),...,(2,1),(2,2),...,(6,6)\}$, also würfeln mit zwei Würfeln, Würfel sind unterscheidbar. 
		\end{bsp}
		\begin{defi}
			Sei $\Omega\neq\varnothing$ beliebige Menge und $\ms{S}$ eine Sigmaalgebra über $\Omega$. Dann heißt $(\Omega, \ms{S})$ Messraum. 
		\end{defi}
		\begin{defi}
			Sei $\mu$ ein Maß auf $\ms{S}$ und $(\Omega, \ms{S})$ Messraum. Dann heißt $(\Omega, \ms{S}, \mu)$ Maßraum. 
		\end{defi}
		\begin{bsp}
			$(\Omega, 2^\Omega, \mu)$, $\Omega\neq\varnothing$ endlich, $C=2^\Omega, \mu(A)=\frac{|A|}{|\Omega|}$ ist der Laplace-Wahrscheinlich\-keitsraum.
		\end{bsp}
		\begin{satz}
			Seien $\mu_n$ Inhalte auf $\ms{C}$, und existiere $\mu(A)=\lim_{n\to\infty} \mu_n(A)$. Dann ist $\mu$ ein Inhalt. 
		\end{satz}
		\begin{bew}
			$A=\sum_{i=1}^k A_i$, $\mu(A)=\sum_{i=1}^k \mu_n(A_i)$, für $n\to\infty$ gehen beide Seiten gegen $\mu$, stimmt also. 
		\end{bew}
		\begin{satz}[Satz von Vitali-Hahn Saks:]
			Wenn $\ms{C}$ ein Sigmaring ist und $\mu_n$ endliche Maße und für alle $A\in\ms{C}: \mu(A)=\lim_{n\to\infty} \mu_n(A)$, dann ist $\mu$ auch ein Maß. 
		\end{satz}
		\begin{bew}
			noch nicht, Eigenschaften fehlen noch. 
		\end{bew}
		\begin{satz}
			Sei $\mu$ ein Inhalt/Maß auf einem Ring. Dann gilt:
			\begin{enumerate}
				\item Monotonie: 
				\[ A,B\in\ms{R}, A\subseteq B\Rightarrow \mu(A)\le \mu(B) \]
				\item Additionstheorem:
				\[ \mu(A\cup B)=\mu(A)+\mu(B)-\mu(A\cap B) \]
				\item Allgemeineres Additionstheorem:
				\begin{align*}
				\mu\left(\bigcup_{i=1}^n A_i\right)&=\sum_{J\subseteq\{1,...,n\}, J\neq \varnothing} (-1)^{|J|-1}\mu\left(\bigcap_{i\in J} A_i\right)\\
				&=\sum_{k=1}^n (-1)^{k-1} S_k \:\:\:\text{für } S_k=\sum_{i\le i_1<...<i_k\le n} \mu\left(\bigcap_{k=1}^n A_{i_k}\right)
				\end{align*}
				\item Subadditivität:
				\[ \mu\left(\bigcup_{i=1}^n A_i\right)\le \sum_{i=1}^n \mu(A_i) \]
			\end{enumerate}
			
		\end{satz}
		\begin{bew}
			\begin{enumerate}
				\item Es gilt:
				\[ B=A\cup (B\setminus A)\Rightarrow \mu(B)=\mu(A)+\mu(B\setminus A)\ge \mu(A) \]
				Nun ist außerdem mit $\mu(A)<\infty$:
				\[ \mu(B\setminus A)=\mu(B)-\mu(A) \]
				\item
				Für $A,B\in\ms{R}$:
				\[ \mu(B\setminus A)=\mu(B\setminus (A\cap B))=\mu(B)-\mu(A\cap B) (\text{ wenn }\mu(A\cap B)<\infty) \]
				\[ \Rightarrow \mu(A\cup B)=\mu(A)+\mu(B)-\mu(A\cap B) \]
				Außerdem (Zusatz für zwei Mengen):
				\begin{align*} \mu(A\cup B\cup C)&=\mu((A\cup B)\cup C)\\&=\mu(A)+\mu(B)+\mu(C)-\mu(A\cap B)-\mu(A\cap C)-\mu(B\cap C)+\mu(A\cap B\cap C) \end{align*}
				\item Es gilt:
				\[ A,B\in\ms{R}: \mu(A\cup B)=\mu(A\cup (B\setminus A))=\mu(A)+\mu(B\setminus A)\le \mu(A)+\mu(B) \]
				\[ \Rightarrow \mu\left(\bigcup_{i=1}^n A_i\right)\le \sum_{i=1}^n \mu(A_i) \]
				\item Induktion (wahrscheinlich)
			\end{enumerate}
		\end{bew}
		\begin{satz}
			Sei $\mu$ Inhalt auf $\ms{R}$, $A_n, n\in\N, A\subseteq\ms{R}$, dann gilt:
			\[ \sum_{n\in\N} A_n\subseteq A\Rightarrow \sum_{n\in\N}\mu(A_n)\le \mu(A) \]
		\end{satz}
		\begin{bew}
			Es gilt:
			\[ \sum_{n=1}^N A_n\subseteq A\Rightarrow \mu\left(\sum_{n=1}^N A_n\right) \le \mu(A) \]
			\[ \Rightarrow \sum_{n=1}^N \mu(A_n)\le \mu(A) \]
			Für $n\to\infty$:
			\[ \sum_{n\in\N} \mu(A_n)\le \mu(A) \]
		\end{bew}
		
		\section{Folgerungen für Maße}
		
		\begin{satz}
			Sei $\mu$ ein Maß auf $\ms{R}$:
			\begin{enumerate}
				\item Stetigkeit von unten:
				\[ A_n\uparrow A, A_n, A\in\ms{R} \]
				\[ \Rightarrow \mu(A)=\lim_{n\to\infty} \mu(A_n) \]
				\item Stetigkeit von oben:
				\[ A_n\downarrow A, A_n, A\in\ms{R}\land \mu(A_1)<\infty \]
				\[ \Rightarrow \mu(A)=\lim_{n\to\infty} \mu(A_n) \]
			\end{enumerate}
		\end{satz}
			
		\begin{bew}
			\begin{enumerate}
				\item Sei $B_1=A_1$ und $B_n=A_n\setminus A_{n-1}$. Nun sind $B_n$ disjunkt und $A_n=\sum_{i=1}^n B_i$. Nun gilt:
				\[ \mu(A_n)\sum_{i=1}^n \mu(B_i) \]
				und:
				\[ A=\sum_{i=1}^\infty B_i \]
				\[ \Rightarrow \mu(A)=\sum_{i=1}^\infty \mu(B_i)=\lim_{n\to\infty}\sum_{i=1}^n \mu(B_i)=\lim_{n\to\infty} \mu(A_n) \]
				\item 
				\[ \mu(A)=\lim_{n\to\infty}\mu(A_n) \]
				\[ \mu(A_1\setminus A)=\lim_{n\to\infty} \mu(A_1\setminus A_n)=\lim_{n\to\infty} \mu(A_1)-\lim_{n\to\infty} \mu(A_n) \]
			\end{enumerate}
		\end{bew}
			
		\section{Eigenschaften von Maßen (Inhalten) auf Ringen(Semiringen)}
			\begin{satz}
				Sei $\mu$ ein Maß auf dem Ring $\ms{R}$, $A_n\uparrow A$, $A_n, A\in \ms{R}$. Dann gilt
				\[ \mu(A)=\lim_{n\to\infty} \mu(A_n) \]
				Entsprechendes für $A_n\downarrow A$.
			\end{satz}
			\begin{satz}
				Sei $\mu$ Inhalt auf Ring $\ms{R}$ ist genau dann ein Maß, wenn $\mu$ stetig von unten ist.
			\end{satz}
			\begin{bew}
				Seien $A_n, A\in\ms{R}$, $A=\sum_{n\in\N} A_n$, $A_n$ paarweise disjunkt. Sei 
				\[ B_n=\sum_{i=1}^n A_i \]
				Nun gilt $B_n\uparrow A$. $\mu$ ist nun stetig von unten, also
				\[ \mu(A)=\lim_{n\to\infty}\mu(B_n)=\lim_{n\to\infty}\mu(\sum_{i=1}^n A_i)=\lim_{n\to\infty} \sum_{i=1}^n \mu(A_i)=\sum_{i=1}^\infty \mu(A_i) \]\arge
			\end{bew}
			\begin{satz}
				Sei $\mu$ ein endlicher Inhalt auf einem Ring $\ms{R}$. Dann ist $\mu$ genau dann ein Maß, wenn er stetig von oben bei $\varnothing$ ist, also
				\[ A_n\downarrow \varnothing\Rightarrow \mu(A_n)\to 0. \]
			\end{satz}
			
			\begin{bew}
				Sei $A_n,A\in\ms{R}, A=\sum_{n=1}^\infty A_n$. \newline
				\zz: $\mu(A)=\sum_{n=1}^\infty \mu(A_n)$. \newline
				Nämlich:
				\[ A=\sum_{i=1}^n A_i\cup \sum_{i=n+1}^\infty \]
				\[ B_n:=\sum_{i=n+1}^\infty \Rightarrow B_n=A\setminus(\sum_{i=1}^n A_i)\in\ms{R} \]
				Nun gilt:
				\[ \mu(A)=\sum_{i=1}^n \mu(A_i)+\mu(B_n) \]
				Nun gilt:
				\[ \lim_{n\to\infty} B_n=\bigcap_{n\in\N} B_n=\bigcap_{n\in\N} A\setminus \left(\bigcup_{i=1}^n A_i\right)=A\setminus\left(\bigcup_{n\in\N}\bigcup_{i=1}^n A_i\right)=\varnothing \]
				Also $B_n\downarrow \varnothing$. Also:
				\[ \mu(A)=\lim_{n\to\infty} \left( \sum_{i=1}^n A_i + \mu(B_n)\right)=\sum_{i=1}^\infty + 0 \]
				\arge
			\end{bew}
			
			\begin{bem}
				Dieses Argument kann auch umgedreht werden. Dies werden wir später zumindest einmal benutzen.
			\end{bem}
			
			\begin{satz}
				Sei $\mu$ ein Maß auf dem Ring(Semiring) $\ms{R}$, $A_n, A\in\ms{R}$ mit
				\[ A\subseteq \bigcup_{n\in\N} A_n \]
				so gilt
				\[ \mu(A)\le \sum_{n\in\N}\mu(A_n).\:\:(\mu\text{ ist abzählbar-, bzw sigmasubadditiv}) \]
			\end{satz}
			
			\begin{bew}
				Sei $B_n=A\cap\bigcup_{i=1}^n A_i=\bigcup_{i=1}^n A\cap A_i$. Es gilt also $B_n\uparrow A$. Aus der endlichen Subadditivität erhalten wir:
				\[ \mu(B_n)\le\sum_{i=1}^n \mu(A_i\cap A)\le \sum_{i=1}^n\mu(A_i)\le \sum_{i=1}^\infty \mu(A_i) \]
				\[ \Rightarrow \mu(A)=\lim_{n\to\infty} \mu(B_n)\le\sum_{i=1}^\infty \mu(A_i) \]
			\end{bew}
			
			\begin{satz}
				Sei $\mu$ ein Maß auf dem Sigmaring $\ms{R}$ und $A_n$ eine Folge von Mengen aus $\ms{R}$. Dann gilt:
				\[ \limsup_{n\to\infty} A_n = \bigcap_{n\in\N}\bigcup_{k\ge n} A_k \]
			\end{satz}
			
			\begin{satz}
				Lemma von Borel Cantelli:\newline
				Sei $\mu$ ein Maß auf einem Sigamring $\ms{R}$. Ist $\sum_{n\in\N} \mu(A_n)<\infty$ für $A_n\in\ms{R}$, so gilt:
				\[ \mu(\limsup_{n\to\infty} A_n)=0 \]
			\end{satz}
			
			\begin{bew}
				Sei $\varepsilon>0$ beliebig. Es gilt:
				\[ \mu(\limsup A_n)\le \mu\left(\bigcup_{k\ge n_0} A_k\right)\le \sum_{k\ge n_0} \mu(A_k)\le \varepsilon \]
				\arge
			\end{bew}
			
			\begin{bem}
				Als Hausübung: Ist $\mu$ endliches Maß auf einem Sigmaring, so gilt
				\[ \mu(\limsup_{n\to\infty} A_n)\ge \limsup_{n\to\infty} \mu(A_n). \]
			\end{bem}
			
			\begin{bsp}[Additionstheorem]
				Die Anzahl der Permutationen von $n$ Elementen ohne Fixpunkt. 
				\[ \mathbb{P}(\text{kein Fixpunkt})=1-\mathbb{P}(\text{Fixpunkt})=1-\mathbb{P}\left(\bigcup A_i\right) \]
				mit $A_i=[i$ ist Fixpunkt $]$. 
				\[ \mathbb{P}\left(\bigcup A_i\right)=\sum_{i=1}^n\mathbb{P}(A_i)-\sum_{1\le i_1\le i_2\le n} \mathbb{P}(A_{i_1}\cap A_{i_2})+\sum \mathbb{P}(A_{i_1}\cap A_{i_2}\cap A_{i_2})-... \]
				Es gilt:
				\[ \mathbb{P}(A_i)=\frac{(n-1)!}{n!} \]
				\[ \mathbb{P}(A_i\cap A_0)=\frac{(n-2)!}{n!} \]
				\[ \mathbb{P}(A_{i_1}\cap ... \cap A_{i_k})=\frac{(n-k)!}{n!} \]
				Jetzt: (was auch immer $S_k$ ist...)
				\[ S_k=\frac{(n-k)!}{n!} \left(\begin{array}{c} n \\ k \end{array}\right)=\frac{1}{k!} \]
				Damit:
				\[ \mathbb{P}\left(\bigcup A_i\right)=\sum_{k=1}^n (-1)^{k-1} \frac{1}{k!} \]
				\[ \Rightarrow \mathbb{P}(\text{kein Fixpunkt})=1-\sum_{k=1}^n(-1)^{k-1} \frac{1}{k!}=\sum_{k=0}^n (-1)^k \frac{1}{k!}\stackrel{\rightarrow}{n\to\infty} \frac{1}{e} \]
			\end{bsp}
			
			\section{Bedingte Wahrscheinlichkeit}
			\begin{defi}
				Sei $(\Omega, \ms{S}, \P)$ ein Wahrscheinlichkeitsraum. Nun heißt $A,B\in\ms{S}$ Ereignisse. Gilt $\P(B)\neq 0$ so heißt 
				\[ \P(A|B):=\frac{\P(A\cap B)}{\P(B)} \]
				die bedingte Wahrscheinlichkeit. 
			\end{defi}
			
			\begin{defi}
				Ereignisse $A$ und $B$ heißen unabhängig, wenn
				\[ P(A\cap B)=\P(A)\P(B). \]
			\end{defi}
			
			\begin{defi}
				Allgemeiner heißen Ereignisse $A_1,...,A_n$ unabhängig, wenn
				\[ \P\left(\bigcap_{i=1}^n A_i\right) = \prod_{i=1}^n \P(A_i). \]
			\end{defi}
			
			\begin{defi}
				Ereignisse $A_1,...,A_n$ heißen paarweise unabhängig, wenn:
				\[ \forall i,j\in\{1,...,n\}: i\neq j\Rightarrow \P(A_i\cap A_j)=\P(A_i)\P(A_j). \]
			\end{defi}
			
			\begin{bem}
				Es gilt:
				\[ \P(A\cap B)=\P(B)\P(A|B)=\P(A)\P(B|A) \]
				und:
				\[ \P(A_i\cap...\cap A_n)=\P(A_1)\P(A_2|A_1)\P(A_3|A_1\cap A_2)...P(A_n|A_1\cap...\cap A_n) \]
				Dies ist das Multiplikationstheorem für Wahrscheinlichkeiten. 
			\end{bem}
			
			\begin{bsp}[Bedingte Wahrscheinlichkeiten (Multiplikationstheorem)]
				In einer Urne liegen zwei schwarze und drei weiße Kugeln. Es wird 3-mal ohne Zurücklegen gezogen, wobei das Ziehen der Laplace-Wahrscheinlichkeit folgt. Nun ist
				\[ \P(\text{Alle 3 Kugeln weiß})=\P(A_1\cap A_2\cap A_3) \]
				wobei $A_i=$ "`$i$-te Kugel ist weiß"'. Also
				\[ \P(A_1\cap A_2\cap A_3)=\P(A_1)\P(A_2| A_1)\P(A_3|A_1\cap A_2) \]
				mit
				\[ P(A_1)=\frac{3}{5} \]
				\[ P(A_2| A_1)=\frac{2}{4}=\frac{1}{2} \]
				\[ P(A_3|A_2\cap A_1)=\frac{1}{3} \]
				und damit 
				\[ \P(\text{Alle 3 Kugeln weiß})=\frac{1}{10} \]
			\end{bsp}
			
			\begin{bsp}
				Selbe Voraussetzungen wie im vorigen Beispiel. Nun ist
				\begin{align*}
				\P(\text{genau 2 Kugeln weiß})&=\P(\text{wws})+\P(\text{wsw})+\P(\text{sww})\\
				&=\P(A_1\cap A_2\cap A_3^c)+\P(A_1\cap A_2^c+A_3)+\P(A_1^c\cap A_2 \cap A_3)\\
				&=\frac{3}{5}\frac{2}{4}\frac{2}{3}+\frac{3}{5}\frac{2}{4}\frac{2}{3}+\frac{2}{5}\frac{3}{4}\frac{2}{3}=3\cdot\frac{12}{60}=\frac{3}{5}.
				\end{align*}
				Dieses Beispiel kann analog auf jede Anzahl an Kugeln fortgesetzt werden.
			\end{bsp}
			
			\begin{satz}[Borel-Cantelli II]
				Sei $(\Omega, \ms{S}, \P)$ ein Wahrscheinlichkeitsraum. Sei $A_n\in\ms{S}$ eine Folge unabhängiger Ereignisse. \newline
				Ist nun 
				\[ \sum_{n=0}^\infty \P(A_n)=\infty \]
				so folgt
				\[ \P(\limsup_{n\to\infty} A_n)=1 \]
			\end{satz}
			
			\begin{bew}
				Definition des $\limsup$ war:
				\[ \limsup_{n\to\infty} A_n=\bigcap_{n\in\N}\bigcup_{k\ge n} A_k \]
				und damit nach den de Morgan'schen Regeln:
				\[ (\limsup_{n\to\infty} A_n)^c=\bigcup_{n\in\N}\bigcap_{k\ge n} A_k^c \]
				Betrachten wir nun $\bigcap_{k\ge n} A_k^c$. Die $A_k^c$ sind nun auch unabhängig. (siehe Übung) Also:
				\[ \bigcap_{k\ge n} A_k^c = \lim_{N\to\infty} \bigcap_{k=n}^N A_k^c \]
				\[ \Rightarrow \P\left(\bigcap_{k\ge n} A_k^c\right)=\lim_{N\to\infty}\prod_{k=n}^\infty \P(A_k^c)=\prod_{k=n}^\infty \P(A_k^c)=\prod_{k=n}^\infty (1-\P(A_k)) \]
				mit $1+x\le e^x$ folgt
				\[ \prod_{k=n}^\infty (1-\P(A_k)) \le\prod_{k=n}^\infty e^{-\P(A_k)}=e^{-\sum_{k\ge n}^\infty \P(A_k)}=\lim_{n\to \infty} -e^{-n}=0 \]
				Damit:
				\[ \P\left(\bigcup_{n=1}^\infty\bigcap_{k=n}^\infty A_k\right)\le \sum_{n=1}^\infty \P\left(\bigcap_{k=n}^\infty A_k\right)=\sum_{n=1}^\infty 0=0 \]\arge
			\end{bew}
			
			\section{Der Fortsetzungssatz für Maßfunktionen}
			
			In diesem Abschnitt werden wir den folgenden Satz beweisen:
			\begin{satz}[Fortsetzungssatz für Maßfunktionen]
				Sei $\mu$ ein Maß auf einem Ring $\mf{R}$. Dann gilt:
				\begin{enumerate}
					\item $\mu$ kann zu einem Maß $\widetilde{\mu}$ auf dem erzeugten Sigmaring fortgesetzt werden. 
					\item Wenn $\mu$ sigmaendlich ist, dann ist $\widetilde{\mu}$ eindeutig bestimmt. 
				\end{enumerate}
				
			\end{satz}
			
			\begin{bem}
				Wir werden $\widetilde{\mu}$ im Folgenden immer mit $\mu$ bezeichnen, da es nicht wichtig ist, ob wir auf einem Ring oder auf dem erzeugten Sigmaring arbeiten. 
			\end{bem}
			
			\begin{bem}
				Die Motivation für diesen Satz ist das klassische Ausschöpfungs-, bzw Exhaustionsprinzip, das z.B. Archimedes und Eudoxos bearbeitet haben. Dabei wurde die Fläche eines Kreises durch Rechtecke approximiert. Damit ist ($A$ ist die Fläche des Kreises, $B$ die Fläche der Vierecke)
				\[ \mu^+(A)=\inf\{\mu(B):A\subseteq B, B\in\mf{R}\} \]
				\[ \mu^-(A)=\sup\{\mu(B):B\subseteq A, B\in\mf{R}\} \]
				wenn $\mu^+(A)=\mu^-(A)$, dann ist $A$ messbar (im Sinn von Jordan). Dann $\mu^*$ das Jordon-Maß. 
				\begin{align*}
				\mu^*(A) &= \inf\left(\sum_{n\in\N}\mu(B_n)\right), B_n\in\mf{R}, A\subseteq \bigcup_{n\in\N} B_n\\
				&=\inf\{\sum_{n\in\N}\mu(B_n): B_n\in\mf{R}, A\subseteq \sum_{n\in\N} B_n\}
				\end{align*}
				Die letzte Gleichheit folgt durch Zeigen von $\le$ und $\ge$. 
				
			\end{bem}
			
			\begin{defi}
				Das Maß von einem Maß $\mu$ erzeugte Maß
				\[ \mu^*(A) =\inf\{\sum_{n\in\N}\mu(B_n): B_n\in\mf{R}, A\subseteq \sum_{n\in\N} B_n\} \]
				heißt äußeres Maß oder Jordan-Maß. Hierbei wird 
				\[ \inf\varnothing=\infty \]
				gesetzt. 
			\end{defi}
			
			\begin{defi}
				Ist $\mu(\Omega)<\infty$, so ist 
				\[ \mu_*(A)=\mu(\Omega)-\mu^*(A^c) \]
				das innere Maß. 
			\end{defi}
			
			\begin{defi}[vorläufige Definition]
				$A$ heißt messbar, falls
				\[ \forall E\in\mf{R}: \mu(E)=\mu^*(E\cap A)+\mu^*(E\setminus A). \]
			\end{defi}
			
			\begin{defi}
				$A$ heißt messbar, wenn
				\[ \forall B\subseteq \Omega: \mu^*(B)=\mu^*(B\cap A)+\mu^*(B\setminus A). \]
			\end{defi}	
			
			\begin{satz}[Eigenschaften von äußeren Maßfunktionen]
				Sei $\mu$ ein Maß und $\mu^*$ das von $\mu$ erzeugte äußere Maß. Dann gilt:
				\begin{enumerate}
					\item $\mu^*(A)\ge 0$
					\item $\mu^*(\varnothing)=0$
					\item Monotonie: 
					\[ A\subseteq B\subseteq \Omega \Rightarrow \mu^*(A)\le \mu^*(B) \]
					\item Sigmasubadditivität: 
					\[ A\subseteq \bigcup_{n\in\N} A_n\subseteq \Omega \]
					\[ \Rightarrow \mu^*(A)\le \sum_{n\in\N} \mu^*(A_n) \]
				\end{enumerate}
			\end{satz}
			
			\begin{defi}
				Eine Funktion $\mu^*: 2^\Omega \to [0,\infty]$ heißt eine äußere Maßfunktion, wenn sie die Eigenschaften 1.-4. besitzt.
			\end{defi}	
			
			\begin{bem}
				Will man zeigen, dass $\mu^*$ ein äußeres Maß ist, so muss man nur 1.,2. und 4. zeigen, 3. folgt dann automatisch. 
			\end{bem}
			
			\begin{bew}
				Eigenschaften 1. und 2. sind klar. Bleibt also noch 4. zu zeigen, 3. folgt ja automatisch. \newline
				Sei also $A\subseteq \bigcup_{n\in\N} A_n$. Zu zeigen ist nun, dass 
				\[ \mu^* (A)\le \sum_{n\in\N} \mu^*(A_n) \]
				wenn $\sum_{n\in\N} \mu^*(A_n)=\infty$, so sind wir fertig.\newline
				Sei also $\sum_{n\in\N} \mu^*(A_n)<\infty$. Dann ist
				\[ \mu^*(A_n)=\inf\{\sum_{k\in\N}\mu(B_k): A_n\subseteq \bigcup B_k, B_k\in\mf{R}\} \]
				Sei $\varepsilon >0$. Für $B_{nk}\in\mf{R}: A_n\subseteq\bigcup_{k\in\N} B_{nk}$ und $\sum_{k\in\N} \mu(B_{nk})\le \mu^*(A_n)+\frac{\varepsilon}{2}$. Nun ist
				\[ \bigcup_{n\in\N} A_n\subseteq \bigcup_{n\in\N}\bigcup_{k\in\N} B_{nk} \]
				und damit
				\[ \mu^*\left(\bigcup_{n\in\N} A_n\right)\le \sum_{n\in\N}\sum_{k\in\N} \mu(B_nk) \le \sum_{n\in\N} (\mu^*(A_n)+\frac{\varepsilon}{2^n})=\sum_{n\in\N} \mu^* (A_n)+\varepsilon \]
				\[ \Rightarrow \mu\left(\bigcup_{n\in\N}A_N\right)\le\sum_{n\in\N} \mu^* (A_n) \]
				\arge
			\end{bew}
			
			\begin{bsp}
				Sei $|\Omega|\ge 3$ und 
				$$\mu^*(A)=\left\{\begin{array}{l}
					0: A=\varnothing\\
					1: A\notin\{\varnothing, \Omega\}, A\subseteq\Omega\\
					2: A=\Omega
				\end{array}\right. $$
			\end{bsp}
			
			\begin{defi}
				$A\subseteq \Omega$ heißt messbar ($\mu^*$-messbar), wenn 
				\[ \forall B\subseteq \Omega: \mu^*(B)=\mu^*(B\cap A)+\mu^*(B\cap A^c). \]
			\end{defi}
			
			\begin{bem}
				Um die Messbarkeit von $A$ zu zeigen, genügt es zu zeigen, dass
				\[ \mu^*(B)\ge \mu^*(B\cap A)+\mu^*(B\cap A^c), \]
				da die Ungleichung "`$\le$"' trivialerweise immer erfüllt ist. 
			\end{bem}
			
			\begin{defi}
				$\mf{M}_{\mu^*}$ bezeichnet das System aller $\mu^*$-messbaren Mengen. Ist klar, um welches Maß $\mu^*$ es sich handelt (oder das egal ist), so schreiben wir einfach $\mf{M}$.
			\end{defi}
			
			\begin{satz}
				\begin{enumerate}
					\item $\mf{M}$ ist eine Sigmaalgebra, $\mu^*|_\mf{M}$ ein Maß.
					\item Wenn $\mu^*$ von einem Maß $\mu$ auf einem Ring $\mf{R}$ erzeugt wird und $\mu^*(B)=\mu(B)$, so folgt $\mf{R}\subseteq \mf{M}$.
				\end{enumerate}  
			\end{satz}
			
			\begin{bew}
				Wir beweisen zunächst 2.:\newline
				Sei $B\subset\Omega, A\in\mf{R} B_n\in\mf{R}, B\subseteq \bigcup_{n\in\N} B_n, \mu^*(B)<\infty$. Dann ist
				\begin{align*}
				\sum_{n\in\N}\mu(B_n)&=\sum_{n\in\N}\mu\left((B_n\cap A)\cup(B_n\cap A^c)\right)\\
				&=\sum_{n\in\N} \left(\mu(B_n\cap A)+\mu(B_n\setminus A)\right)\\
				&=\sum_{n\in\N}\mu(B_n\cap A)+\sum_{n\in\N} \mu(B_n\setminus A)\\
				&\ge \mu^*(B\cap A)+\mu^*(B\cap A^c)
				\end{align*}
				\[ \Rightarrow \mu^*(B)\ge \mu^*(B\cap A)+\mu^*(B\setminus A) \]
				\arge
				Sei nun $A\in \mf{R}, A\subseteq \bigcup A_n, A_n\in\mf{R}$. 
				\[ \mu(A)\le \sum \mu(A_n) \]
				wurde schon gezeigt. Sei jetzt $A_1=A$, $A_n=\varnothing$ für $n>1$. Dann folgt 
				\[ \mu^*(A)\ge\mu(A), \]
				A ist also messbar.\newline\newline
				Für 1. erste Behauptung: $\mf{M}$ ist Algebra und $\mu^*|_m$ ist additiv. Wir wollen zeigen:
				\[ A_1,A_2\text{ messbar}\Rightarrow A_1\cup A_2 \text{ messbar} \]
				\[ A \text{ messbar}\Rightarrow A^c \text{ messbar} \]
				Das zweite folgt direkt daraus, dass ${A^c}^c=A$ und die Definition von "`messbar"' diesbezüglich symmetrisch ist. \newline
				Für das erste sei $B\subseteq\Omega$. Nun ist $A_1$ messbar, also
				\[ \mu^*(B)=\mu^*(B\cap A_1)+\mu^*(B\cap A_1^c) \]
				und mit
				\[ \mu^*(B\cap A_1)=\mu^*(B\cap A_1\cap A_2)+\mu^*(B\cap A_1\cap A_2^c) \]
				\[ \mu^*(B\cap A_1^c)=\mu^* (B\cap A_1^c\cap A_2)+\mu^*(B\cap A_1^c\cap A_2^c) \]
				ergibt sich:
				\begin{align*}
				\mu^*(B)&=\mu^*(B\cap A_1\cap A_2)+\mu^*(B\cap A_1\cap A_2^c)+\mu^* (B\cap A_1^c\cap A_2)+\mu^*(B\cap A_1^c\cap A_2^c)\\
				&\ge \mu^*\left((B\cap A_1\cap A_2)\cup(B\cap A_1\cap A_2^c)\cup (B\cap A_1^c\cap A_2)\right)+ \mu^*(B\cap (A_1\cup A_2)^c)\\
				&=\mu^*(B\cap(A_1\cup A_2))+\mu^*(B\cap (A_1\cup A_2)^c)
				\end{align*}
				Damit ist $\mf{M}$ tatsächlich eine Algebra.\newline
				Um nachzuweisen, dass $\mu^*|_\mf{M}$ additiv ist, seien $A_1, A_2\in \mf{M}$, $A_1\cap A_2=\varnothing$. Über die Messbarkeit von $A_1$ erhalten wir:
				\[ \mu^*(A_1\cup A_2)=\mu^*((A_1\cup A_2)\cap A_1)+\mu^*((A_1\cup A_2)\cap A_1^c)=\mu^*(A_1)+\mu^*(A_2) \]
				Nun bleibt noch zu zeigen, dass $\mf{M}$ Sigmaalgebra ist, seien also $A_n\in \mf{M}, A_n\text{ disjunkt}, B\subseteq \Omega$.\newline
				\zz:
				\[ \mu^*(B)\ge \mu^*\left(B\cap \bigcup_{n\in \N} A_n\right)+\mu^*\left(B\setminus \bigcup_{n\in\N} A_n\right) \]
				Wir wissen schon:
				\begin{align*}
				\mu^*(B)&=\mu^*\left(B\cap \bigcup_{n=1}^N A_n\right)+\mu^*\left(B\setminus\bigcup_{n=1}^N A_n\right)\\
				&\ge \mu^*\left(B\cap \bigcup_{n=0}^N A_n\right)+\mu^*\left(B\setminus\bigcup_{n\in\N} A_n\right)\\
				&=\sum_{n=1}^N\mu^*(B\cap A_n)+\mu^*\left(B\setminus\bigcup_{n\in\N} A_n\right)\\
				\end{align*}
				Für $n\to\infty$ erhalten wir also
				\[ \mu^*(B)\ge \sum_{n\in\N} \mu^*(B\cap A_n)+\mu^*\left(B \setminus \bigcup_{n\in\N} A_n\right)\ge\mu^*\left(\bigcup_{n\in\N} (B\cap A_n)\right)+\mu^*\left(B\setminus\bigcup_{n\in\N} A_n\right) \]
				\arge
			\end{bew}
			
			\begin{bem}
				Der erste Teil des Fortsetzungssatzes ist damit bewiesen. Bleibt also noch der folgende Satz zu zeigen:
			\end{bem}
			
			\begin{satz}
				Ist $\widetilde{\mu}$ eine Fortsetzung von $\mu$ auf $\mf{R}_\sigma(\mf{R})$ ist, dann gilt 
				\[ \widetilde{\mu}=\mu^*|_{\mf{R}_\sigma} \]
			\end{satz}
			
			\begin{satz}
				Ist $\mu$ auf $\mf{R}$ sigmaendlich, dann auch auf dem erzeugten Sigmaring.
			\end{satz}
			
			\begin{bew}
				\zz:
				\[ \mf{R}^*=\{A\subseteq \Omega: A \text{ ist sigmaendlich}\}=\{A\subseteq \exists B_1\in\mf{R}, n\in\N: \mu(B_n)<\infty \land A\subseteq\bigcup B_n\} \]
				ist Sigmaring. \newline
				\begin{itemize}
					\item $A,B\in\mf{R}^*\Rightarrow A\setminus B\in\mf{R}^*$: trivial
					\item $A_n\in\mf{R}^*, n\in\N\Rightarrow \bigcup A_n\in\mf{R}^*$:\newline
					Sei $A_n\subseteq \bigcup B_{nk}, B_{k}\in\mf{R}, \mu(B_{nk})<\infty$. Dann ist
					\[ \bigcup A_n\subseteq \bigcup_{n}\bigcup_k B_{nk} \]
					und damit folgt die Behauptung.
				\end{itemize}
			\end{bew}
			
			\begin{satz}
				Für $A\in\mf{R}_\sigma(\mf{R}): \widetilde{A}\le \mu^*(A)$
			\end{satz}
			
			\begin{bew}
				Sei $A\in\mf{R}, A\subseteq \bigcup B_n, B_n\in\mf{R}$. Nun gilt:
				\[ \sum_{n\in\N}\mu(B_n)=\sum_{n\in\N} \widetilde{\mu}(B_n)\ge \widetilde{\mu}\left(\bigcup B_N\right)\ge \widetilde{\mu}(A) \]
				Nimmt man das $\inf$ über alle $(B_n)_{n\in\N}$, so erhält man $\mu^*(A)$.
			\end{bew}
			
			\begin{satz}
				$\widetilde{\mu}(A)=\mu^*(A)$ (siehe oben)
			\end{satz}
			
			\begin{bew}
				Ist $A$ sigmaendlich, so folgt:
				\[ \exists B_n\in\mf{R}, n\in\N, \mu(B_n)<\infty, a\subseteq \bigcup B_n, \]
				wobei wir die $B_n$ oBdA als disjunkt annehmen können, da wir sie notfalls disjunkt machen können.\newline
				Nun ist
				\[ \widetilde{\mu}(A)=\widetilde{\mu}\left(A\cap\bigcup_{n\in\N}B_n\right)=\widetilde{\mu}\left(\bigcup_{n\in\N}A\cap B_n\right)=\sum_{n\in\N} \widetilde{\mu}(A\cap B_n) \]
				Nun zeigen wir:
				\[ \widetilde{\mu}(A\cap B_n)=\mu^*(A\cap B_n), \]
				dann können wir die obere Gleichungskette nach hinten durchlaufen und sind fertig.\newline
				Also: Wir wissen: 
				\[ \widetilde{\mu}(A\cap B_n)\le \mu^*(A\cap B_n) \]
				\[ \widetilde{\mu}(A^c\cap B_n)\le\mu^*(A^c\cap B_n) \]
				Außerdem, da $A$ messbar:
				\[ \mu(B_n)=\mu^*(B_n)=\mu^*(A^c\cap B_n)+\mu(A\cap B_n)\ge \widetilde{\mu}(A^c\cap B_n)+\widetilde{\mu}(A^c\cap B_n)=\widetilde{\mu}(B_n)=\mu(B_n), \]
				womit für $\ge$ auch $=$ folgt und $\widetilde{\mu}(A\cap B_n)=\mu^*(A\cap B_n)$ bewiesen ist. Damit folgt also auch:\[ \widetilde{\mu}=\mu^*|_{\mf{R}_\sigma(\mf{R})} \]
			\end{bew}
			
			\begin{bem}
				Nun ist der Fortsetzungssatz für Maßfunktionen vollständig bewiesen.
			\end{bem}
			
			\section{Zusammenhang zwischen dem Maß auf dem Ring und dem Maß auf dem Sigmaring}
			\begin{satz} [Approximationstheorem I]
				Sei $\mu$ ein sigmaendliches Maß auf einem Ring $\mf{R}$. Sei $A\in\mf{R}_\sigma(\mf{R}), \mu(A)<\infty$. Dann gilt
				\[ \forall \varepsilon>0: \exists B\in\mf{R}: \mu(A\Delta B)<\varepsilon \]
			\end{satz}
			
			\begin{bew}
				Mit $\mu(A)<\infty$ und
				\[ \mu(A)=\mu^*(A)=\inf\left\{\sum_{n\in\N}\mu(B_n): B_n\in\mf{R}, A\subseteq \bigcup_{n\in\N} B_n\right\} \]
				folgt: Wir wählen ein $(B_n)$, sodass $\sum_{n\in\N} \mu(B_n)\le \mu(A)+\frac{\varepsilon}{2}$. Das geht, weil wir ja beliebig nahe an das Infimum herankommen können. Wir wählen nun $N$ so, dass $\sum_{n>N}\mu(B_n)<\frac{\varepsilon}{2}$. Sei weiters
				\[ B:=\bigcup_{n=1}^N B_n\in\mf{R}. \]
				Dann folgt:
				\[ \mu(A\Delta B)=\mu (A\setminus B)+\mu(B\setminus A) \]
				Außerdem:
				\[ A\setminus B\subseteq\bigcup_{n\in\N} B_n\setminus \bigcup_{n=1}^N B_n\subseteq \bigcup_{n>N}B_n. \]
				Damit gilt:
				\[ \mu(A\setminus B)\le\sum_{n>N} \mu(B_n)<\frac{\varepsilon}{2} \]
				\[ \mu(B\setminus A)\le \mu\left(\left(\bigcup_{n\in\N} B_n\right)\setminus A\right)=\mu\left(\bigcup_{n\in\N} B_n\right)-\mu(A)\le \sum_{n\in\N} \mu(B_n)-\mu(A)<\frac{\varepsilon}{2} \]
				und wir sind fertig. 
			\end{bew}
			
			\begin{bem}
				Es gilt auch 
				\[ |\mu(A)-\mu(B)|\le \mu(A\Delta B) \]
			\end{bem}
			
			\begin{bem}
				Wir nehmen nun an, dass $\Omega\in\mf{R}_\sigma(\mf{R})$, der erzeugte Sigmaring ist also schon eine Sigmaalgebra. 
			\end{bem}
			
			
			\begin{defi}
				Sei $(\Omega, \mf{S}, \mu)$ Maßraum. Ist $\mu(A)=0$, so heißt $A$ Nullmenge.
				
			\end{defi}
			
			\begin{satz}
				Ist $A$ messbar, so kann man $A$ schreiben als Vereinigung einer Menge aus dem Sigmaring und einer Nullmenge, also
				\[ A=F\cup N, F\in\mf{R}_\sigma, N\subseteq M\in\mf{R}_\sigma:\mu(M)=0 \]
			\end{satz}
			
			\begin{bew}
				Sei $A\subseteq \Omega$. Mit 
				\[ \mu^*(A)=\inf\left\{\sum_{n\in\N} \mu(B_n), B_n\in\mf{R}, A\subseteq \bigcup_{n\in\N}B_n \right\} \]
				erhalten wir über
				\[ \sum_{n\in\N} \mu(B_n)\ge \mu\left(\bigcup_{n\in\N}B_n\right) \]
				das folgende:
				\[ \mu^*(A)\ge\inf\left\{\mu(B):B\in\mf{R}_\sigma, A\subseteq B\right\} \]
				und damit folgt
				\[ \mu^*(A)=\inf\left\{\mu(B):B\in\mf{R}_\sigma, A\subseteq B\right\} \]
				Wir nehmen nun ein $(C_n)$ mit $C_n\in\mf{R}, \bigcup_{n\in\N}=\Omega, C_n$ disjunkt und $\forall n\in\N: \mu(C_n)<\infty$. Ist $A\in m_{\mu^*}$ messbar, $\mu^*(A\cap C_n)<\infty$, $\mu^*(A\cap C_n)=\inf\{\mu(B):B\in\mf{R}_\sigma, A\cap C_n\subseteq B\}$, dann wählen wir ein $B_k\in\mf{R}_\sigma$, sodass
				\[ A\cap C_n\subseteq B_k, \mu(B_k)\le \mu^*(A\cap C_n)+\frac{1}{k}. \]
				Sei 
				\[ D_n=\bigcap_{k\in\N} B_k\]
				also 
				\[ A\cap C_n\subseteq D_n, \mu(D_n)=\mu^*(A\cap C_n) \]
				Analog: $E_n\in\mf{R}_\sigma, A^c\cap C_n\subseteq E_n, \mu(E_n)=\mu^*(A^c\cap C_n)$:
				\[ \mu(C_n)=\mu^*(C_n\cap A)+\mu(C_n\cap A^c)=\mu(E_n)+\mu(D_n) \]
				oBdA: $E_n,D_n\subseteq C_n$. Nun ist 
				\[ \mu(D_n)=\mu(C_n)-\mu(D_n)=\mu(C_n\subseteq D_n) \]
				Über
				\[ D_n\supseteq A\cap C_n \land F_n:=C_n\setminus E_n\subseteq A\cap C_n\]
				\[ \mu(D_n\setminus F_n)=\mu(D_n)-\mu(F_n)=0 \]
				\[ F_n\subseteq A\cap C_n\subseteq F_n\cup (D_n\setminus F_n) \]
				erhalten wir
				\[ \bigcup_{n\in\N} F_n\subseteq A\subseteq \bigcup_{n\in\N} F_n\cup \bigcup_{n\in\N}(D_n\setminus F_n) \]
				Wir betrachten
				\[ \mu\left(\bigcup_{n\in\N} (D_n\setminus F_n)\right)\le \sum_{n\in\N} \mu(D_n\setminus F_n)=0 \]
				Nun können wir $A$ schreiben als
				\[ A=F\cup N, F\in\mf{R}_\sigma, N\subseteq M\in\mf{R}_\sigma:\mu(M)=0 \]
			\end{bew}
			
			\begin{defi}
				Ein Maßraum $(\Omega,\mf{S},\mu)$ heißt vollständig, wenn 
				\[ A\in\mf{S}, \mu(A)=0, B\subseteq A\Rightarrow B\in\mf{S} \]
			\end{defi}
			
			\begin{defi}
				Sei $(\Omega,\mf{S}, \mu)$ Maßraum. Mit
				\[ \overline{\mf{S}}:=\{A\cup N, A\in\mf{S}, \exists M\in\mf{S}: N\subseteq M, \mu(B)=0 \} \]
				und
				\[ \overline{\mu}(A\cup N)=\mu(A) \]
				heißt der vollständige Maßraum $(\Omega, \overline{\mf{S}}, \overline{\mu})$ die Vervollständigung von $(\Omega, \mf{S}, \mu)$.
			\end{defi}
			
			\begin{satz}
				Ist $\mu*$ das von einem Maß $\mu$ auf dem Ring $\mf{R}$ erzeugte äußere Maß, so ist ein $A\subseteq \Omega$ messbar genau dann, wenn
				\[ \forall B\in\mf{R}: (\mu^*(B)=)\mu(B)=\mu^*(B\cap A)+\mu^*(B\setminus A). \]
				Ist zusätzlich $\mu(\Omega)<\infty$ ($\mu^*(\Omega)<\infty$), dann ist $A$ messbar, wenn 
				\[ \mu(\Omega)=\mu^*(A)+\mu^*(A^c). \]
			\end{satz}
			
			\begin{bew}
				Eine Richtung ist klar. \newline
				Für die andere sei $C\subseteq \Omega$. \newline
				\zz:
				\[ \mu^*(C)=\mu^*(C\cap A)+\mu^*(C\setminus A), \]
				also nur die Richtung $\ge$, $\le$ wird durch die Subadditivität schon garantiert. Sei nun $(B_n)$ eine Überdeckung von $C$, $C\subseteq \bigcup_{n\in\N} B_n$. Nun gilt:
				\[ \sum_{n\in\N}\mu(B_n)=\sum_{n\in\N} \mu^*(B_n\cap A) + \sum_{n\in\N}\mu^*(B_n\setminus A)\ge \mu^*(C\cap A)+\mu^*(C\setminus A) \]
				Durch Infimumbildung ergibt sich die Behauptung.   \newline
				Sei nun $\mu(\Omega)<\infty$. Sei $E\in\mf{R}$, dann ist
				\[ \mu^*(A)=\mu^*(A\cap E)+\mu^*(A\cap E^c) \]
				\[ \mu^*(A^c)=\mu^*(A^c\cap E)+\mu^*(A^c\cap E^c) \]
				und somit 
				\[ \mu(\Omega)=\mu^*(A)+\mu^*(A^c)=\mu^*(A\cap E)+\mu^*(A\cap E^c)+\mu^*(A^c\cap E)+\mu^*(A^c\cap E^c)\ge \mu(E)+\mu(E^c)=\mu(\Omega) \]
				Damit folgt statt $\ge$ Gleichheit. 
			\end{bew}
			
		\section{Maße auf $(\R, \mf{B})$}
			
			Die Frage, die sich stellt ist: Ist $\mu^*$ auf $\R$ frei definiert, wann gilt $\mf{B}\subseteq \mf{M}_{\mu^*}$?
			
			\begin{defi}
				Seien $A,B\in\R$. Dann ist der Abstand 
				\[ d(A,B):=\inf\{|x-y|, x\in A, y\in B\}. \]
				Ein äußeres Maß $\mu^*$ heißt arithmetisch, wenn 
				\[ \forall A,B\in\R: d(A,B)>0\Rightarrow \mu^*(A\cup B)=\mu^*(A)+\mu^*(B) \]
			\end{defi}
			
			\begin{satz} [Satz von Carathéodory]
				$\mf{B}\subseteq \mf{M}_{\mu^*}$ genau dann, wenn $\mu^*$ arithmetisch ist.
			\end{satz} 
			
			\begin{bew}
				Dieser Satz erfordert einige Trickserei und wird hier nicht bewiesen.
			\end{bew}
			
		\section{Maße auf $(\R, \mf{B})$, zweiter Anlauf}
			Im folgenden ist immer $\mf{T}:=\{(a,b], a\le b, a,b\in\R\}$.
			
			\begin{satz}
				$\mu$ ist genau dann endliches Maß auf $\mf{T}$, wenn
				\[ \forall x\in\R\exists\delta(x)>0: \mu((x-\delta(x),x])<\infty \]
			\end{satz}	
			
			\begin{bew}
				In der Übung.
			\end{bew}
			
			\begin{defi}
				$\mu$ auf $(\R, \mf{B})$ heißt Lebesgue-Stieltjes Maß, oder lokalendlich, wenn jede beschränkte Borelmenge endliches Maß hat. 
			\end{defi}
			
			\begin{bem}
				Dazu muss man ein Maß finden, dass für alle $a<b$ $\mu((a,b])$ festlegt. Dies ist nicht ganz frei möglich, die Additivität muss erfüllt werden, also 
				\[ \mu((a,c])=\mu((a,b])+\mu((b,c]). \]
				
				Wir beginnen dazu mit einem Spezialfall, dass $\mu(\R)<\infty$:
			\end{bem}
			
			\begin{bsp}
				Sei
				\[ F(x)=\mu((-\infty,x])<\infty \]
				dann ist für $a<b$:
				\[ (-\infty,a]\cup (a,b]=(-\infty,b] \]
				\[ \mu((-\infty,a])+\mu((a,b])=\mu((-\infty,b]) \]
				\[ \Rightarrow \mu((a,b])=F(b)-F(a) \]
			\end{bsp} 
			
			\begin{defi}
				$F:\R\to\R$ heißt Verteilungsfunktion von $\mu$, wenn $\mu((a,b])=F(b)-F(a)$. 
			\end{defi}
			
			\begin{bem}
				\[ F(x)=\mu((0,x]), x\ge 0, \]
				damit:
				\[ \mu((a,b])=F(b)-F(a) \]
				und $F(0)=0$. \newline
				für
				\[ \mu((x,0])=F(0)-F(x) \]
				\[ \Rightarrow F(x)=-\mu((x,0]), \]
				eine Verteilungsfunktion muss also die Form
				\[ F(x)=\left\{\begin{array}{l}
				\mu((0,x]): x\ge 0\\
				-\mu((x,0]): x<0
				\end{array}\right. \]
				dies funktioniert, siehe Aufgaben (\zz $\mu((a,b])=F(b)-F(a)$). Dies fassen wir im folgenden Satz zusammen:
			\end{bem}
			
			\begin{satz}
				Zu jeder Lebesgue-Stieltjes Maßfunktion gibt es eine Verteilungsfunktion. Diese ist bis auf eine additive Konstante eindeutig bestimmt. 
			\end{satz}
			
		\section{Ergänzungen zu bedingten Wahrscheinlichkeiten}
			\begin{satz}[Satz von der vollständigen Wahrscheinlichkeit]
				Sei $(\Omega, \mf{S}, \P)$ ein Wahrscheinlichkeitsraum.
				Sei dann $(B_i, i\in I)$ eine Partition, $I$ höchstens abzählbar mit $B_i\in\mf{S}, \P(B_i)>0, \sum_{i\in I}B_i=\Omega$ und $A\in\mf{S}$. Dann ist
				\[ \P(A)=\sum_{i\in I} \P(B_i)\P(A|B_i). \]
			\end{satz}
			
			\begin{bew}
				Es gilt:
				\[ \P(A)=\P(A\cap \Omega)=\P\left( A\cap\bigcup_{i\in I}B_i\right)=\P\left(\bigcup_{i\in I} A\cap B_i\right)=\sum_{i\in I} \P(A\cap B_i)=\sum_{i\in I} \P(B_i)\P(A|B_i) \]
			\end{bew}
			
			\begin{satz}[Satz von Bayes]
				Sei $(\Omega, \mf{S}, \P)$ ein Wahrscheinlichkeitsraum.
				Sei wieder $(B_i, i\in I)$ eine Partition, $I$ höchstens abzählbar mit $B_i\in\mf{S}, \P(B_i)>0, \sum_{i\in I}B_i=\Omega$ und $A\in\mf{S}$. Zusätzlich zu vorher gelte $\P(A)>0$. Dann gilt:
				\[ \P(B_i|A)=\frac{\P(A\cap B_i)}{\P(A)}=\frac{\P(B_i)\P(A|B_i)}{\P(A)}=\frac{\P(B_i)\P(A|B_i)}{\sum_{j\in I}\P(B_j)\P(A|B_j)} \]
			\end{satz}
			
			\begin{defi}
				Die $\P(B_i)$ in den Sätzen vorher heißen a-priori Wahrscheinlichkeiten, $\P(B_i|A)$ die a-posteriori Wahrscheinlichkeiten. 
			\end{defi}
			
			\begin{bsp}
				Es gibt vier Blutgruppen, A, B, AB, 0. Die Blutgruppe der Frau ist A, die des Sohnes ist 0. Wie ist die Wahrscheinlichkeit für die Blutgruppe des Mannes?\newline
				Mit zusätzlichem Wissen über Genetik, kann man über die Wahrscheinlichkeiten $p_a,p_b,p_0$ für das Auftreten der Allele $a,b,0$ die Wahrscheinlichkeit der Blutgruppen ausrechnen. In der Bevölkerung haben Blutgruppe 0 40\% der Bevölkerung, Blutgruppe A 47\%, B 9\% und AB 4\%. Damit erhalten wir:
				\[ 0.4=p_0^2 \]
				\[ 0.47=p_a^2+2p_ap_0 \]
				\[ 0.09=p_b^2+2p_bp_0 \]
				\[ 0.04=2p_ap_b \]
				Eine gute Approximation ist
				\[ p_a\approx \frac{9}{30}, \:\:\:\:\:p_b\approx \frac{2}{30}, \:\:\:\:\: p_0\approx\frac{19}{30}. \]
				Damit erhält man:
				\[ \P(\text{Sohn 0|0})=1 \]
				\[ \P(\text{Sohn 0|A})=\frac{1}{2} \]
				\[ \P(\text{Sohn 0|B})=\frac{1}{2} \]
				und 
				\[ \P(\text{0|Sohn 0})=\frac{p_0^2}{p_0^2+p_ap_0+p_bp_0}=p_0, \]
				genauso für A und B, also ist die Wahrscheinlichkeit für Blutgruppe 0 am größten. 
			\end{bsp}
			
		\section{Eigenschaften von Verteilungsfunktionen}
			\begin{satz}
				Sei $F:\R\to\R$ eine Verteilungsfunktion. Dann gilt:
				\begin{enumerate}
					\item Monotonie:
					\[ a\le b\Rightarrow F(a)\le F(b) \]
					\item Rechtsstetigkeit:
					\[ b_n\downarrow b\Rightarrow F(b_n)\downarrow F(b) \]
				\end{enumerate}
			\end{satz}
			
			\begin{bew} $\text{  }$ %newline
				
				\begin{enumerate}
					\item trivial
					\item Sei $a\le b$, $b_n\downarrow b$, also
					\[ (a,b_n]\downarrow (a,b] \]
					und über die Stetigkeit von oben von Maßen
					\[ \mu((a,b_n])\to \mu((a,b]) \]
					\[ \Rightarrow F(b_n)-F(a)\to F(b)-F(a) \]
				\end{enumerate}
			\end{bew}
			
			\begin{satz}
				Sei $F:\R\to\R$ nichtfallend und rechtsstetig. Dann ist durch 
				\[ \mu_F((a,b]):=F(b)-F(a) \]
				ein Maß auf $\mf{T}=\{(a,b], a\le b, a,b,\in\R\}$ definiert.
			\end{satz}
			
			\begin{bew}$\text{  }$
				\begin{enumerate}
					\item $\mu_F$ ist Inhalt:\newline
					\[ \mu_F(\varnothing)=\mu_F((a,a])=F(a)-F(a)=0 \]
					und
					\[ \mu_F((a,b])\geq 0 \]
					folgt sofort. Für die Additivität benutzen wir, dass $\mf{T}$ ein Semiring im engeren Sinn ist. Ist außerdem die Vereinigung zweier Intervalle wieder ein Intervall, so hat die Vereinigung die Form:
					\[ (a,b]\cup (b,c]=(a,c], \]
					also
					\[ \mu_F((a,b])+\mu_F((b,c])=F(b)-F(a)+F(c)-F(b)=F(c)-F(a)=\mu_F((a,c]). \]
					\item Sigmaadditivität:\newline
					Sei also 
					\[ (a,b]=\sum_{n\in\N} (a_n,b_n]. \]
					Dann wissen wir schon, dass, da $\mu_F$ Inhalt ist, gilt:
					\[ \mu_F((a,b])\ge\sum_{n\in\N} \mu_F((a_n,b_n]). \]
					\zz:
					\[ \mu_F((a,b])\le\sum_{n\in\N} \mu_F((a_n,b_n]) \]
					Sei $a'>a$, $F(a')\le F(a)+\varepsilon$ und $b_n'>b_n$, $F(b_n')\le F(b_n)+\frac{\varepsilon}{2^n}$. Dann ist
					\[ [a',b]\subseteq (a,b]=\bigcup_{n\in\N}(a_n,b_n]\subseteq \bigcup_{n\in\N} (a_n,b_n') \]
					Nach dem Satz von Heine-Borel gibt es also eine endliche Teilüberdeckung, also
					\[ \exists N\in\N: [a',b]\subseteq\bigcup_{n=1}^N(a_n,b_n') \]
					\[ \Rightarrow (a',b]\subseteq\bigcup_{n=1}^N (a_n,b_n']. \]
					Nun haben wir eine endliche Vereinigung und wir erhalten aus der endlichen Subadditivität des Inhalts $\mu_F$:
					\[ \mu_F((a',b])\le\sum_{n=1}^{N}\mu_F((a_n,b_n'])\le\sum_{n=1}^{\infty}\mu_F((a_n,b_n']) \]
					\[ F(b)-F(a)-\varepsilon\le F(b)-F(a')\le\sum_{n\in\N}(F(b_n')-F(a_n) \]
					\[ \Rightarrow F(b)-F(a)-\varepsilon\le\sum_{n\in\N} (F(b_n)+\frac{\varepsilon}{2^n}-F(a_n))=\sum_{n\in\N}(F(b_n)-F(a_n))+\varepsilon \]
					\[ \Rightarrow F(b)-F(a)\le \sum_{n\in\N}(F(b_n)-F(a_n))+2\varepsilon, \]
					da $\varepsilon$ beliebig, erhalten wir die Behauptung. 
					
				\end{enumerate}
			\end{bew}
			
			\begin{bem}
				Für Wahrscheinlichkeitsmaße $\mu$ hat die Verteilungsfunktion 
				\[ F(x):=\mu((-\infty,x]) \]
				die zusätzlichen Eigenschaften:
				\begin{itemize}
					\item
					\[ 0\le F\le 1 \]
					\item
					\[ \lim_{x\to-\infty}F(x)=0 \]
					\item
					\[ \lim_{x\to +\infty} F(X)=1. \]
				\end{itemize}
				Eine Verteilungsfunktion, die das erfüllt, heißt Verteilungsfunktion im engeren Sinn. 
			\end{bem}
			
		\section{Maße von Mengen mit Verteilungsfunktionen}
			Ab diesem Kapitel werden wir offene Intervallgrenzen auch mit eckigen Klammern schreiben.\newline
			Wir wissen schon:
			\[ \mu(]a,b])=F(b)-F(a). \]
			Was passiert, für$\mu([a,b]), \mu(]a,b[), \mu([a,b[)$? 
			\[ \mu([a,b])=\mu(\bigcap_{n\in\N}]a-\frac{1}{n},b])=\lim_{n\to\infty}\mu(F(b)-F(a-\frac{1}{n}))=F(b)-F(a-0) \]
			\[ \mu(]a,b[)=\mu(\bigcup_{n\in\N}]a,b-\frac{1}{n}])=\lim_{n\to\infty}(F(b-\frac{1}{n})-F(a))=F(b-0)-F(a) \]
			\[ \mu([a,b[)=F(b-0)-F(a-0) \]
			Und damit auch 
			\[ \mu(\{x\})=\mu([x,x])=F(x)-F(x-0) (=\text{Sprunghöhe von $F\text{ in $x$}$}) \]
			\begin{satz}
				Jedes (sigma-)endliche Maß $\mu$ auf $(\Omega, \mf{S})$ lässt sich darstellen als Summe eines stetigen Maßes $\mu_c$ und eines diskreten Maßes $\mu_d$, wobei
				\begin{itemize}
					\item $\mu_d$ diskret, wenn es eine Menge $D$ gib, die höchstens abzählbar ist, sodass
					\[ \mu(D^c)=0. \]
					\item $\mu_c$ stetig, wenn 
					\[ \forall w\in\Omega: \mu_c(\{w\})=0. \]
				\end{itemize}
				Nämlich
				\[ \mu(A)=\mu(A\cap D^c)+\mu(A\cap D)=0+\mu(\bigcup_{x\in A\cap D} \{x\})=\sum_{x\in A\cap D}\mu(\{x\})=\sum_{x\in A}\mu(\{x\}) \]
			\end{satz}
			
			\begin{bew}
				In der Übung.
			\end{bew}
			
			\begin{bsp}
				Sei $\mu$ ein endliches Lebesgue-Stieltjes Maß auf $(\R,\mf{B})$. Für die Verteilungsfunktion $F(x)=\mu(]-\infty,x[)$ kann man nun, da $\mu$ dargestellt werden kann als
				\[ \mu=\mu_c+\mu_d \]
				auch zerlegen in 
				\[ F=F_c+F_d, F_d(x)=\sum_{y\le x} \mu_d(\{y\}).\]
				Wir erhalten den folgenden Satz:
			\end{bsp}
			
			\begin{satz}
				Jede diskrete Verteilungsfunktion (Verteilungsfunktion eines diskreten, endlichen Maßes) auf $\R$ lässt sich anschreiben als
				\[ F(x)=\sum_{y\le x} p(y). \]
				Ist $\sum_{y\in\R} p(y)=1$, so nennen wir $p$ Wahrscheinlichkeitsfunktion. Umgekehrt gibt es zu jeder Funktion $p$ mit $p(y)\ge 0$ eine diskrete Verteilungsfunktion. 
			\end{satz}
			
			\begin{defi}
				Ein Wahrscheinlichkeitsmaß $\P$ auf $(\R,\mf{B})$ heißt Verteilung. 
			\end{defi}
			
			\begin{satz}
				Ist eine Verteilungsfunktion $F(x)$ (stückweise) stetig differenzierbar, $f(x):=F'(x)\ge0$, so ist
				\[ \mu_F(]a,b])=\int_{a}^{b} f(x)dx. \]
				$f(x)$ heißt dann Dichtefunktion. 
			\end{satz}
			
			\begin{bem}
				Ist $\mu_F(\R)=1$, so ist
				\[ 1=\int_{-\infty}^{+\infty} f(x)dx. \]
			\end{bem}
			
			\begin{bem}
				Wir werden anstatt des Riemann-Integrals bald ein Lebesgue-Integral schreiben.
			\end{bem}
			
			\begin{bsp} [Standardnormalverteilung]
				\[ \varphi(x)=\frac{1}{\sqrt{2\pi}}e^{-\frac{x^2}{2}}. \]
				Aus der Analysis ist schon bekannt
				\[ \int_{-\infty}^{+\infty}e^{-\frac{x^2}{2}}dx=\sqrt{2\pi}, \]
				also
				\[ \int_{-\infty}^{+\infty}\varphi(x)dx=1. \]
				Wir erhalten die Verteilungsfunktion
				\[ \Phi(x):=\int_{-\infty}^{x}\varphi(x)dx. \]
				Dann ist
				\[ \P_\Phi(]a,b])=\Phi(b)-\Phi(a). \]
				Zum Beispiel also 
				\[ \P_\Phi(]-1,2])=\Phi(2)-\Phi(-1)=0.9772-0.1587=0.8185, \]
				\[ \Phi(1.67)=0.9525 \]
			\end{bsp}
			
			\begin{bsp}
				Allgemeiner nimmt man 
				\[ \mathcal{N}(\mu,\sigma^2,x)=\frac{1}{\sqrt{2\pi \sigma^2}}e^\frac{-(x-\mu)^2}{2\sigma^2} \]
			\end{bsp}
			
		\section{Mehrdimensionale Lebesgue-Stieltjes Maße und Verteilungsfunktionen}
			Der hier verwendete Maßraum ist $(\R^d, \mf{B}_d)$ mit 
			\[ \mf{B}_d:=\mf{A}_\sigma\left(\{]a,b]: a,b\in\R, a\le b\}\right), \]
			wobei die Ungleichung $a\le b$ komponentenweise zu verstehen ist, also
			\[ a\le b:\Leftrightarrow\forall i\in\{1,...,d\}: a_i\le b_i \]
			und 
			\[ ]a,b]:=]a_1,b_1]\times]a_2,b_2]\times...\times]a_d,b_d] \]
			
			\begin{defi}
				Sei $\mu$ ein Lebesgue-Stieltjes Maß auf $(\R^d, \mf{B}_d)$, wenn für beschränkte Mengen $A\in\mf{B}_d$
				\[ \mu(A)<\infty. \]
			\end{defi}
			
			\begin{bem}
				Sei $\mu$ ein endliches Maß. Dann können wir die Verteilungsfunktion wieder anschreiben als
				\[ F(x)=\mu(]-\infty,x])=\mu(]-\infty,x_1])\times...\times\mu(]-\infty,x_d]). \]
				Genügt dies, um $\mu$ festzulegen?
			\end{bem}
			
			\begin{bsp}
				Für $d=2$ erhalten wir:
				\[ \mu(]a,b])=F(b_1,b_2)-F(a_1,b_2)-F(b_1,a_2)+F(a_1,a_1). \]
				Wir können den Satz von oben also zmd. für den 2-dimensionalen Raum erweitern:
			\end{bsp}
			
			\begin{satz}
				$F$ ist eine Verteilungsfunktion von einem Lebesgue-Stieltjes Maß $\mu$, wenn 
				\begin{itemize}
					\item $F$ rechtsstetig ist, also
					\[ x_n\downarrow x\Rightarrow F(x_n)\downarrow F(x) \]
					\item $F$ monoton ist, also
					\[ a\le b\Rightarrow F(b_1,b_2)-F(a_1,b_2)-F(b_1,a_2)+F(a_1,a_2)\ge 0 \]
				\end{itemize}
			\end{satz}
			
			\begin{bew}
				Analog zum 1-dimensionalen Fall.
			\end{bew}
			
			\begin{bsp}
				Für $d\ge 2$ erhalten wir:
				\[ \mu(]a,b])=\mu(]a_1,b_1]\times...\times]a_d,b_d])=\sum_{e\in\{0,1\}^d} F(ae+b(1-e)), \]
				wobei 
				\[ ae+b(1-e)=\big(a_1e_1+b_1(1-e_1),...,a_de_d+b_d(1-e_d)\big). \]
			\end{bsp}
			
			\begin{defi}[Differenzoperatoren]
				\begin{align*} \Delta_i(a_i,b_i): &\R^{\R^d}\to\R^{\R^d};\\&f\mapsto \Delta_i (a,b)f(x_1,...,x_d):=f(x_1,...,x_{i-1},b_i,x_{i+1},...x_d)-f(x_1,...,x_{i-1},a_i,x_{i+1},...,x_d)
				\end{align*}
			\end{defi}
			
			\begin{bsp}
				$d=2$. $f(x_1,x_2)=x_1x_2$. 
				\[ \Delta_1(4,17)f(x_1,x_2)=17x_2-4x_2-13x_2-13x_2 \]
				bzw
				\[ \Delta_1(a_1,b_1)f(x_1,x_2)=(b_1-a_1)x_2 \]
				\[ \Delta_1(a_1,b_1)\Delta_2(a_2,b_2)f(x_1,x_2)=(b_1-a_1)b_2-(b_1-a_1)a_2=(b_1-a_1)(b_2-a_2) \]
			
			\end{bsp}
			\begin{bem}
				Damit ist (für $d\in\N$)
				\[ \mu_F(a,b)=\Delta_1(a_1,b_1)\Delta_2(a_2,b_2)...\Delta_d(a_d,b_d)F \]
				Und
				\[ \Delta_i(a_i,b_i) F(x_1,...,x_d)=\int_{a_i}^{b_i}\frac{\partial}{\partial x_i} F(x_1,...,x_d)dx_i \]
			\end{bem}
			
			\begin{bsp}
				Endliche Maße:
				\[ F(x)=\mu(]-\infty,x]) \]
				Wir betrachten den Spezialfall für $d=2$. Dann ist
				\[ \mu(]0,x])=F(x_1,x_2)-F(x_1,0)-F(0,x_2)+F(0,0). \]
				Setze  $F(x_1,0)=F(0,x_2)=0$. Dann ist für $x>0$
				\[ F(x_1,x_2)=\mu(]0,x_1]\times]0,x_2]) \]
				und für $x_1\ge 0, x_2 <0$
				\[ \mu(]0,x_1]\times]x_2,0])=F(x_1,0)-F(0,0)-F(x_1,x_2)+F(0,x_2)=-F(x_1,x_2). \]
				Dies lässt sich quadrantenweise durchführen. \newline
				Allgemein: 
				\[ F(x)=\mu(]\min(x,0),\max(x,0)])\sgn(x), \]
				wobei das Minimum und Maximum koordinatenweise zu verstehen ist und 
				\[ \sgn(x)=\prod_{i=1}^{n}\sgn(x_i) \]
			\end{bsp}
			
			\begin{defi}
				Das $d$-dimensionale Lebesguemaß $\lambda_d$ ist
				\[ \lambda_d(]a,b])=\prod_{i=1}^d(b_i-a_i) \]
				und
				\[ F(x_1,...,x_d)=x_1\cdots x_d. \]
				Mit Hilfe des Fortsetzungssatzes erhalten wir das Maß $\lambda_d$ auf $\mathfrak{B}_d$.\newline
				Die $\lambda_d^*$-messbaren Mengen werden mit $\mathfrak{L}_d$ ($d$-dimensionale Lebesguemengen) bezeichnet, wobei
				\[ A\in \mathfrak{L}_d\Leftrightarrow A=B\cup N, B\in\mathfrak{B}_d, \exists M\in\mathfrak{B}_d: N\subseteq M, \lambda_d(M)=0 \] 
			\end{defi}
			
			\begin{satz}
				Sei $\lambda_d$ das Lebesguemaß auf $\mathfrak{B}_d$. Dann gilt:
				\begin{itemize}
					\item $\lambda_d$ ist translationsinvariant:
					\[ A\oplus c:=\{x+c:x\in A\}, \]
					\[ A\in\mathfrak{L}_d, c\in\R^d\Rightarrow A\oplus c\in \mathfrak{L}_d, \lambda_d(A\oplus c)=\lambda_d(A) \]
				\end{itemize}
			\end{satz}
			
			\begin{bew}
				$\text{   }$
				\begin{itemize}
					\item Wir zeigen zunächst
					\[  A\in\mathfrak{L}_d, c\in\R^d\Rightarrow A\oplus c\in \mathfrak{L}_d. \]
					Sei $A=]a,b]$ und $\mathfrak{S}:=\{A\in\mathfrak{B}_d: A\oplus c\in \mathfrak{B}_d\}$. $\mathfrak{S}$ ist Sigmaalgebra, damit gilt die Behauptung. $\lambda_d(A\oplus c)=\lambda_d(A)$ für $A\in\mathfrak{B}$ ist klar, da sie auf dem Semiring gilt, damit auch auf den Borelmengen. 
				\end{itemize}
			\end{bew}
			
			\begin{satz}
				Wenn $\mu$ auf $(\R^d, \mathfrak{B}_d)$ ein translationsinvariantes Lebesgue-Stieltjes Maß ist, dann gilt
				\[ \mu=c\lambda_d, c\ge 0. \]
			\end{satz}
			
			\begin{bew}
				Wir führen den Beweis nur im Fall $d=1$, in höheren Dimensionen funktioniert der Beweis analog.\newline
				Es gilt mit $c:=\mu(]0,1])$
				\[ \mu(]a,b])=\mu(]0,b-a]) \]
				\[ \mu(]0,1])=\mu\left(\bigcup_{i=1}^n ]\frac{i-1}{n},\frac{i}{n}]\right)=\sum_{i=1}^{n}\mu(]\frac{i-1}{n},\frac{i}{n}])=n\mu(]0,\frac{1}{n}])
				\]
				\[ \Rightarrow \mu(]0,\frac{1}{n}])=\frac{c}{n} \]
				und damit schließlich
				\[ \mu(]0,\frac{m}{n}])=\mu\left(\bigcup_{i=1}^m]\frac{i-1}{n},\frac{i}{n}]\right)=m\mu(]0,\frac{1}{n}])=c\frac{m}{n}. \]
				Für $x\in\Q^+$ folgt dann $\mu(]0,x])=cx$. Für $x\in\R^+$ wähle $x_n\in\Q^+$ mit $x_n\downarrow x$, dann gilt
				\[ \mu(]0,x])=\mu\left(\bigcap_{n\in\N} ]0,x_n]\right)=\lim\limits_{n\to\infty}\mu(]0,x_n])=\lim\limits_{n\to \infty}cx_n=cx \]
				also die Behauptung
				\[ \mu(]a,b]=c(b-a)=c\lambda(]a,b]). \]
			\end{bew}
			
			\begin{defi}
				Sei $\Omega=[0,1]$, dann ist 
				\[ x\sim y:\Leftrightarrow y-x\in \Q \]
				eine Äquivalenzrelation.\newline
				Dann zerlegen wir $\Omega$ in Äquivalenzklassen und bilden mithilfe des Auswahlaxioms eine Menge $V$, die aus jeder Äquivalenzklasse genau einen Vertreter wählt. Eine solche Menge heißt Vitali-Menge und ist nicht Lebesgue-Messbar.
			\end{defi}
			
			\begin{bew}
				Wäre $V$ Lebesgue messbar, so hätte sie ein Lebesgue-Maß. Sei $q\in\Q\cap[-1,1]$. Sind nun $q_1\neq q_2$ zwei solche rationalen Zahlen, dann ist
				\[ (V\oplus q_1)\cap (V\oplus q_2)=\varnothing. \]
				Nämlich: Wäre dies nicht so, dann würde für ein $y\in(V\oplus q_1)\cap (V\oplus q_2)$ gelten
				\[ x_1:=y-q_1\in V\text{ und } x_2:=y-q_2\in V \]
				\[ \Rightarrow x_2-x_1=q_1-q_2\in \Q\Rightarrow x_1\sim x_2, x_1\neq x_2\lightning \]
				Wäre nun $V\in\mathfrak{L}$, dann auch
				\[ \bigcup_{q\in\Q\cap[-1,1]}V\oplus q=W, \]
				also
				\[ \lambda(W)=\sum_{q\in\Q\cap[-1,1]}\lambda(V\oplus q) \]
				mit 
				\[ [0,1]\subseteq W\subseteq[-1,2] \]
				folgt
				\[ 1\le\lambda(W)\le 3, \]
				nun gilt also
				\[ \lambda(V)=0\Rightarrow \lambda(W)=0 \]
				\[ \lambda(V)>0\Rightarrow\lambda(W)=\infty  \lightning \]
			\end{bew}
			
			\begin{bem}
				Übliche Schlamperei:
				\[ \R^n\times\R^m=\R^{m+n},  \]
				und
				\[ \mathfrak{B}_n\times\mathfrak{B}_m=\mathfrak{B}_{n+m} \]
			\end{bem}
			
			\begin{satz}
				Seien $\Omega_1,\Omega_2$ Mengen und $\mathfrak{C}_1, \mathfrak{C}_2$ Mengensysteme über $\Omega_1,\Omega_2$. Dann ist
				\[ \mathfrak{A}_\sigma(\mathfrak{C}_1)\times \mathfrak{A}_\sigma(\mathfrak{C}_2)=\mathfrak{A}_\sigma(\mathfrak{C}_1\otimes\mathfrak{C}_2), \]
				wobei
				\[ \mathfrak{C}_1\otimes\mathfrak{C}_2:=\{A_1\times A_2: A_1\in \mathfrak{C}_1, A_2\in\mathfrak{C}_2\} \]
				und
				\[ \mathfrak{A}_\sigma(\mathfrak{C}_1)\times\mathfrak{A}_\sigma(\mathfrak{C}_2):=\mathfrak{A}_\sigma(\mathfrak{A}_\sigma(\mathfrak{C}_1\otimes\mathfrak{C}_2))\ \]
			\end{satz}
			
			\begin{bew}
				Kann man mit dem Prinzip der guten Mengen machen, vielleicht in der Übung. 
			\end{bew}
			
		\section{Approximationssätze und Regularität}
			
			\begin{defi}
				Sei $\mu$ ein Inhalt auf $(\R^d, \mathfrak{B}_d)$ bzw $(\R^d, \mathfrak{S})$ mit $\mathfrak{B}_d\subseteq\mathfrak{S}$, dann heißt $A\in\mathfrak{S}$ regulär von oben, wenn
				\[ \mu(A)=\inf\{\mu(U):A\subseteq U, U\text{ offen}\}. \]
				$\mu$ heißt dann regulär von oben, wenn alle $A\in\mathfrak{S}$ regulär von oben. 
			\end{defi}
			
			\begin{defi}
				Sei $\mu$ ein Inhalt auf $(\R^d, \mathfrak{B}_d)$ bzw $(\R^d, \mathfrak{S})$ mit $\mathfrak{B}_d\subseteq\mathfrak{S}$, dann heißt $A\in\mathfrak{S}$ regulär von unten, wenn
				\[ \mu(A)=\sup\{\mu(K):K\subseteq A, K\text{ kompakt}\}. \]
				$\mu$ heißt dann regulär von unten, wenn alle $A\in\mathfrak{S}$ regulär von oben. 
			\end{defi}
			
			\begin{defi}
				Wenn $\mu$ bzw $A\in\mathfrak{S}$ sowohl regulär von oben als auch regulär von unten sind, dann heißen sie regulär. 
			\end{defi}
			
			\begin{satz}
				Ein regulärer Inhalt ist ein Maß. 
			\end{satz}
			
			\begin{bew}
				ist eh klar. 
			\end{bew}
			
			\begin{satz}
				Sei $\mu$ ein Lebesgue-Stieltjes Maß auf $(\R^d, \mathfrak{B}_d)$, dann ist $\mu$ regulär von oben.
			\end{satz}
			
			\begin{satz}
				Ist $\mu$ ein sigmaendliches Maß auf $(\R^d, \mathfrak{B}_d)$, so ist $\mu$ regulär von unten. 
			\end{satz}
			
			Zusammenfassend ergibt das dann:
			\begin{satz}
				Jedes Lebesgue-Stieltjes Maß ist regulär. 
			\end{satz}
			
			\begin{bew}[1. Satz]
				Wir schränken $\mu$ auf den Semiring $\mathfrak{T}=\{]a,b], a\le b, a,b\in\R^d\}$ ein. Dann erzeugt $\mu|_\mathfrak{T}$ ein äußeres Maß $\mu^*$. Mithilfe der Eindeutigkeit der Fortsetzung gilt 
				\[ \forall A\in\mathfrak{B}: \mu^*(A)=\mu(A), \]
				also 
				\[ \mu(A)=\mu^*(A)=\inf\left\{\sum_{n\in\N}\mu(]a_n,b_n]), A\subseteq\bigcup_{n\in\N} ]a_n,b_n]\right\}. \]
				Im Fall $\mu(A)=\infty$ sind wir fertig. Wir betrachten also $\mu(A)<\infty$. Dann
				\[ \exists a_n,b_n,n\in\N: A\subseteq \bigcup_{n\in\N}]a_n,b_n]\text{ und } \sum_{n\in\N}\mu(]a_n,b_n])\le\mu(A)+\varepsilon \:\:\:\forall \varepsilon>0 \]
				Wähle nun $b_n'>b_n$ sodass 
				\[\mu(]a_n,b_n'[)\le\mu(]a_n,b_n[)+\frac{\varepsilon}{2^n}\]
				Dann ($U:=\bigcup_{n\in\N} ]a_n,b_n'[$)
				\[ \mu(U)=\mu\left(\bigcup_{n\in\N} ]a_n,b_n'[\right)\le\sum_{n\in\N}\mu(]a_n,b_n'[)\le\sum_{n\in\N}\mu(]a_n,b_n])+\varepsilon \]
			\end{bew}
			
			\begin{satz}
				Ein endliches/sigmaendliches Maß $\mu$ auf $(\R^d, \mathfrak{B}_d)$ ist regulär von unten.	
			\end{satz}
			
			\begin{bew}
				Auch hier begnügen wir uns mit $d=1$, der allgemeine Fall läuft genau so, ist nur mehr Schreibarbeit.\newline
				Zunächst ist $\mu$ endlich und damit regulär von oben. Sei $A\in\mathfrak{B}$, dann gibt es ein offenes $U$ mit $A^c\subseteq U, \mu(U)\le\mu(A^c)+\varepsilon$. Bekannterweise ist dann $U^c$ abgeschlossen und
				\[ \mu(U^c)=\mu(\Omega)-\mu(U)\ge \mu(\Omega)-\mu(A^c)-\varepsilon=\mu(A)-\varepsilon, \]
				wir können also A durch eine abgeschlossene Menge approximieren. Es fehlt also noch der Schritt der Beschränktheit. $K_N:=U^c\cap[-N,N]$ ist nun also kompakt und monoton steigend, $K_N\subseteq K_{N+1}$ und $\bigcup_{N\in\N}K_N=U^c$. Weiters ist
				\[ \mu(U^c)=\mu\left(\bigcup_{N\in\N} K_N\right)=\lim\limits_{n\to\infty}\mu(K_N) \]
				\[ \Rightarrow \exists N\in\N: \mu(K_N)\ge\mu(A)-2\varepsilon, \]
				damit ist für endliche Maße die Regularität von unten gezeigt. \newline
				Sei nun $\mu$ sigmaendlich, dann 
				\[ \exists (B_n)_{n\in\N}\in\mathfrak{B}, \forall n\in\N:\mu(B_n)<\infty \]
				mit oBdA $B_n\subseteq B_{n+1}$. (oBdA weil wir ja jede Mengenfolge monoton machen können)\newline
				Für $A\in\mathfrak{B}$ gilt dann
				\[ A=A\cap\Omega=A\cap\bigcup_{n\in\N} B_n=\bigcup_{n\in\N} A\cap B_n, \]
				also ist 
				\[ \mu(A)=\lim\limits_{n\to\infty}\mu(B_n\cap A). \]
				Wähle nun $M<\mu(A)$. Also 
				\[ \exists n\in N:\mu(B_n\cap A)> M. \]
				Wir definieren das (endliche) Maß
				\[ \mu_n(A):=\mu(B_n\cap A). \]
				Damit gibt es ein kompaktes $K\subseteq A$ mit $\mu_n(K)>M$. Damit folgt
				\[ \mu(K)\ge\mu(K\cap B_n)>M. \]
				\arge
			\end{bew}
			
			\begin{bem}
				Eine Funktion $F:\R^d\to\R$ heißt Verteilungsfunktion im engeren Sinn, wenn es ein Wahrscheinlichkeitsmaß $\P$ auf $(\R^d, \mathfrak{B}_d)$ mit 
				\[ F(x)=\P(]-\infty,x])\:\:\:\left(=\P(]-\infty,x_1]\times...\times]-\infty,x_d])\right) \]
				gibt und $F$ rechtsstetig ist, also
				\[ \Delta_1(a_1,b_1)...\Delta_d(a_d,b_d)F\ge0. \]
				Zusätzlich muss ein solches $F$ nichtfallend in jeder Argumentvariable $x_1,...,x_d$ sein, also
				\[ \forall i=1,...,d:\lim\limits_{x_i\to-\infty}F(x_1,...,x_d)=0 \]  
				\[ \lim\limits_{\min(x_1,...,x_d)\to\infty}F(x_1,...,x_d)=1 \]
			\end{bem}
			
			\chapter{Das Lebesgue-Integral}
			
			Motivation für dieses Kapitel: Wir wollen einen neuen Integralbegriff auf Basis des Riemann-Integrals definieren,
			\[ \int f=\int_0^\infty \mu([f>x])dx, \]
			wobei $f$ auf beliebigen Mengen definiert sein darf, also wenn $\mu$ Maß auf einem Messraum $(\Omega, \mathfrak{S})$, dann ist 
			\[ f:\Omega\to\R \]
			und $\mu([f>x])$ definiert sein soll, also $[f>x]\in\mathfrak{\sigma}$, wobei
			\[ [f>x]:=\{\omega\in\Omega: f(\omega)>x\}. \]
			
			\begin{defi}
				Seien $(\Omega_1, \mathfrak{S}_1)$, $(\Omega_2, \mathfrak{S}_2)$ zwei Messräume, dann heißt 
				\[ f:\Omega_1\to\Omega_2 \]
				messbar bezüglich $(\Omega_1, \mathfrak{S}_1)$ und $(\Omega_2, \mathfrak{S}_2)$ (oder kürzer $\mathfrak{S}_1-\mathfrak{S}_2$-messbar), wenn
				\[ f^{-1}(\mathfrak{S}_2)\subseteq\mathfrak{S}_1, \]
				also wenn $\forall A\in\mathfrak{S}_2: f^{-1}(A)\in\mathfrak{S}_1$. Für eine solche Funktion schreiben wir 
				\[ f:(\Omega_1,\mathfrak{S}_1)\to(\Omega_2, \mathfrak{S}_2). \]
				Eine Funktion
				\[ f:(\Omega,\mathfrak{S})\to(\R,\mathfrak{B}) \]
				heißt dann $\mathfrak{S}$-messbar bzw
				\[ f:(\R^{d_1},\mathfrak{S}_{d_1})\to(\R^{d_2},\mathfrak{B}_{d_2}) \]
				heißt Borelmessbar,
				\[ f::(\R^{d_1},\mathfrak{L}_{d_1})\to(\R^{d_2},\mathfrak{B}_{d_2}) \]
				heißt Lebesguemessbar. 
			\end{defi}
			
			\begin{satz}
				Sei $\mathfrak{C}$ ein Mengensystem über $\Omega_2$, das $\mathfrak{S}_2$ erzeugt, $\mathfrak{S}_2=\mathfrak{A}_\sigma(\mathfrak{C})$, dann ist $f:\Omega_1\to\Omega_2$ $\mathfrak{S}_1-\mathfrak{S}_2$-messbar genau dann, wenn
				\[ f^{-1}(\mathfrak{C})\subseteq \mathfrak{S}_1 \]
			\end{satz}
			
			\begin{bew}
				$\mathfrak{A}_\sigma(f^{-1}(\mathfrak{C}))=f^{-1}(\mathfrak{A}_\sigma(\mathfrak{C}))$ wurde schon zu Beginn der Vorlesung gezeigt. Wegen
				\[ f^{-1}(\mathfrak{A}_\sigma(\mathfrak{C}))=f^{-1}(\mathfrak{S}_2), \]
				sind wir an dieser Stelle schon fertig. 
			\end{bew}
			
			\begin{bem}
				Für $\mathfrak{S}_2=\mathfrak{B}$ ist $\mathfrak{C}$ z.B. die Menge der halboffenen Intervalle. Wir können aber auch $\mathfrak{C}=\{]-\infty,b],b\in\R\}$ oder $\mathfrak{C}=\{U\subseteq \R, U\text{ offen}\}$ hernehmen. Damit können wir schon einige Sätze beweisen.
			\end{bem}
			
			\begin{satz}
				Sei $f:\R\to\R$, bzw $f:\R^d\to\R$, dann ist $f$ Borelmessbar, wenn $f$
				 \begin{itemize}
				 	\item monoton oder
				 	\item stetig
				 \end{itemize}
				 ist. 
			\end{satz}
			
			\begin{bew}
				trivial, mit Erzeugendensystem $\mathfrak{C}=\{U\subseteq \R, U\text{ offen}\}$ und dem Faktum, dass eine Funktion genau dann stetig ist, wenn das Bild offener Mengen wieder offen ist. 
			\end{bew}
			
			\begin{satz}
				Ist 
				\[ f_1:(\Omega_1,\mathfrak{S}_1)\to(\Omega_2,\mathfrak{S}_2) \]
				und
				\[ f_2:(\Omega_2,\mathfrak{S}_2)\to(\Omega_3,\mathfrak{S}_3), \]
				dann ist auch
				\[ f_2\circ f_1:(\Omega_1,\mathfrak{S}_1)\to(\Omega_3,\mathfrak{S}_3) \]
				messbar. 
			\end{satz}
			
			\begin{bew}
				Sei $A\in\mathfrak{S}_3$, dann ist
				\[ (f_2\circ f_1)^{-1}(A)=f_1^{-1}(f_2^{-1}(A))\in\mathfrak{S}_1, \]
				da $f_2^{-1}(A)\in\mathfrak{S}_2$. \arge
			\end{bew}
			
			\begin{satz}
				Seien $(\Omega_1,\mathfrak{S}_1),(\Omega_2,\mathfrak{S}_2),(\Omega_3,\mathfrak{S}_3)$ Messräume. Wir bilden den Produktraum $(\Omega_2\times\Omega_3, \mathfrak{S}_2\times\mathfrak{S}_3)$. Dann ist
				\[ f:\Omega_1\to\Omega_2\times\Omega_3, f=(f_2,f_3) \]
				genau dann
				\[ f:(\Omega_1,\mathfrak{S}_1)\to (\Omega_2\times\Omega_3, \mathfrak{S}_2\times\mathfrak{S}_3), \]
				wenn
				\[ f_2:(\Omega_1,\mathfrak{S}_1)\to (\Omega_2,\mathfrak{S}_2) \]
				und
				\[ f_3:(\Omega_1,\mathfrak{S}_1)\to(\Omega_3,\mathfrak{S}_3). \]
			\end{satz}
			
			\begin{bew}
				\begin{itemize}
					\item "$\Rightarrow$":\newline
					Sei $A\in\mathfrak{S}_2$. Dann ist $A\times\Omega_3\in\mathfrak{S}_2\times\mathfrak{S}_3$. Damit ist $f^{-1}(A\times\Omega_3)\in\mathfrak{S}_1$, da $f^{-1}(A\times\Omega_3)=f_2^{-1}(A)$. 
					
					\item "$\Leftarrow$":\newline
					Seien $A\in\mathfrak{S}_2, B\in\mathfrak{S}_3$, dann ist
					\[ f_2^{-1}(A\times B)=f_2^{-1}(A)\cap f_3^{-1}(B)\in\mathfrak{S}_1. \]
					Da $C:=\{A\times B: A\in\mathfrak{S}_2, B\in\mathfrak{S}_3\}$ $\mathfrak{S}_2\times\mathfrak{S}_3$ erzeugt, folgt die Behauptung. 
				\end{itemize}
				\arge
			\end{bew}
			
			\begin{defi}
				Ist $(\Omega, \mathfrak{S}, \P)$ ein Wahrscheinlichkeitsraum, so nennt man
				\[ S:(\Omega,\mathfrak{S})\to(\R^d,\mathfrak{B}_d) \]
				eine $d$-dimensionale Zufallsvariable oder einen $d$-dimensionalen Zufallsvektor. Bei $d=1$ spricht man von der Zufallsvariable.
			\end{defi}
			
			\begin{satz}
				Ist $f:\R^{d_1}\to\R^{d_2}$ stetig, so ist $f$ Borel-messbar. 
			\end{satz}
			
			\begin{bew}
				folgt direkt daraus, dass wenn $f$ stetig ist, das Urbild jeder offenen Menge wieder offen ist. 
			\end{bew}
			
			\begin{satz}
				Ist $f:\R\to\R$ monoton, so ist $f$ Borel-messbar. 
			\end{satz}
			
			\begin{bew}
				Siehe Kusolitsch Beweis von Folgerung 7.10.
			\end{bew}
			
			\begin{satz}
				$f:=(f_1,...,f_d): (\Omega, \mf{S})\to(\R^d,\mathfrak{B}_d)$ genau dann, wenn 
				\[ \forall i=1,...,d: f_i:(\Omega,\mf{S})\to(\R,\mf{B}). \]
			\end{satz}
			
			\begin{bew}
				Siehe Kusolitsch Beweis Satz 7.11.
			\end{bew}
			
			\begin{satz}
				Aus $f_i: (\Omega,\mf{S})\to(\R,\mf{B})$, $i=1,2$ folgt
				\begin{enumerate}
					\item $f_1+f_2: (\Omega,\mf{S})\to \R)$,
					\item $f_1f_2: (\Omega,\mf{S})\to \R)$,
					\item $f_1\land f_2: (\Omega,\mf{S})\to \R)$,
					\item $f_1\lor f_2: (\Omega,\mf{S})\to \R)$.
				\end{enumerate}
			\end{satz}
			
			\begin{bew}
				Kuso Folgerung 7.14.
			\end{bew}
			
			\begin{defi}[7.14]
				
			\end{defi}
			
		\section[Erweiterte $\R$-Funktionen]{Erweitert reellwertige Funktionen}
			Whaaaaat??\newline
			Kuso abschreiben... S.86\newline
			\begin{satz}
				Sei $f_n$ eine Folge messbarer Funktionen. Dann ist 
				\[ M:=[\liminf f_n=\limsup f_n]\in\mf{S} \]
			\end{satz}
			
			\begin{bew}
				Siehe Kusolitsch Beweis Satz 7.20. 
			\end{bew}
			
			\begin{satz}[7.24]
				
			\end{satz}
			
		\section{Treppenfunktionen}
			\begin{defi}
				Eine Funktion 
				\[ t:\Omega\to\R \]
				heißt Treppenfunktion, wenn es eine endliche Zerlegung $A_1,...,A_n$ von $\Omega$ und reelle Zahlen $\alpha_1,...,\alpha_n$ gibt mit
				\[ \forall \omega\in\Omega: t(\omega)=\sum_{i=1}^n\alpha_i\mathds{1}_{A_i}(\omega). \]
			\end{defi}
			
			\begin{lemma}
				Eine Funktion $t:\Omega\to\R$ ist genau dann eine Treppenfunktion, wenn es Mengen $B_1,...,B_m$ und reelle Zahlen gibt, sodass $t=\sum_{j=1}^m\beta_j\mathds{1}_{B_j}$. 
			\end{lemma}
			
			\begin{bew}
				Siehe Beweis Lemma 7.26, Kusolitsch.
			\end{bew}
			
			\begin{bem}
				Sind die oben genannten Mengen $A_i$ und $B_i$ alle messbar, so ist auch $t$ messbar. Die Umkehrung gilt jedoch im Allgemeinen nicht, man kann also auch eine messbare Treppenfunktoin mit Hilfe einer nichtmessbaren Zerlegung darstellen kann, z.B. $t\equiv 0=0\mathds{1}_A+0\mathds{1}_{A^c}$ mit $A\notin \mf{S}$. 
			\end{bem}
			
			\begin{defi}
				$t=\sum_{i=1}^{n}x_i\mathds{1}_{[t=x_i]}$ ist die kanonische Darstellung einer messbaren Treppenfunktion. 
			\end{defi}
			
			\begin{satz}
				Zu jeder messbaren positiven Funktion $f$ gibt es eine monoton steigende Folge $(t_n)$ aus positiven Treppenfunktionen, sodass 
				\[ \forall \omega\in\Omega: f(\omega)=\lim_{n\to\infty} t_n(\omega). \]
				Weiters gibt es zu jeder messbaren Funktion $f$ eine Folge $(t_n)$ aus Treppenfunktionen, sodass 
				\[ \forall \omega\in\Omega: f(\omega)=\lim_{n\to\infty} t_n(\omega) \]
				und
				\[ \forall n\in\N: |t_n|\le |f|. \]
				Ist $f$ beschränkt, so konvergiert $(t_n)$ gleichmäßig gegen $f$. 
			\end{satz}
			
			\begin{bew}
				Siehe Beweis Satz 7.30, Kusolitsch. 
			\end{bew}
			
		\section{Konvergenzarten}
			\begin{defi}
				Zwei Funktionen $f,g$ sind fast überall gleich, falls sie auf dem Komplement einer Nullmenge gleich sind. 
			\end{defi}
			
			\begin{defi}
				Eine Folge $(f_n)$ messbarer Funktionen konvergiert gleichmäßig $\mu-$fast überall (bzw $P-fs$) gegen eine Funktion $f$, wenn es eine $\mu$-Nullmenge $N$ gibt, sodass $(f_n)$ auf $N^c$ gleichmäßig konvergiert. 
			\end{defi}
			
			\begin{defi}
				Eine messbare Funktion $f$ auf einem Maßraum heißt $\mu$-fast überall beschränkt, wenn es ein $c\in\R$ gibt mit $\mu(|f|>c)=0$.
				\[ \lVert f\rVert_\infty:=\esssup f:=\inf\{c\in\R: \mu(|f|>c)=0\} \]
				wird als das essentielle Supremum von $f$ bezeichnet. 
			\end{defi}
			
			\begin{satz}
				Sei \[ \mathfrak{F}:=\{f:(\R^{(n)},\mathfrak{B}_{(n)})\to(\R^{(m)},\mf{B}_{(m)})\}. \]
				Nun ist $\mf{F}$ die kleinste Menge der reellen Funktionen, die die stetigen Funktionen enthält und bezüglich der Bildung von punktweisen Grenzwerten abgeschlossen ist
			\end{satz}
			
			\begin{bew}
				Sei $\mf{G}$ diese kleinste Menge. Diese existiert, wie man leicht über Durchschnittbildung über alle diese Systeme zeigen kann, nachdem man auch gezeigt hat, dass die gewünschten Eigenschaften durch Durchschnittbildung erhalten werden.\newline
				$\mf{G}\subseteq \mf{F}$ ist offensichtlich. Bleibt also noch "$\supseteq$" zu zeigen:\newline
				Da jede messbare Funktion als Limes von Treppenfunktion dargestellt werden kann, ist zu zeigen, dass $\mf{G}$ alle Treppenfunktionen enthält, wofür es genügt zu zeigen, dass $\mf{G}$ alle Indikatorfunktionen enthält und abgeschlossen bzgl. Addition und Multiplikation ist:
				\begin{itemize}
					\item Abgeschlossenheit:\newline
					\zz:
					\[ f,g\in\mf{G}\Rightarrow f+g,fg\in\mf{G}. \]
					Wir zeigen dies über das Prinzip der guten Mengen: Sei für $f\in\mf{G}$
					\[ \mf{G}(f):=\{g\in\mf{G}: f+g,fg\in\mf{G}\}. \]
					Nun ist $\mf{G}$ bzgl. Limesbildung abgeschlossen:
					\[ g_n\in\mf{G}(f), g_n\to g\Rightarrow g\in\mf{G} \]
					\[ \Rightarrow f+g_n\in\mf{G}(f), f+g_n\to g \]
					und
					\[ \Rightarrow fg_n\in\mf{G}(f), fg_n\to g, \]
					und damit $g\in \mf{G}(f)$. Sind nun $f,g$ stetig, so ist auch $fg$ und $f+g$ stetig und daher in $\mf{G}$, womit
					\[ \mf{G}\subseteq \mf{G}(f) \]
					folgt, also für stetiges $f$ und $g\in\mf{G}$ folgt $f+g, fg\in\mf{G}$ und analog für vertauschte Rollen von $f$ und $g$. Dann ist $g\in\mf{G}(f)$, womit $\mf{G}$ abgeschlossen bzgl. Addition und Multiplikation ist. 
					\item Enthält Indikatoren von Borelmengen
					Sei 
					\[ \mf{C}=\{A\subseteq B: A()\in\mf{G}\}. \]
					Wir müssen nun zeigen, dass $\mf{C}=\mf{B}$. Wir zeigen also zunächst, dass $\mf{C}$ Sigmaalgebra ist. 
					\begin{itemize}
						\item Algebra:\newline
						Sei $A\in\mf{C}$, dann ist $A()\in\mf{G}$. Dann ist
						\[ A^c()=1-A()\in\mf{G}\Rightarrow A^c\in\mf{C}. \]
						Seien $A,B\in\mf{C}$. Dann ist 
						\[ A\cap B()=A()B()\in\mf{G}. \]
						\item Sigmaalgebra:\newline
						Es genügt zu zeigen: Wenn $A_n\uparrow A$, $A_n\in\mf{C}$, so folgt $A\in\mf{C}$. Nun sind also $A_n()\in\mf{G}$, damit also $A_n()\uparrow A()$ (punktweise) und damit $A\in\mf{C}$ aus der Abgeschlossenheit bzgl punktweisen Limiten. 
					\end{itemize}
					Nun zeigen wir $a\le b: ]a,b]\in\mf{C}$. Wir definieren:
					\[ f_n(\omega)=\left\{\begin{array}{ll}
					0&\text{für } \omega \le a\\
					(\omega-a)n&\text{für }a<\omega\le a+\frac{1}{n}\\
					1&\text{für }a+\frac{1}{n}<\omega\le b\\
					1-(\omega-b)n&\text{für }b<\omega\le b+\frac{1}{n}\\
					0&\text{für } \omega >b+\frac{1}{n}
					\end{array}\right. \]
					Diese sind alle stetig und konvergieren gegen $]a,b]()$, damit also $]a,b]\in\mf{C}$. 
				\end{itemize}
			\end{bew}
			
			\begin{defi}
				Sei $P$ eine Aussage und $(\Omega,\mf{S}, \mu)$ ein Maßraum. Wir sagen, $P$ gilt fast überall oder fast sicher, wenn es eine Menge $N\in\mf{S}$, $\mu(N)=0$ gibt mit $P(\omega)$ für alle $\omega \in N^c$.
			\end{defi}
			
			\begin{defi}
				Seien $(\Omega_1,\mf{S}_1,\mu)$ ein Maßraum und $(\Omega_2,\mf{S}_2)$ ein Messraum. Sei $f:\Omega_1\to\Omega_2$. $f$ heißt fast überall messbar, wenn
				\[ \exists \Omega_1'\in\mf{S}_1: \mu(\Omega_1'^c)=0, \]
				wobei $f$ auch nur auf $\Omega_1'$ definiert sein kann. Dann ist
				\[ f:(\Omega_1',\mf{S}\cap\Omega_1')\to(\Omega_2,\mf{S}_2) \]
			\end{defi}
			
			\begin{defi}
				$f_n\to f$ heißt $\mu$-fast überall
				\[ f_n: (\Omega,\mf{S},\mu)\to(\R,\mf{B}), \]
				wenn es $N\in\mf{S}:\mu(N)=0$ mit 
				\[ f_n(\omega)\to f(\omega) \]
				für fast alle $\omega\in N^c$. 
			\end{defi}
			
			\begin{defi}[gleichmäßige Konvergenz]
				$f_n\to f$ ist gleichmäßig konvergent, wenn
				\[ \forall \varepsilon>0\exists n_0(\varepsilon)\forall \omega\in\Omega\forall n\ge n_0(\varepsilon): |f_n(\omega)-f(\omega)|<\varepsilon \]
			\end{defi}
			
			\begin{defi}[fast überall gleichmäßige Konvergenz]
				$f_n\to f$ ist fast überall gleichmäßig konvergent für
				\[ f_n,f:(\Omega,\mf{S},\mu)\to (\R,\mf{B}) \]
				wenn es eine Menge $M\in\mf{S},\mu(M)=0$ gibt mit $f_n\to f$ gleichmäßig auf $M^c$. 
			\end{defi}
			
			\begin{bem}
				$f_n\to f$ gleichmäßig, wenn, wie aus der Analysis bekannt,
				\[ \lVert f_n-f\lVert_{\sup}=\sup\{|f_n(\omega)-f(\omega)|:\omega\in\Omega\}\to0. \]
			\end{bem}
			
			\begin{defi}
				Sei $f(\Omega,\mf{S})\to(\R,\mf{B})$, $(\Omega,\mf{S},\mu)$ Maßraum. Dann ist das Essentielle Supremum von $f$
				\[ \esssup f:=\inf\{y\in\R:\mu([f>y])=0\}. \]
			\end{defi}
			
			\begin{bem}
				Es gilt
				\[ \mu([f>\esssup f])=0, \]
				da
				\[ \mu([f>\esssup f])=\mu\left(\bigcup_{n\in\N} [f>\esssup f+\frac{1}{n}]\right)\le \sum_{n\in\N} \mu([f>\esssup f+\frac{1}{n}])=\sum_{n\in\N}0=0. \]
			\end{bem}
			
			\begin{satz}
				Sei $c>0$. Dann ist
				\[ \esssup cf=c\esssup f. \]
				Weiters ist für $f,g\ge 0$
				\[ \esssup f+g\le \esssup f+\esssup g \]
			\end{satz}
			\begin{bew}
				Folgt direkt aus der Definition:
				\begin{align*} \esssup cf&=\inf\{y\in\R: \mu([cf>y])=0\}\\&=\inf\{c\cdot z\in\R: \mu([cf>cz])=0\}\\&=\inf c\odot \{z\in\R: \mu([f>z])=0\}\\&=c\esssup f \end{align*}
				Weiters
				\begin{align*} \mu([f+g> \esssup f+\esssup g])&\le \mu([f>\esssup f]\cup[g>\esssup g])\\&\le\mu([f>\esssup f])+\mu([g>\esssup g])\\&=0+0=0 \end{align*}
			\end{bew}
			
			\begin{defi}
				\[ \lVert f\rVert_\infty :=\esssup |f|. \]
				Dies ist fast eine Norm, die erste Eigenschaft fehlt, da
				\[ \lVert f\rVert_\infty=0\Leftrightarrow \mu([|f|>0])=0 \]
			\end{defi}
			
			\begin{bew}
				\[ \norm{cf}_\infty=|c|\norm{f}_\infty \]
				folgt direkt aus einem oben bewiesenen Satz.
				\[ \snorm{f+g}=\esssup|f+g|\le\esssup(|f|+|g|)\le \esssup|f|+\esssup|g|=\snorm{f}+\snorm{g}. \]
				Es fehlt 
				\[ \snorm{f}=0\Leftrightarrow f=0, \]
				denn es gilt
				\[ \snorm{f}=0\Leftrightarrow f=0 \text{ fast überall.} \]
			\end{bew}
			
			\begin{defi}
				Sei
				\[ \Li(\Omega,\mf{S},\mu):=\left\{f:(\Omega,\mf{S})\to(\overline{\R},\overline{\mf{B}}): f\text{ ist fast überall messbar, }\snorm{f}<\infty\right\}, \]
				dann ist 
				\[ f\sim g\Leftrightarrow\snorm{f-g}=0\Leftrightarrow f=g\text{ fast überall} \]
				eine Äquivalenzrelation (trivial).  Damit ist 
				\[\Li(\Omega,\mf{S},\mu)=\Li\setminus\sim \]
				und $\snorm{.}$ eine Norm auf $\Li$ und somit auch $\Li$ ein normierter Vektorraum, bzw. sogar ein Banachraum. 
			\end{defi}
			
			\begin{bew}[$\Li$ ist vollständig]
				Sei $f_n$ Cauchyfolge bezüglich $\snorm{.}$, also $\lim\limits_{m,n\to\infty}\snorm{f_n-f_m}=0$. Damit konvergiert $f_n$ gleichmäßig und somit auch in $\Li$. 
			\end{bew}
			
			\begin{defi}
				Sei $(f_n)$ eine Folge von Funktionen,
				\[ f_n:(\Omega,\mf{S})\to(\R,\mf{B}), \]
				die gegen ein $f$ fast gleichmäßig konvergiert, wenn 
				\[ \forall \varepsilon>0\exists A\in\mf{S}: \mu(A^c)<\varepsilon \text{ mit } f_n\to f\text{ gleichmäßig auf }A. \]
			\end{defi}
			
			\begin{bsp}
				Sei $([0,1],\mf{B})\cap[0,1],\lambda\big|_{\mf{B}\cap[0,1]})$ und
				\[ f_n(\omega)=\omega^n, \]
				dann ist
				\[ \lim_{n\to\infty}f_n(\omega)=\left\{\begin{array}{ll}
				0:&0\le\omega<1\\1:&\omega=1
				\end{array}\right., \]
				wir können also ein beliebig kleines Intervall $A$ um 1 herausnehmen, sodass $f_n$ gleichmäßig auf $A^c$ konvergiert, also konvergiert $f_n$ fast gleichmäßig. 
			\end{bsp}
			
			\begin{satz}[Satz von Egorov]
				Sei $(\Omega,\mf{S},\mu)$ endlich
				\[ f_n,f:(\Omega,\mf{S})\to(\R,\mf{B}) \]
				dann ist
				\[ f_n\to f\,\,\mu-\text{fast überall}\Leftrightarrow f_n\to f\,\,\mu-\text{fast gleichmäßig.} \]
			\end{satz}
			
			\begin{bew}$ $
				\begin{itemize}
					\item "$\Leftarrow$" gilt natürlich immer.
					\item "$\Rightarrow$":\newline
					Sei $f_n(\omega)\to f(\omega)$, also
					\[ \forall \varepsilon>0\exists N=N(\varepsilon)\forall n\ge N(t) |f_n(\omega)-f(\omega)<\varepsilon. \]
					Es gibt $B\in\mf{S}, \mu(B^c)=0, f_n\to f$ punktweise auf $B$. Dann gilt für
					\[ A_{N,\varepsilon}=\{\omega:|f_n(\omega)-f(\omega)|<\varepsilon \text{ für }n\ge N, \]
					wenn wir $\varepsilon$ fix wählen:
					\[ \bigcup_{N\in\N}A_{N,\varepsilon}\supset B \]
					\[ \Rightarrow \lim_{N\to\infty}=\mu(B)=\mu(\Omega) \]
					Und für $\varepsilon=\frac{1}{2^K}$ wählen wir $N_K$, sodass
					\[ \mu(A_{N_\varepsilon,\frac{1}{2^K}})\ge\mu(\Omega)-\frac{1}{2^K}. \]
					Mit 
					\[ C_K:=\bigcap_{k\ge K} A_{N_k,\frac{1}{2^k}} \]
					gilt dann
					\[ \mu(C_K^c)\le\sum_{k\ge K}\mu(A_{N_k,\frac{1}{2^K}})\le \frac{2}{2^K} \]
					Dann gilt klarerweise
					\[ f_n\to f\text{ auf }C_K\text{ gleichmäßig,} \]
					womit die Behauptung bewiesen ist. 
				\end{itemize}
			\end{bew}
			
			\begin{defi}
				Sei $(\Omega,\mf{S},\mu)$
				\[ f_n,f: (\Omega,\mf{S})\to(\R,\mf{B}). \]
				Dann ist $f_n\to f$ im Maß (in Wahrscheinlichkeit, wenn $(\Omega, \mf{S},\mu)$ ein Wahrscheinlichkeitsraum ist), wenn
				\[ \forall \varepsilon>0: \lim_{n\to\infty}\mu([|f_n-f|]\ge\varepsilon])=0. \]
			\end{defi}
			
			\begin{bem}
				Diese Konvergenz ist später wichtig in der Statistik. Dies ist auf endlichen Maßräumen die schwächste Konvergenzart. 
			\end{bem}
			
			\begin{satz}
				Gilt $f_n\to f$ im Maß und $f_n\to g$ im Maß, so folgt
				\[ f=g \text{ fast überall}. \]
			\end{satz}
			
			\begin{bew}
				Es gilt nun für ein $\varepsilon>0$
				\[ \lim_{n\to\infty} \mu([|f_n-f|\ge \frac{\varepsilon}{2}])=0 \]
				\[ \lim_{n\to\infty} \mu([|f_n-g|\ge \frac{\varepsilon}{2}])=0 \]
				Also existiert für ein $\delta>0$ ein $n\in\N$, sodass
				\[ \mu([|f_n-f|\ge\frac{\varepsilon}{2}])<\delta. \]
				Betrachte nun
				\[ \mu([|f-g|>\varepsilon])=\mu([f-f_n+f_n-g>\varepsilon])\le\mu([|f-f_n|+|f_n-g|>\varepsilon]) \]
				\[ \le \mu([|f_n-f|>\frac{\varepsilon}{2})+\mu([|f_n-g|>\frac{\varepsilon}{2}])\le 2\delta \]
				und da $\delta$ beliebig ist
				\[ \mu([|f-g|>\varepsilon])=0\Rightarrow\mu([|f-g|>0])=0, \]
				womit $f=g$ fast überall. 
			\end{bew}
			
			\begin{bem}
				\[
				\begin{array}{c c}
					&\text{fast überall gleichmäßig}\\
					&\Downarrow\\
					&\text{fast gleichmäßig}\\
					\hfill\schircherarrow&\Downarrow\\
					\text{fast überall}&\text{im Maß}
				\end{array}\]
			\end{bem}
			
			\begin{defi}
				Sei $(f_n)\to f$ eine Folge von fast überall messbaren, fast überall endlichen, rellwertigen Funktionen. Sei $f$ ebenfall fast überall messbar, fast überall endlich. Wenn 
				\[ \forall\varepsilon>0: \mu([|f_n-f|>\varepsilon)\to 0\Rightarrow f_n\to f\text{ im Maß} \]
				gilt, dann nennen wir $f_n$ Barock. \colorbox{yellow}{??? - nicht sicher ob das stimmt}
			\end{defi}
			
			\begin{lemma}
				Sei $\mathcal{L}_0=\{f:(\Omega,\mf{S})\to(\R,\mf{B})\}$. Dann ist
				\[ d(f,g):=\inf\{\varepsilon>0: \mu([|f-g|>\varepsilon])<\varepsilon\} \]
				eine Pseudometrik auf $\mathcal{L}_0$. $d$ heißt Lévy-Metrik. 
			\end{lemma}
			
			\begin{bew}
				$d(f,f)=0$, $d(f,g)=d(g,f)$ ist klar. Zu zeigen bleibt also noch die Dreiecksungleichung:
				\[ \forall f,g,h\in\mathcal{L}_0: d(f,g)\le d(f,g)+d(g,h). \]
				Für $d(f,g)=\infty$ oder $d(g,h)=\infty$ ist die Behauptung natürlich erfüllt. Seien also die entsprechenden Pseudometriken kleiner $\infty$. Dann gilt für ein $\delta >0$:
				\[ \mu([|f-g|>d(f,g)+\frac{\delta}{2}])\le d(f,g)+\frac{\delta}{2} \]
				\[ \mu([|g-h|>d(g,h)+\frac{\delta}{2}])\le d(g,h)+\frac{\delta}{2} \]
				und damit
				\[ \mu([|f-h|>d(f,g)+d(g,h)+\delta])
				\le\mu([|f-g|+|g-h|>d(f,g)+d(g,h)+\delta]) \]
				\[ \le\mu([|f-g|>d(f,g)+\frac{\delta}{2}])+\mu([|g-h|>d(g,h)+\frac{\delta}{2}])\le d(f,g)+d(g,h)+\delta \]
				\[ \Rightarrow d(f,g)\le d(f,g)+d(g,h)+\xcancel{{\delta}}, \]
				da $\delta>0$ beliebig ist.
			\end{bew}
			\begin{bem}
				Es gilt
				\[ d(f,g)=0\Leftrightarrow f=g \;\;\mu-\text{fast überall} \]
				Mithilfe der Äquivalenzrelation
				\[ f\sim g:\Leftrightarrow f=g\;\;\mu\text{-fast überall}, \]
				zerfällt $\mathcal{L}_0$ in Äquivalenzklassen. Auf \[L_0=\mathcal{L}_0\setminus\sim=\{[f]_\sim: f\in\mathcal{L}_0\}, [f]_\sim=\{g\in\mathcal{L}_0: g\sim f\}\]
				ist damit $d$ eine Metrik.
			\end{bem}
			
			\begin{defi}
				$(f_n), f_n\in\mathcal{L}_0$ ist eine Cauchyfolge im Maß, wenn 
				\[ \forall\varepsilon>0: \lim_{m,n\to\infty}\mu([|f_n-f_m|>\varepsilon])=0, \]
				das heißt
				\[ \forall \delta>0\exists n_0\forall m,n\ge n_0: \mu([|f_n-f_m|>\varepsilon]<\delta \]
			\end{defi}
			
			Im Folgenden arbeiten wir auf den folgenden Satz hin:
			\begin{satz}
				$(L_0,d)$ ist vollständig. 
			\end{satz}
			
			\begin{satz}
				Sei $(f_n)$. Es gilt
				\[ f_n\to f\text{ im Maß}\Leftrightarrow d(f_n,f)\to 0. \]
			\end{satz}
			
			\begin{bew}
				Gelte $f_n\to f$ im Maß., also
				\[ \lim_{n\to\infty} \mu([|f_n-f|>\varepsilon])\to 0 \]
				\[ \Rightarrow \exists n_0: n\ge n_0: \mu([|f_n-f|>\varepsilon])<\varepsilon \]
				\[ \forall n\ge n_0\Rightarrow d(f_n,f)\le \varepsilon \]
				\[ \Rightarrow d(f_n,f)\to 0. \]
				Gelte $d(f_n,f)\to 0$. Wähle dann $\varepsilon>0, \varepsilon>\delta>0$.  Dann 
				\[ \exists n_0\forall n\ge n_0: d(f_n,f)<\delta  \]
				\[ \Rightarrow \mu([|f_n-f|>\delta])<\delta \]
				\[ \Rightarrow \mu([|f_n-f|>\varepsilon])<\delta, \]
				also konvergiert $f_n\to f$ im Maß. 
			\end{bew}
			
			\begin{satz}
				Sei $(f_n)$ Cauchyfolge im Maß. Dann existiert eine Teilfolge $(f_{n_k})$, die fast gleichmäßig konvergiert. 
			\end{satz}
			
			\begin{bew}
				Sei $(f_n)$ eine Cauchyfolge im Maß, also bzgl der Lévy-Metrik. Dann gilt
				\[ \forall \varepsilon>0\exists n_0(\varepsilon): \forall n,m\ge n_0(\varepsilon): \mu([|f_n-f_m|>\varepsilon])<\varepsilon. \]
				Wir setzen $n_k=n_0(2^{-k})$, so dass $n_{k+1}>n_k$. Nun gilt
				\[ \forall n,m\ge n_k: \mu([|f_n-f_m|>2^{-k}])\le 2^{-k} \]
				speziell für $n=n_k, m=n_{k+1}$:
				\[ \mu([|f_n-f_m|>2^{-k}])<2^{-k} \]
				Behauptung: $(f_{n_k})$ konvergiert fast gleichmäßig. Nämlich: Sei $\varepsilon>0$. Dann
				\[ \exists k_0: 2^{-k_0}<\varepsilon. \]
				Mit 
				\[ A_\varepsilon:=\bigcup_{k>k_0}[|f_{n_k}-f_{n_{k+1}}|>2^{-k}] \]
				folgt
				\[ \mu(A_\varepsilon^c)\le \sum_{k>k_0} 2^{-k}=2^{-k_0}<\varepsilon. \]
				Mit $r>l>k_0$ folgt dann für $\omega\in A_\varepsilon$
				\[ |f_{n_r}-f_{n_l}|\le |f_{n_r}(\omega)-f_{n_{r-1}}(\omega)+...+|f_{n_{l+1}}(\omega)-f_{n_l}(\omega)|\le 2^{-(r-1)}+...+2^{-l}\le 2\cdot 2^{-l}, \]
				also ist $f_{n_k}$ auf $A_\varepsilon^C$ eine gleichmäßige Cauchyfolge . Damit
				\[ \forall \omega\in A_\varepsilon^c:f(\omega)=\lim_{n\to\infty} f_n(\omega), \]
				also konvergiert  $(f_{n_k})$ fast gleichmäßig, da
				\[ d(f,f_n)\le d(f_n,f_{n_k})+d(f_{n_k},f) \]
			\end{bew}
			
			Erinnerung:
				Sei 
				\[ f: \Omega_1\to \Omega_2, \]
				$\mf{S}_2$ Sigmaalgebra über $\Omega_2$, $f^{-1}(\mf{S}_2)$ ist Sigmaalgebra über $\Omega_1$ (und zwar die kleinste Sigmaalgebra, bezüglich der $f$ messbar ist).
			
			\begin{defi}
				Sei $(f_i)_{i\in I}$ eine Familie von Funktionen $\Omega\to\Omega_i$, $\mf{S}_i$ Sigmaalgebra über $\Omega_i$. Die von $(f_i)_{i\in I}$ erzeugte Sigmaalgebra $\mf{S}_\sigma((f_i)_{i\in I})$ ist die kleinste Sigmaalgebra, bezüglich der alle $f_i$ messbar sind.
				\[ \mf{S}_\sigma((f_i)_{i\in I})=\mf{S}_\sigma(\{f_i^{-1}(B), i\in I, B\in\mf{S}_i\} \]
			\end{defi}
			
			\begin{bem}
				Anschaulich: $\{f: \mf{S}_\sigma((f_i)_{i\in I})\to (\R,\mf{B})\}$ sind alle Funktionen, die wir aus den Funktionen $(f_i)_{i\in I}$ berechnen können. (kleine Lüge, eigentlich ist es alles, was wir "`vernünftig"' aus den Funktionen berechnen können)
			\end{bem}
			
			\begin{satz}
				Sei $f:\Omega_1\to\Omega_2$, $(\Omega_2,\mf{S}_2)$ ein Messraum. Dann ist
				\[ g:\Omega_1\to \R \]
				genau dann bezüglich $f^{-1}(\mf{S}_2)$ messbar (also $g:(\Omega_1,f^{-1}(\mf{S}_2))\to(\R,\mf{B})$) wenn es ein $h:(\Omega_2,\mf{S}_2)\to(\R, \mf{B})$ mit $g=h\circ f$ gibt. 
			\end{satz}
			
			\begin{bew}
				$\Leftarrow$ ist klar.\newline
				Wir zeigen also $\Rightarrow$: (Kochrezept für Beweise von Sätzen über messbare Funktionen: Indikatoren $\rightarrow$ Treppenfunktionen $\rightarrow$ messbare Funktion)\newline
				Wir unterscheiden also mehrere Fälle:
				\begin{enumerate}[(1)]
					\item $g$ ist Indikatorfunktion, also $g=A(), A\in f^{-1}(\mf{S}_2)$
					\[ \Rightarrow \exists B: A=f^{-1}(B) \]
					also gilt mit $h:=B$
					\[ g=h\circ f \]
					und wegen
					\[ g(\omega)=1\Leftrightarrow \omega\in A\Leftrightarrow f(\omega)\in B\Leftrightarrow h(f(\omega))=1 \]
					sind wir fertig.
					\item $g$ ist Treppenfunktion, also $g=\sum_{i=1}^n a_i A_i$, aus dem vorigen Punkt erhalten wir $A_i()=h_i\circ f$, damit
					\[ g=\sum_{i=1}^{n}a_i\cdot h_i\circ f=\left(\sum_{i=1}^{n}a_i\cdot h_i\right)\circ f \]
					die Behauptung.
					\item $g$ ist messbar, also existiert eine Folge $g_n$ von Treppenfunktionen, sodass
					\[ g_n\to g, \]
					also 
					\[ g_n=h_n\circ f, h_n:(\Omega_2, \mf{S}_2)\to(\R,\mf{B}). \]
					An dieser Stelle wäre es schön, wenn wir $h=\lim h_n$ setzen könnten, das geht aber nicht immer, aber $h=\limsup h_n$ geht. Damit ist die Behauptung bewiesen. 
				\end{enumerate}
			\end{bew}
			
		\section{Messbare Funktionen und Maße}
			\begin{defi}
				Seien $(\Omega_1,\mf{S}_1,\mu_1),(\Omega_2,\mf{S}_2,\mu_2)$ Maßräume. Dann heißt $f:\Omega_1\to \Omega_2$ maßtreu, wenn
				\[ f: (\Omega_1,\mf{S}_1)\to (\Omega_2, \mf{S}_2) \]
				und
				\[ \forall B\in\mf{S}_2: \mu_2(B)=\mu_1(f^{-1}(B)). \]
			\end{defi}
			
			\begin{satz}
				Sei $(\Omega_1,\mf{S}_1,\mu_1)$ ein Maßraum und $(\Omega_2,\mf{S}_2)$ ein Messraum. Für eine Funktion
				\[ f:(\Omega_1,\mf{S}_1)\to (\Omega_2,\mf{S}_2) \]
				kann man ein eindeutig bestimmtes Maß
				\[ \mu_2(B)=\mu_1(f^{-1}(B)) \]
				definieren, sodass $f$ eine maßtreue Abbildung wird. $\mu_2$ heißt das von $f$ induzierte Maß. 
			\end{satz}
			
			\begin{bew}
				Der Beweis, dass $\mu_2$ ein Maß ist, wird (fast vollständig) dem Leser überlassen. Seien nun $B_n$ disjunkt. Dann ist
				\[ \mu(\bigcup_n B_n)=\mu_1(f^{-1}(\bigcup_n B_n))=\mu_1(\bigcup_n f^{-1}(B_n)=\sum_n \mu_1(f^{-1}(B_n))=\sum_n\mu_2(B_n). \]
			\end{bew}
			
			\begin{bem}
				Ist $(\Omega,\mf{S}, \P)$ ein Wahrscheinlichkeitsraum so heißt $X:(\Omega,\mf{S})\to(\Omega_2,\mf{S}_2)$,
				\[ \P\circ X^{-1}=\P_X \]
				die Verteilung von $X$. 
			\end{bem}
			
		\section[Zufallsvariable/Verteilungen]{Zufallsvariable und ihre Vertilungen}
			\begin{defi}
				Sei $(\Omega,\mf{S},\P)$ ein Wahrscheinlichkeitsraum und 
				\[ X_i: (\Omega,\mf{S})\to (\Omega_i, \mf{S}_i), i\in I. \]
				Wir nennen $(X_i)_{i\in I}$ unabhängig, wenn 
				\[ \forall n\in N\forall \{i_1,...,i_n\}\subseteq I: \P\left(\bigcap_{k=1}^n X_{i_k}^{-1}(A_{i_k})\right)=\prod_{k=1}^n\P(X_{i_k}^{-1}(A_{i_k})), A_{i_k}\in\mf{S}_{i_k}. \]
			\end{defi}
			
			\begin{bem}
				Sei $(\Omega, \mf{S}, \P)$ ein Wahrscheinlichkeitsraum. $(\mf{S}_i)_{i\in I}$ Teilsigmaalgebra von $\mf{S}$ (d.h. $\mf{S}_i\subseteq\mf{S}$), dann heißen $(\mf{S}_i)_{i\in I}$ unabhängig, wenn 
				\[ \forall n\in \N,\forall \{i_1,...,i_n\}\subseteq I:\P\left(\bigcap_{k=1}^n A_{i_k}\right)=\prod_{k=1}^{n}\P(A_{i_k}) \]
			\end{bem}
			
			\begin{bem}
				Ab jetzt: $\Omega_i=\R$, $\mf{S}_i=\mf{B}$. 
			\end{bem}
			
			\subsection{Diskrete Verteilungen}
			
				\begin{defi}[Alternativ- oder Bernoulliverteilung]
					Sei $X$ eine Indikatorfunktion $X=A(), A\in\mf{S}$. 
					\[ \P_X(\{1\})=\P(A), \P(A)=p,\P_x(\{0\})=1-\P(A) \]
					mit $0\le p\le 1$ heißt Alternativ- oder Bernoulliverteilung. $p=\frac{1}{2}$ werden wir dann als Münzwurf bezeichnen. 
				\end{defi}
				
				\begin{defi}[Diskrete Gleichverteilung: (Laplacescher Warhscheinlichkeitsraum)]
					
					\[ \P(X=x)=\frac{1}{b-a+1}, x=a,a+1,...,b; a,b\in\Z \]
				\end{defi}
				
				\begin{defi}[Binomialverteilung]
					Seien $A_1,...,A_n$ unabhängig $\P(A)=p$. 
					\[ X=\text{Anzahl der Ereignisse, die eintreten}\in\{0,...,n\} \]
				\end{defi}
				
				\begin{bsp}
					Binomialverteilung: 
					\[ \P(X=n)=\P(A_1\cap ...\cap A_n)=\P(A_1)\cdots\P(A_n)=p^n \]
					\[ \P(X=0)=\P(A_1^c\cap...\cap A_n^c)=(1-p)^n \]
					Alles weitere ist etwas komplizierter:
					\[ \P(X=1)=\P((A_1\cap A_2^c\cap...\cap A_n^c)\cup(A_1^c\cap A_2\cap ...)\cup...\cup(A_1^c\cap...\cap A_{n-1}^c\cap A_n))= \]
					\[ n\cdot p^1\cdot (1-p)^{n-1}. \]
					Der allgemeine Fall sieht dann wie folgt aus:
					\[ \P(X=k)=\binom{n}{k}p^k (1-p)^{n-k}, k=0,...,n \]
					mit der Konvention $\binom{n}{k}=0$, wenn $k<0$ oder $k>n$. Diese Verteilung heißt Binomialverteilung $B(n,p)$. 
				\end{bsp}
				
				\begin{defi}[Poisson-Verteilung]
					$\P(X=n)=\frac{\lambda^ne^{-\lambda}}{n!}, n=0,1,...; \lambda>0$. Dies ist der Grenzfall der Binomialverteilungen für $np\to \lambda$ für $n\to\infty$ (vgl Übung.)
				\end{defi}
				
				\begin{defi}[Geometrische Verteilung]
					\[G(p): \P(X=n)=p(1-p)^n, n\ge 0\] 
					\[\tilde{G}(p), \P(X=n)=p(1-p)^{n-1}, n\ge 1\]
				\end{defi}
				
				\begin{defi}[negative Binomialverteilung]
					\[ NB(\alpha,p): \P(X=n)=(1-p)^np^\alpha\binom{n+\alpha-1}{n}, n\ge 0 \]
					
				\end{defi}
				
				\begin{defi}[Hypergeometrische Verteilung]
					Bsp: Sei eine Urne mit $N$ Kugeln, $A$ schwarze, $N-A$ weiße. Es werden $n$ Kugeln ohne zurücklegen gezogen. Sei 
					\[ X=\text{Anzahl der schwarzen Kugeln unter den gezogenen} \]
					Dann ist
					\[ H(N,A,n):\P(X=x)=\frac{\binom{A}{x}\binom{N-A}{n-x}}{\binom{N}{n}}\]
					\[\left(=\binom{n}{x}\frac{A\cdot(A-1)\cdots (A-x+1)\cdot(N-A)\cdots(N-A-n+x+1)}{N\cdots (N-n+1)}\right) \]
				\end{defi}
			
			\subsection{Stetige Verteilungen}
				\begin{defi}[Stetige Gleichverteilung]
					\[ \P(A)=\frac{\lambda(A)}{b-a}, A\subseteq[a,b], A\in\mf{B}, \]
					\[ F_X(x)=\left\{\begin{array}{ll}
					0: &x<a\\
					\frac{x-a}{b-a}: &a\le x<b\\
					1: &x\ge b
					\end{array}\right\}=\int_{-\infty}^{x}f_x(u) du, \]
					\[ f_x(x)=\begin{cases}
					\frac{1}{b-a}: &a\le x\le b\\0: &\text{sonst}
					\end{cases} \]
				\end{defi}
				
				\begin{defi}[Normalverteilung]
					\[ F_X(x)=\Phi\left(\frac{x-\mu}{\sigma}\right) \]
					\[ \P_X=\P\circ X^{-1} \]
					\[ \Phi(x)=\frac{1}{\sqrt{2\pi}}\int_{-\infty}^{x}e^{-\frac{x^2}{2}} \]
					\[ \Rightarrow \Phi\left(\frac{x-\mu}{\sigma}\right)=\frac{1}{\sqrt{2\pi\sigma^2}} \int_{-\infty}^{x}e^{-\frac{(x-\mu)^2}{2\sigma^2}}\]
				\end{defi}
				
				\begin{defi}
					Ist die Verteilungsfunktion $F_X$ differenzierbar, so heißt $f_X=F_X'$ die Dichte (-funktion) von $X$. Es gilt hierbei
					\[ f\text{ ist Dichte}\Leftrightarrow f\ge 0, \int_{-\infty}^{\infty}f(x)dx=1. \] 
				\end{defi}
				
				\begin{defi}[Exponentialverteilung $E(\lambda)$]
					\[ f(x)=\lambda e^{-\lambda x}, x\ge 0 \]
					bzw
					\[ F(x)=(1-e^{-\lambda x}), x\ge 0. \]					
				\end{defi}
				
				\begin{defi}[Gammaverteilung]
					\[ f(x)=\frac{x^{\alpha-1}\lambda^\alpha}{\Gamma(\alpha)}e^{-\lambda x}, x\ge 0, \alpha \ge 0, \lambda >0 \]
					\[ \Gamma(\alpha)=\int_0^\infty x^{\alpha-1}e^{-x}dx \]
					wobei
					\[ \Gamma(\alpha+1)=\alpha\Gamma(\alpha) \]
					und
					\[ \forall n\in\N: \Gamma(n)=(n-1)! \]
					gilt.
				\end{defi}
				
				\begin{defi}[Betaverteilung 1. Art, $B_1(\alpha, \beta)$]
					\[ f(x)=\frac{1}{B(\alpha,\beta)} x^{\alpha-1}(1-x)^{\beta-1}, 0\le x\le 1\]
					\[ B(\alpha, \beta)=\int_0^1 x^{\alpha-1}(1-x)^{\beta-1} dx=\frac{\Gamma(\alpha)\Gamma(\beta)}{\Gamma(\alpha+\beta)} \]
				\end{defi}
				
				\begin{defi}[Betaverteilung 2. Art, $B_2(\alpha, \beta)$]
					\[ f(x)=\frac{1}{B(\alpha,\beta)}\cdot \frac{x^{\alpha-1}}{(1+x)^{\alpha+\beta}}, x\ge 0.\]
				\end{defi}
				
				\begin{defi}[Chiquadratverteilung mit $n$ Freiheitsgraden]
					\[ \Gamma(\frac{n}{2},\frac{1}{2})=\chi^2 n, \]
					ist für uns derzeit nicht besonders wichtig, aber in der Statistik schon. 
				\end{defi}
				
				\begin{defi}
					Sei $X$ eine Zufallsvariable. Dann bedeutet $X\sim F$, dass $X$ die Verteilung $F$ besitzt. 
				\end{defi}
				
				\begin{bsp}
					Sei $X$ eine Zufallsvariable mit stetiger und streng monotoner Verteilungsfunktion $F_X$. Sei 
					\[ Y = F_X(X) \]
					die Verteilung von $Y$. Da $F_X$ stetig und streng monoton steigend ist, existiert eine Umkehrfunktion $F_X^{-1}$. Damit ist
					\[ F_Y(y)=\P(Y\le y)=\P(F_X(X)\le y)=\P(X\le F_X^{-1}(y))=F(F^{-1}(y))=y. \]
					Damit hat $Y$ die Verteilungsfunkiton $F(X)\sim U(0,1)$. Setzt man also eine Zufallsvariable in ihre Verteilungsfunktion ein, so erhält man eine Gleichverteilung. \newline
					Sei umgekehrt $U$ auf $[0,1]$ gleichverteilt, dann ist $F^{-1}(U)\sim F$. (Beweis übergangen)
				\end{bsp}
				
				\begin{defi}[Verallgemeinerte Inverse]
					Die Verallgemeinerte Inverse einer Verteilungsfunktion $F$ ist 
					\[ F^{-1}(y)=\inf \{x: F(x)\ge y\}. \]
					Damit folgt 
					\[ x<F^{-1}(y)\Rightarrow F(x)<y. \]
				\end{defi}
				
				\begin{satz}
					Sei $U\sim U[0,1]$ und $F$ eine Verteliungsfunktion. Dann ist
					\[ F^{-1}(U)\sim F. \] 
				\end{satz}
				
				\begin{bew}
					Es gilt
					\[ F^{-1}(U)\le x\Leftrightarrow U\le F(x), \]
					damit ist
					\[ [F^{-1}(U)\le x]=[U\le F(x)] \]
					\[ \Rightarrow \P(F^{-1})\le x)=\P(U\le F(x))=F(x), \]
					und damit sind wir fertig. 
				\end{bew}
				
				\begin{satz}
					Sei $F$ stetig und $X\sim F$. Dann gilt
					\[ F(X)\sim U(0,1). \]
				\end{satz}
				
				\begin{bew}
					Es gilt:
					\[ \P(F(X)<y)=\P(Y<F(y))=F(F^{-1}(y)-0)\stackrel{stetig}{=}F(F^{-1}(y))\stackrel{stetig}{=}y \]
				\end{bew}
				
		\section{Das Integral}
			Anschaulich: Beim Lebesgue Integral wird im Gegensatz zum Riemann-Integral der Grenzwert nicht über vertikale, sondern über horizontale "`Scheiben"' gebildet. 
			\begin{defi}
				Sei 
				\[ f:(\Omega,\mf{S})\to (\R_0^+, \mf{B}) \]
				in einem Maßraum $(\Omega, \mf{S}, \mu)$. Dann definieren wir das Lebesgue-Integral als
				\[ \int fd\mu=\int_0^\infty \mu([f>y])dy, \]
				also als uneigentliches Riemann-Integral, wobei wir das Integral gleich $\infty$ setzen, falls es ein $y$ gibt, sodass $\mu([f>y])=\infty$. 
			\end{defi}
			
			\begin{defi}[Allgemeiner]
				Für
				\[ f:(\Omega,\mf{S})\to (\R, \mf{B}) \]
				definieren wir 
				\[ f_+:=\max (f,0) \]
				\[ f_-:=\max(-f,0). \]
				Dann gilt 
				\[ f=f_+-f_- \]
				und wir können 
				\[ \int fd\mu=\int f_+d\mu-\int f_-d\mu \]
				schreiben, falls der Ausdruck definiert ist, also nicht "`$\infty-\infty$"' ist. Andernfalls sagen wir, dass $\int f d\mu$ nicht existiert. Existiert $\int fd\mu$, so nennen wir $f$ integrierbar, falls 
				\[ \int fd\mu<\infty. \] 
			\end{defi}
			
			\begin{defi}
				Sei 
				\[ \mf{L}_1(\Omega,\mf{S}, \mu):=\left\{f: \left|\int fd\mu\right|<\infty\right\}, \]
				also die Menge der integrierbaren Funktionen. 
			\end{defi}
			
			\begin{bem}[Notation]
				Man kann auch 
				\[ \int fd\mu=\int f(x) d\mu(x)=\int f(x)\mu(dx) \]
				schreiben. Wissen wir, von welchem Maß die Rede ist, so schreibt man auch einfach
				\[ \int f. \]
				Außerdem schreibt man für eine messbare Menge $A\in\mf{S}$
				\[ \int_A f=\int A()\cdot f, \]
				also das bestimmte Integral. 
			\end{bem}
			
			\begin{satz}
				$  $
				\begin{enumerate}[(1)]
					\item Sei $f\ge 0$. Dann gilt
					\[ \int f\ge 0, \]
					wobei
					\[ \int f=0\Leftrightarrow f=0 \mu-\text{fast überall} \]
					gilt. 
					\item Sei $f\le g$, so folgt
					\[ \int f\le \int g. \]
					\item Sei $c\ge 0$, dann gilt
					\[ \int cf=c\int f. \]
					\item Seien $f,g$. Dann gilt
					\[ \int (f+g)=\int f+\int g. \] 
					\item Satz von der monotonen Konvergenz, Satz von Beppo-Levi:\newline
					 Sei $f_n\uparrow f$. Dann folgt
					\[ \int f_n\uparrow \int f \]
					\item Sei $f=\sum_{i=1}^n a_i A_i(),a_i\ge 0, A_i\in\mf{S}$ disjunkt, also eine Treppenfunktion. Dann gilt
					\[ \int fd\mu=\sum_{i=1}^n a_i\mu(A_i) \]
					\item Gilt $f=g$ $\mu$-fast überall, so folgt
					\[ \int f=\int g \]
					\item 
					\[ \mu(A):=\int_A f=\int A() f \]
					ist ein Maß. 
				\end{enumerate}
			\end{satz}
			
			\begin{bew}
				(1)-(3) kann man elementar durch Übergang zum Riemann-Integral zeigen. (4) ist mühsam, daher schieben wir diesen Punkt auf. (8) ist klar, da $\mu([f\ge y])=\mu([g\ge y])$.\newline
				\begin{enumerate}[1]
					\item[(6)] Es gilt
					\[ \int f=\int_0^\infty\mu([f>y]) dy. \]
					wobei
					\[ [f>y]=\bigcup_{a_i>y} A_i \]
					und damit
					\[ \mu([f>y])=\sum_{i: a_i>y} \mu(A_i)=\sum_{i=1}^n[y<a_i]\cdot\mu(A_i) \]
					\[ \Rightarrow \int f=\int_0^\infty\sum_{i=1}^{n}[y<a_i]\mu(A_i)dy=\sum_{i=1}^n\mu(A_i)\int_0^\infty [y<a_i]dy=\sum_{i=1}^n\mu(A_i)\int_0^{a_i} dy=\sum_{i=1}^{n}\mu(A_i)\cdot a_i \]
					\arge 
					\item[(5)] Sei $f_n$ mit $f_n\le f_{n+1}$, damit folgt also nach (2)
					\[ \int f_n\le\int f_{n+1}\Rightarrow \exists I=\lim\int f_n, \]
					womit
					\[ f_n\le f: \int f_n\le \int f\Rightarrow I\le \int f \]
					und
					\[ \int f=\int_0^\infty \mu([f>y])dy=\lim\limits_{a\to 0, b\to \infty} \int_a^b\mu([f>y])dy.  \]
					Wir wählen nun $M<\int f$, also
					\[ \exists 0<t_0<t_1<...<t_N<\infty: \sum_{i=1}^{n}(t_i-t_{i-1})\cdot \mu([f>t_i]). \]
					Betrachte
					\[ [f_n>t_i]\uparrow [f>t]\Rightarrow \mu([f_n>t_i])\to \mu([f>t_i])\Rightarrow \lim\int f_n>M\Rightarrow \lim_{n\to\infty} \int f_n\ge\int f \]
					\arge
					\item[Anmerkung: ]Ist $f_n$ Treppenfunktion mit $f_n\uparrow f$, dann ist 
					\[ \int f=\lim_{n\to\infty} \int f_n, \]
					dies wird häufig als Definition des Integrals verwendet. Alternativ wird manchmal auch
					\[ \int f=\sup\left\{\int g: g\le f, g\text{ Treppenfunktion}\right\} \]
					verwendet. Wir erhalten somit ein "`Kochrezept"', um Sätze über Integrale zu beweisen:
					\[ (\text{Indikatorfunktion}\rightarrow) \text{Treppenfunktionen}\rightarrow \text{nichtnegative messbare Funktion}\rightarrow\text{messbare Funktion} \]
					\item[(4)] Seien $f,g$ Treppenfunktionen,
					\[ f=\sum_{i=1}^{n}a_i A_i()\qquad g=\sum_{j=1}^{m}b_j B_j()\qquad \sum A_i=\sum B_j =\Omega. \]
					Dann ist
					\[ f+g=\sum_{i=1}^{n}\sum_{j=1}^{m}(a_i+b_j)\cdot(A_i\cap B_j)() \]
					\[ \Rightarrow \int f+g=\sum_{i=1}^{n}\sum_{j=1}^{m} (a_i+b_j)\mu(A_i\cap B_j)=\sum_{i=1}^{n}a_i\sum_{j=1}^{m} \mu(A_i\cap B_j)+\sum_{j=1}^{m}b_j\sum_{i=1}^{n}\mu(A_i\cap B_j)=\int f+\int g, \]
					wir haben die Behauptung also für Treppenfunktionen. Seien also $f,g$ messbar und $f_n$, $g_n$ Treppenfunktionen,
					\[ f_n\uparrow f\qquad g_n\uparrow g, \]
					dann ist
					\[ f_n+g_n\uparrow f+g, \]
					also
					\[ \int (f+g)=\lim_{n\to\infty}\int (f_n+g_n)=\lim_{n\to\infty}\int f_n+\lim_{n\to\infty} \int g_n=\int f+\int g  \]
					\arge 
					\item[(8)] \zz: sigmaadditiv:\newline
					Sei
					\[ A=\sum_{n\in\N} A_n, \]
					also
					\[ A()=\sum_{n\in\N}A_n(). \]
					Damit ist
					\[ \int_A f=\int A() f=\sum_{n\in\N}A_n()f=\lim_{N\to\infty}\left(\sum_{n=1}^{N} A_n()\right)=\lim_{N\to\infty}\sum_{n=1}^N\int A_n() f=\sum_{n=1}^\infty \int A_n() f. \]
					\arge
				\end{enumerate}
				
			\end{bew}
			
			
			\begin{satz}
				Für eine messbare Funktion $f$ gilt nun
				\begin{enumerate}[(1)]
					\item $f=g$ $\mu$-fast überall, dann folgt
					\[ \int f=\int g \]
					\item 
					\[ \int cf=c\int f \]
					\item 
					\[ \int f+g=\int f+\int g \]
					\item 
					\[ \sigma(A)=\int_A fd\mu \]
					ist sigmaadditiv. 
				\end{enumerate}
			\end{satz}
				
			\begin{bew}
				Generell gilt
				\[ \int f=\int f_+-\int f_- \]
				\begin{enumerate}[(1)]
					\item ist klar
					\item ist $c>0$ so ist 
					\[ (cf)_+=c f_+, (cf)_-=cf_-, \]
					für $c<0$
					\[ (cf)_+=-cf_-, (cf)_+=-cf_+, \]
					für $c=0$ ist die Behauptung klar.\arge
					\item Wir nehmen oBdA an, dass
					\[ \int f_-<\infty \int g_-<\infty. \]
					Nun ist
					\[ (f+g)_-\le f_-+g_-.\Rightarrow \int (f+g)_-<\infty, \]
					ist also die rechte Seite definiert, so auch die linke.
					Dann ist 
					\[ f+g=(f+g)_+-(f+g)_-=f_+-f_-g_+-g_- \]
					\[ \Rightarrow(f+g)_++f_-+g_-=(f+g)_-+f_++g_+ \]
					\[ \Rightarrow\int (f+g)_+-\int f_-+\int g_-=\int(f+g)_-+\int f_++\int g_+ \]
					\[ \Rightarrow \int (f+g)=\int (f+g)_+-\int (f+g)_-=\int f_+-\int f_-+\int g_+-\int g_- \]
					\arge
					\item
					\[ \sigma(A)=\int_A fd\mu=\int_{A}f_+-\int_A f_- \]
					und damit
					\[ \sigma(\sum A_n)=\sigma_+(\sum A_n)-\sigma_-(\sigma A_n)=\sum\sigma_+(A_n)-\sum\sigma_-(A_n)=\sum\sigma(A_n), \]
					da die Integrale existieren müssen. \arge  
				\end{enumerate}
			\end{bew}
				
			\begin{satz}
				Sei $f$ fast überall messbar. Dann kann man $f$ erweitern:
				\[ \exists \tilde{f}\text{ messbar}: f=\tilde{f}\quad\mu-\text{fast überall} \]
			\end{satz}
				
			\begin{bew}
				?? - nicht mitgekommen...
			\end{bew}
			
			\begin{bem}
				Für komplexe Funkionen
				\[ f: (\Omega, \mf{S})\to(\C, \mf{B}(\C)), \]
				wobei $\mf{B}(\C)$ zweidimensionale Borelmengen sind, können wir schreiben
				\[ f=f_1+if_2, \]
				womit wir
				\[ \int f=\int f_1+i\int f_2 \]
				erhalten. Die obigen Sätze gelten somit auch für komplexe Funktionen/Konstanten. 
			\end{bem}
			
			\begin{satz}
				Es gilt für reelle Funkionen $f$:
				\[ \left|\int fd\mu\right|\le \int |f|d\mu. \]
			\end{satz}
			
			\begin{bew}
				\[ \left|\int fd\mu\right|=\left|\int f_+-\int f_-\right|\le \int f_++\int f_-=\int |f| \]
			\end{bew}
			
			\begin{bem}
				Dies gilt auch für $f:(\Omega,\mf{S})\to(\C,\mf{B}(\C))$. 
			\end{bem}
			
			\begin{defi}
				Der Erwartungswert von $X$ ist
				\[ \mathbb{E}=\int X d\P. \]
			\end{defi}
			
			\begin{satz}
				Seien $X,Y$ unabhängig mit endlichen Integralen. Dann gilt
				\[ \E(X Y)=\E(X)\E(Y). \]
			\end{satz}
			
			\begin{bew}
				Seien $X,Y$ Treppenfunktionen, $X=\sum_{i=1}^n a_i A_i(), Y=\sum_{j=1}^{m}b_j B_j()$. Nun ist
				\[ XY=\sum_{i=1}^{n}\sum_{j=1}^{m}a_ib_j(A_i\cap A_j)(). \]
				Nun gilt
				\[ \P(X=a_i, Y=b_j)=\P(A_i\cap B_j)=\P(X=a_i)\P(Y=b_j)=\P(A_i)\P(B_j) \]
				Damit folgt
				\[ \int XY=\sum_{i=1}^n\sum_{j=1}^m a_ib_j\P(A_i\cap B_j)=\sum_{i=1}^{n}\sum_{j=1}^{m}a_ib_j\P(A_i)\P(B_j)=\int X\int Y. \]
				Seien nun $X,Y\ge 0$ messbar. Dann können wir 
				\[ X=\lim_{n\to\infty} X_n,\qquad X_n:=\frac{1}{2^n}\lfloor\min(X,n)\cdot 2^n\rfloor \]
				\[ \Rightarrow X_n\uparrow X, Y_n\uparrow Y\Rightarrow X_nY_n\uparrow XY. \]
				\[ \Rightarrow \int XY=\lim\int X_nY_n=\lim \int X_n\int Y_n=\int X\int Y. \]
				Für den Allgemeinen Fall teilen wir 
				\[ X=X_+-X_- \]
				\[ Y=Y_+-Y_- \]
				auf. Damit folgt
				\[ \int XY=\int (X_+-X_-)(Y_+-Y_-)=\stackrel{\text{selbst überprüfen}}{...}=\int X\int Y \]
			\end{bew}
			
			\begin{defi}
				Gilt
				\[ \int XY=\int X\int Y, \]
				dann heißten $X$ und $Y$ unkorreliert. 
			\end{defi}
			
			\begin{satz}
				Seien $(\Omega_1,\mf{S}_1,\mu)$ und $(\Omega_2,\mf{S}_2)$ ein Maß- und ein Messraum. Seien weiters 
				\[ f: (\Omega_1,\mf{S}_1)\to(\Omega_2,\mf{S}_2) \]
				\[ g: (\Omega_2,\mf{S}_2)\to(\overline{\R},\mf{B}(\overline{\R})). \]
				Dann gilt
				\[ \int g\circ f d\mu=\int gd\mu_f, \]
				wobei (Erinnerung)
				\[ A\in\mf{S}_2: \mu_f(A)=\mu\circ f^{-1}(A)=\mu(f^{-1}(A)). \]
			\end{satz}
			
			\begin{defi}
				Ist $g$ Indikatorfunktion, so die Behauptung klar. Aufgrund der Linearität des Integrals gilt dann
				\[ \int A\circ f d\mu=\mu(f^{-1}(A))\mu_f(A).  \]
				Und damit, wenn $g=\sum_{i=1}^{n}a_i A_i$
				\[ \int g\circ f=\int \sum_{i=1}^n a_i A_i\circ f d\mu=\sum_{i=1}^na_I\int A_i\circ f\dagger\mu=\sum_{i=1}^{n}\int A_id\mu_f=\int g d\mu_f \]
				und somit für $g\ge 0$, falls
				\[ g_n\uparrow g\Rightarrow g_n\circ f\uparrow g\circ f \]
				\[ \Rightarrow \int g\circ fd\mu=\lim \int g_n\circ fd\mu=\lim\int g_nd\mu_f=\int gd\mu_f. \]
			\end{defi}
			
			\begin{bsp}
				Sei $f: [a,b]\to\R$ stetig. Dann gilt
				\[ \int_{[a,b]} fd\lambda=\int_a^b f(x)dx. \]
				Beweis in der Übung. 
			\end{bsp}
			
			\begin{bsp}
				Sei $(\Omega,2^\Omega,\zeta)$ ein Maßraum, $\zeta$ das Zählmaß, $\zeta(A)=|A|$. Dann gilt
				\[ \int f(\omega)d\zeta(\omega)=\sum_{\omega\in\Omega} f(\omega). \]
			\end{bsp}
			
		\section[Integration$\leftrightarrow$Grenzwert]{Vertauschbarkeit von Integration und Grenzübergang}
			Der folgende Satz ist schon bekannt:
			\begin{satz}[Beppo-Levi, monotone Konvergenz]
				Sei $f_n\ge 0, f_n$ messbar mit $f_n\uparrow f$. Dann folgt
				\[ \int f=\lim_{n\to\infty}\int f_n. \] 
			\end{satz}
			wir können ihn allerdings noch wie folgt verbessern:
			\begin{satz}
				Seien $f_n$ fast überall messbar, 
				\[ f_n(\omega)\uparrow \text{fast überall,} \]
				 $f_n\ge g$ fast überall für eine Funktion $g$ mit 
				 \[ \int g>-\infty, \]
				 bzw äquivalent dazu, $g\in\mf{L}_1$. Dann gilt
				 \[ \lim \int f_n=\int f. \] 
			\end{satz}
			\begin{bew}
				Es gilt
				\[ f_n+g_-\ge 0\text{ fast überall} \]
				\[ \Rightarrow f_n+g_n\uparrow f+g_-\text{ fast überall} \]
				\[ \Rightarrow \int f+g_-=\lim\int f_n+g_- \]
				\[ \Rightarrow \int f+\cancel{\int g_-}=\lim\int f_n +\cancel{\int g_-} \].
			\end{bew}
			
			\begin{lemma}[Fatou]
				Sei $f_n\ge g$, $g\in\mf{L}_1$. Dann gilt
				\[ \int \liminf f_n\le\liminf\int f_n. \]
				Gilt $f_n\le g$, $g\in\mf{L}_1$, dann gilt
				\[ \limsup\int f_n\le\limsup\int f_n \]
			\end{lemma}
			
			\begin{bew}
				Es gilt
				\[ \liminf f_n=\sup_{n\to\infty}\underbrace{\inf_{k\ge n} f_k}_{=:h_n}. \]
				Nun ist
				\[ h_n\le h_{n+1}, h_n\uparrow \liminf f_n. \]
				Damit ist
				\[ \int \liminf f_n=\lim\int h_n=\liminf \int h_n\le\liminf\int f_n \]
				Analog zeigt man die zweite Behauptung. 
			\end{bew}
			
			\begin{satz}[Deckelung, Satz von Lebesgue, Satz von der Konvergenz durch Majorisierung, Satz von der dominierten Konvergenz]
				Seien $f_n,f$ fast überall messbar, $f_n\to f$ entweder fast überall oder im Maß. Gelte weiters $|f_n|\le g$ fast überall, $g\in\mf{L}_1$. Dann gilt
				\[ \lim\int f_n=\int f. \]
				Weiters gilt
				\[ \lim\int |f_n-f|=0. \]
			\end{satz}
			
			\begin{bew}
				Es ist nichts mehr zu tun, außer für den Fall $f_n\to f$ fast überall. Wir setzen nun alle "`fast überalls"' außer Kraft, indem wir die $f_n,f$ dort gleich 0 setzen. Dann kann man das Lemma von Fatou anwenden. \newline
				Die zweite Behauptung folgt dann trivial für fast überall, für Konvergenz im Maß indirekt: Sei $f_n\to f$ im Maß und wir nehmen an, dass
				\[ \int f_n\not\to\int f. \]
				Nun existiert eine Teilfolge $f_{n_k}$ mit 
				\[ \left|\int f_{n_k}-f\right|>\varepsilon. \]
				Nun geht allerdings 
				\[ f_{n_k}\to f\text{ fast überall,} \]
				also der Widerspruch
				\[ \int f_{n_k}\to\int f. \]
			\end{bew}
			
			\begin{bem}
				Nie wieder $\varepsilon$. (außer gleich unten in der Definition...)
			\end{bem}
			
			\begin{defi}[Gleichmäßige Integrierbarkeit]
				Sei $f_n, f_n\in\mf{L}_1$ heißt gleichmäßig integrierbar, wenn es für jedes $\varepsilon>0$ eine $g_\varepsilon\in\mf{L}_1$ gibt, sodass
				\[ \forall n\in\N: \int |f_n|[|f_n>g_\varepsilon]<\varepsilon,  \]
				oder äquivalent
				\[ \exists h_\varepsilon\in\mf{L}_1, h_\varepsilon\ge 0: \int (|f_n|-f_\varepsilon)<\varepsilon. \]
			\end{defi}
			
			\begin{bew}
				Die Äquivalenz folgt durch ($x,y\ge 0$)
				\[ x[x>y]\ge(x-y)_+ \land x[x>y]\le 2(x-\frac{y}{2})_+.\]
			\end{bew}
			
			\begin{satz}
				Konvergiert $f_n\to f$ fast überall oder im Maß und ist $f_n$ gleichmäßig integrierbar, so folgt
				\[ \int f_n\to\int f. \]
			\end{satz}
			
			\begin{bew}
				Seien oBdA $f_n\ge 0$. Denn ist dies nicht der Fall, können wir $f_n$ in positiven und negativen Teil zerlegen. Da $f_n$ gleichmäßig integrierbar sind, folgt
				\[ \Rightarrow \forall\varepsilon\exists g_\varepsilon\in\mf{L}_1,  g_\varepsilon\ge 0: \int (\underbrace{|f_n|}_{=f_n}-g_\varepsilon)_+\le \varepsilon. \]
				Wir zerlegen nun
				\[ f_n=\min(f_n,g_\varepsilon)+(f_n-g_\varepsilon)_+. \]
				Nun konvergiert $\min(f_n,g_\varepsilon)\to \min(f,g_\varepsilon)$ und 
				\[ \lim_{n\to\infty}\int\min(f_n,g_\varepsilon)=\int\min(f,g_\varepsilon)  \]
				Wir wissen auch, dass
				\[ (f_n-g_\varepsilon)_+\to(f_n-g_\varepsilon)_+. \]
				Nun gilt:
				\[ \int f_n=\int \min(f_n,g_\varepsilon)+\int(f_n-g_\varepsilon)_+ \]
				\[ \Rightarrow\limsup_{n\to\infty} \int f_n=\int\min(f,g_\varepsilon)+\limsup_{n\to\infty} \int (f_n-g_\varepsilon)_+\le\int f+\varepsilon \]
				\[ \Rightarrow\limsup_{n\to\infty}\int f_n\le\int f \]
				Und damit
				\[ \int f=\int \liminf_{n\to\infty} f_n\le \liminf_{n\to\infty} \int f_n \]
			\end{bew}
			
			\paragraph{\it Wichtige Beweise für die Prüfung}
			\begin{itemize}
				\item monotone class theorem
				\item Fortsetzungssatz für Maßfunktionen ($\mu^*$ ist äußeres Maß $\Rightarrow$ $\mf{M}$ ist Sigmaalgebra $\land$ $\mu^*|_\mf{M}$ ist Maß)
				\item $F$ rechtsstetig und nicht-fallend $\Leftrightarrow$ $F$ ist Verteilungsfunktion, d.h.
				\[ \mu_F(]a,b])=F(b)-F(a). \]
				\item "`Approximationssatz"': Ist $f: (\Omega_1,\mf{S}_1)\to(\R,\mf{B})$, so existiert eine Folge von Treppenfunktionen $(f_n)$, so, dass $f_n\to f$, $|f_n|\le|f|$. 
				\item Satz von der dominierten Konvergenz
				\item starkes Gesetz der großen Zahlen 
				\item Zerlegungssatz von Jordan
				\item Zerlegungssatz von Lebesgue
			\end{itemize}
			Nur, was wir noch durchbringen. 
			
			\begin{satz}[Umkehrung des vorhergehenden Satzes]
				Seien $f_n\ge 0$, $f_n,f\in\mf{L}_1$, $f_n\to f$, mit 
				\[ \int f_n\to\int f, \]
				so folgt, dass $(f_n)$ gleichmäßig integrierbar ist.  
			\end{satz}
			
			\begin{bew}
				Es gilt
				\[ f_n=\min(f_n,f)+(f_n-f)_+ \]
				\[ \Rightarrow\underbrace{\int f_n}_{\to \int f}=\underbrace{\int\min(f_n,f)}_{\to\int\min(f,f)=\int f}+\int (f_n-f)_+ \]
				\[ \Rightarrow \int (f_n-f)_+\to 0, \]
				es existiert also ein $n_0\in\N$, sodass
				\[ \forall n>n_0: \int (f_n-f)<\varepsilon, \]
				Nun können wir 
				\[ g_\varepsilon=\max(f_1,...,f_n,f) \]
				wählen und sind fertig. 
			\end{bew}
			
			\begin{defi}
				Sei $X: (\Omega,\mf{S})\to(\overline{\R},\mf{B})$. Ist dann
				\[ \E(X^2)<\infty, \]
				so heißt $\E(X^2)$ das 2. Moment. 
			\end{defi}
			
			\begin{defi}
				$\E(X^k)$ heißt das $k$-te Moment. 
			\end{defi}
			
			\begin{bem}
				Es gilt
				\[ \E(|X|)\le\E(|X|[|X|<1])+\E(|X|[|X|\ge 1]) \le 1+\E(X^2[|X|\ge 1])\le 1+\E(X^2) \]
				Damit können wir die Varianz $\V$ von $X$ definieren:
			\end{bem}
			
			\begin{defi}
				Die Varianz $\V$ von $X$ ist
				\[ \V(X)=\E(X-\E(X))^2.  \]
			\end{defi}
			
			\begin{satz}
				Es gilt
				\[ \V(c\cdot X)=c^2\V(X) \]
				\[ \V(X+a)=\V(X) \]
				Steiner'scher Verschiebungssatz: 
				\[ \E((X-a)^2)=\E(X-\E(X)+\E(X)-a)=\V(X)+2(\E(X)-a)\cdot\underbrace{\E(X-\E(X))}_{\E(X)-\E(X)=0}+(\E(X)-a)^2 \]
				Dies liefert für $a=0$:
				\[ \V(X)=\E(X^2)-\E(X)^2.  \]
			
			\end{satz}
			
			\begin{bew}
				klar. 
			\end{bew}
			
			\begin{bsp}[Gleichverteilung]
				Sei $U[0,1]$. Dann ist
				\[ \E(X)=\int_{[0,1]}xd\lambda(x)=\int_0^1 x dx=\frac{1}{2}, \]
				\[ \Rightarrow \E(X^2)=\int_0^1 x^2 dx=\frac{1}{3}. \]
				\[ \Rightarrow \V(X)=\frac{1}{12} \]	
			\end{bsp}
			
			\begin{defi}[Covarianz]
				Die Covarianz ist
				\[ \Cov(X,Y)=\E((X-\E(X))(Y-\E(Y)))=\E(XY)=\E(X)\E(Y). \]
			\end{defi}
			
			\begin{satz}
				Es gilt
				\begin{align*}
				 \V(X+Y)&=\E((X+Y-\E(X+Y))^2)\\
				 &=\E(X-\E(X)^2+2\E((X-\E(X))(Y-\E(Y))))+\E((Y-\E(Y))^2)\\
				 &=\V(X)+\V(Y)+2\Cov(X,Y)
				\end{align*}
			\end{satz}
			
			\begin{satz}
				Sind $X,Y$ unabhängig (oder unkorelliert,  $\Cov(X,Y)=0$), dann gilt
				\[ \V(X+Y)=\V(X)+\V(Y) \]
			\end{satz}
			
			\begin{satz}[Ungleichung von Markov]
				Seien $(\Omega,\mf{S}, \mu)$, $c>0$ und 
				\[ f: \Omega\to\R^+_0. \]
				Dann ist 
				\[ \mu([f\ge c])\le\frac{1}{c}\int fd\mu \]
			\end{satz}
			
			\begin{bew}
				Sei 
				\[ g: \Omega\to \R, \omega\mapsto g(\omega):=\begin{cases}
					c: & f(\omega)\ge c\\
					0: &\text{sonst}
				\end{cases}. \]
				Jetzt gilt
				\[ g\le f\Rightarrow\int g\le\int f \]
				\[ \Rightarrow \int g=c\cdot \mu([f\ge c]). \]
			\end{bew}
			
			\begin{satz}[Ungleichung von Chebychev]
				Sei $(\Omega,\mf{S}, \P)$ ein Wahrscheinlichkeitsraum, $X$ eine Zufallsvariable, $\V(X)<\infty$, $c>0$. Dann gilt
				\[ \P(|X-\E(X)|>c)\le\frac{\V(X)}{c^2}. \]				
			\end{satz}
			
			\begin{bew}
				trivial?
			\end{bew}
			
			\begin{satz}[Schwaches Gesetz der großen Zahlen]
				Seien $(X_n)$ unkorrelliert, $\E(X_n)=m$, $\V(X_n)=\sigma^2<\infty$ Dann gilt mit
				\[ S_n=X_1+...+X_n, s_0=0 \]
				und
				\[ \overline{X_n}=\frac{S_n}{n} \]
				das folgende:
				\[ \overline{X_n}\to m\text{ in Wahrscheinlichkeit.} \]
			\end{satz}
			
			\begin{bew}
				Sei $\epsilon >0$. Dann soll
				\[ \P(|\overline{X_n}-m|\ge \varepsilon)=\P(|S_n-n\cdot m|\ge n\varepsilon)=\E(|S_n-\E(S_n)|\ge n\cdot \varepsilon) \]
				\[ \le \frac{\V(S_n)}{(n\cdot \varepsilon)^2}=\frac{n\sigma^2}{n^2\epsilon^2}=\frac{\sigma^2}{n\varepsilon^2}\to 0\lightning \]
			\end{bew}
			
		\section[Starke Grenzwertsätze]{Starke Grenzwertsätze (Konvervenz fast sicher)}
			\begin{bem}
				Konvergenz in Wahrscheinlichkeit: Wir müssen $\P(|\overline{X}_n-\E(x)|\ge \varepsilon)$ kontrollieren. \newline
				Wir wollen nun für die Konvergenz fast sicher
				\[ \P(|\overline{X}_n-\E(x)|\ge \varepsilon\text{ für }n\ge n_0) \]
				\[ \Rightarrow \P(\sup_{n\ge n_0} |\overline{X}_n-\E(x)|\ge\varepsilon) \]
				Abschätzungen erhalten.
			\end{bem}
			
			\begin{satz}[Maximalungleichungen]
				Sei $X_1,...,X_n,...$ eine Folge von Zufallsvariablen, die unabhängig sind. Sei dann wieder $S_0=0$, $S_n=X_1+...+X_n$. Nun gilt
				\begin{enumerate}
					\item Ungleichung von Lévy-Ottaviani: ($a,b>(\ge)0$)				
					\[ \P(\max_{k\le n}|S_k|\ge a+b)\le\frac{\P(S_n\ge a)}{1-\max\limits_{1\le i\le n}\P(|S_n-S_i|\ge b)} \] 
					\item Ungleichung von Kolmogorov: Sei $\E(X_i)=0\forall i=1,...,n$, $\V(X_i)<\infty, c>0$. Dann gilt
					\[ \P(\max_{k\le n}|S_k|\ge c)\le\frac{\V(S_n)}{c^2} \]
				\end{enumerate}
			\end{satz}
			
			\begin{bew} $ $
				\begin{enumerate}
					\item Lévy-Ottaviani:\newline
						Sei $A_k=[|S_i<a+b, \forall i<k\land |S_k|\ge a+b]$. Nun sind $A_1,...,A_n$ disjunkt. Nun ist
						\[ [\max\limits_{k\le n}|S_k|\ge a+b]=\bigcup_{k=1}^n A_k \]
						\[ \Rightarrow \P(\max_{k\le n} |S_n|)=\sum_{k=1}^n \P(A_k) \]
						\[ \Rightarrow \P(|S_n|\ge a)\ge \P(|S_n|\ge a, \max_{k\le n}|S_k|\ge a+b)=\P\left([|S_n|\ge a\cap\bigcup_{k=1}^nA_k\right)=\sum_{k=1}^{n}\P(A_k\cap[|S_n|\ge a]) \]
						\[ \ge \sum_{k=1}^n\P(\underbrace{A_k}_{\text{hängt von $X_1,...,X_k$ ab}}\cap[\underbrace{|S_n-S_k|}_{\text{hängt von $X_{k+1},...,X_n$ ab}}\le b) \]
						\[ \ge \sum_{k=1}^n\P(A_k)\left(1-\underbrace{\P(|S_n-S_k|\ge b)}_{\le \max_{i\le n}\P(|S_n-S_i|\ge b)}\right) \]
						\[ \ge (1-\max_{i\le n} \P(|S_n-S_i|\ge b))\sum_{k=1}^n A_k \]
						\arge
					\item Kolmogorov:\newline
						Es gilt:
						\[ \V(S_n)=\E(S_n^2)\ge\E(S_n^2\cdot [\max_{k\le n}|S_n|\ge c]) \]
						Sei nun $A_k=[|S_i|<c, \forall i=1,...,k-1]$. Damit:
						\[ \E(S_n^2\cdot [\max_{k\le n}|S_n|\ge c])=\E(S_n^2\cdot\sum_{k=1}^n A_k)=\sum_{k=1}^n\E(S_n^2A_k)=\sum_{k=1}^n\E((S_k+S_n-S_k)^2A_k) \]
						\[ =\sum_{k=1}^n\left(\E(S_k^2\cdot A_k)+\underbrace{\cancel{2\E(S_kA_k(S_n-S_k))}}_{\text{Ereignisse unabhängig}}+\underbrace{\cancel{\E(A_k(S_n-S_k)^2)}}_{\text{nach unten abschätzen}}\right) \]
						\[ \ge c^2 c^2\sum_{k=1}^n\P(A_k)=c^2\P(\max_{k\le n}|S_k|\ge c).  \]
				\end{enumerate}
			\end{bew}
			
			\begin{bem}
				Nun stellen wir uns die folgende Frage:
				Wann konvergiert $\sum_{n\in\N}X_n$? Das bedeutet, wann konvergiert $\lim_{n\to\infty}S_n$. \newline
				Hier wie auch in den folgenden Sätzen sind $X_n$ immer unabhängig. 
			\end{bem}
			
			\begin{satz}[Äquivalenzzatz von Lévy]
				$\sum_{n\in\N}X_n$ konvergiert genau dann fast sicher, wenn sie in Wahrscheinlichkeit konvergiert. 
			\end{satz}
			
			\begin{bew}
				Konvergiert die Reihe fast sicher, so konvergiert sie auch in Wahrscheinlichkeit, da wir uns in einem Wahrscheinlichkeitsraum, also einem endlichen Maßraum, befinden und somit den Satz von Egorov anwenden können. Wir betrachten also die andere Richtung:\newline
				Konvergiere $(S_n)$ in Wahrscheinlichkeit. Dann ist $(S_n)$ Cauchy-Folge in Wahrscheinlichkeit, d.h. 
				\[ \forall\varepsilon>0\forall\delta>0\exists n_0\in\N\forall n,m\ge n_0: \P(|S_n-S_m|\ge \frac{\varepsilon}{2})<\delta. \]
				Nun gilt
				\[ \P(\sup_{n\ge n_0}|S_n-S_{n_0}|>\varepsilon)=\P\left(\bigcup_{N\ge n_0}[\max_{n_0\le n\le N} |S_n-S_{n_0}|>\varepsilon]\right) \]
				\[ =\lim_{N\to\infty}\P(\max_{n_0\le n\le N}|S_n-S_{n_0}|>\varepsilon)\stackrel{\text{Lévy-Ottaviani}}{\le}\frac{\P(|S_N-S_{n_0}|\ge\frac{\varepsilon}{2})}{1-\max_{n_0\le n\le N}\P(|S_N-S_n|\ge\frac{\varepsilon}{2})}\le \frac{\delta}{1-\delta} \]
				Sei nun $A_n=[\sup\limits_{m\ge n}|S_m-S_n|>\varepsilon]$. Damit ist
				\[ \lim_{n\to\infty}\P(A_n)=0 \]
				wobei natürlich, da $A_n$ fallend ist 
				\[ \lim_{n\to\infty}\P(A_n)=\P(\bigcup A_n)=\P(|S_m-S_n|>\varepsilon\text{ "`unendlich oft"'})=\P(\limsup S_n-\liminf S_n>\varepsilon) \]
				\[ \Rightarrow \P(\limsup S_n\neq \liminf S_n)=0 \]
				womit $S_n$ fast sicher konvergiert. 
			\end{bew}
			
			\begin{satz}[Dreireihensatz von Kolmogorov]
				$\sum_{n\in\N}$ konvergiert genau dann, wenn die folgenden 3 Bedingungen erfüllt sind: ($\forall \epsilon>0$)
				\begin{enumerate}[(1)]
					\item $\sum_{n\in\N}\P(|X_n|>\varepsilon)$
					\item $\sum_{n\in\N}\E(X_n\cdot[|X_n|\le \varepsilon])$
					\item $\sum_{n\in\N}\V(X_n\cdot[|X_n|\le \varepsilon])$
				\end{enumerate}
			\end{satz}
			
			\begin{bew}
				Zu zeigen: Sind alle $|Y_n|\le c$, $Y_n$ unabhängig, so ist
				\[ \sum Y_n\text{ konvergiert}\Leftrightarrow\!\!\!\!\not \quad\sum \E(Y_n)\text{ konvergiert}\land \sum \V(Y_n)\text{ konvergiert}. \]
				Wir zeigen nun nur eine Richtung (von rechts nach links), für die andere fehlt uns noch ein Werkzeug. (siehe Maß 2) 
			\end{bew}
			
			\begin{lemma}[Kronecker]
				Für eine Folge $(a_n)\ge 0, a_n\uparrow\infty$, sodass $\sum_{n\in\N}\frac{x_n}{a_n}$ konvergiert, folgt \[\frac{1}{a_n}\sum_{i=1}^n x_i\to 0.\]
			\end{lemma}
			
			\begin{bew}
				Sei 
				\[ s_n=\sum_{k=n}^\infty \frac{x_n}{a_n}, \]
				also $s_n\to 0$. Nun gilt
				\[ x_n=(s_n-s_{n+1})\cdot a_n \]
				\[ \Rightarrow  \sum_{i=1}^n x_i=\sum_{i=1}^na_i(s_i-s_{i+1})=\sum_{i=1}^n a_is_i-\sum_{n=2}^{n+1} a_{i-1}s_i=a_1s_1-a_ns_{n+1}+\sum_{i=2}^n(a_i-a_{i-1})s_i \]
				\[ \Rightarrow\left|\sum_{i=1}^n x_i\right|\le |a_1s_1|+|a_ns_{n+1}|+\sum_{i=2}^n(a_i-a_{i-1})|s_i|\]
				\[=|a_1s_1|+|a_ns_{n+1}|+\sum_{i=2}^{n_0} (a_i-a_{i-1})|s_i|+\sum_{i=n_0+1}^n(a_i-a_{i-1})|s_i|\le c(\varepsilon)+2a_n\varepsilon  \]
				\[ \Rightarrow\limsup_{n\to\infty}\left|\frac{1}{a_n}\right|\left|\sum_{i=1}^n x_i\right|\le 2\varepsilon  \]
			\end{bew}
			
			\begin{satz}[Cesaro]
				Gelte $a_n\to a$. Dann folgt, dass
				\[ \frac{1}{n}\sum_{i=1}^n a_i\to a. \]
			\end{satz}
			
			\begin{bew}
				Ana letztes Semester. (glaub ich)
			\end{bew}
			
			\begin{satz}[Starkes Gesetz der großen Zahlen]
				$ $
				\begin{enumerate}[(1)]
					\item Starkes Gesetz von Kolmogorov I:\newline
						Sei $(X_n)$ eine Folge von unabhängigen Zufallsvariablen, $m_n:=\E(X_n)$, $\sum_{i=1}^{\infty}\frac{\V(X_n)}{n^2}<\infty$. Dann gilt mit Wahrscheinlichkeit 1:
						\[ \overline{X}_n-\overleftarrow{m}_n=\sum_{i=1}^n\frac{X_i-\E(X_i)}{n}\to 0. \]
					\item Starkes Gesetz von Kolmogorov II:\newline
						Seien $(X_n)$ unabhängig identisch verteilt und integrierbar,
						\[ \E(X_n)=m \text{ endlich}, \]
						so gilt
						\[ \overline{X}_n\to m\text{ fast sicher} \]
				\end{enumerate}
			\end{satz}
			
			\begin{bew}
				$ $
				\begin{enumerate}[(1)]
					\item Gestern haben wir schon gezeigt:
						\[ Y_n\text{ unabhängig, }\sum_{n\in\N}\V(Y_n)<\infty\Rightarrow\sum_{n\in\N} Y_n\text{ ist konvergent}. \]
						Nun müssen wir nur noch geschickt einsetzen: 
						\[ \frac{\V(X_n)}{n^2}=\V\left(\frac{X_n-\E(X_n)}{n}\right) \]
						\[ \Rightarrow \sum_{n\in\N} \frac{X_n-\E(X_n)}{n}\text{ konvergiert.} \]
						Nach dem Kronecker-Lemma für 
						\[ a_n=n, x_n=X_n-\E(X_n), \]
						so konvergiert
						\[ \frac{1}{n}\sum_{n\in\N} (X_n-\E(X_n))\to 0, \]
						also genau die Behauptung. 
					\item Trick: (Konrad $\to$ abschneiden(winsorize))\newline
						Seien 
						\[ Y_n=X_n[|X_n|\le n. \]
						Nun ist
						\[ \P(Y_n\neq X_n)=\P(|X_n|>n) \]
						\[ \Rightarrow \P(Y_n\neq X_n)=\sum_{n=1}^\infty \P(|X_n|>n)=\sum_{n=1}^\infty\P(|X_1|>n)\le \int_0^\infty\P(|X_1|>y)dy=\E(|X_n|)<\infty \]
						Nach Borel-Cantelli ist $Y_n\neq X_n$ nur für endlich viele $n$, womit $\overline{X}_n$ genau dann konvergiert, wenn $\overline{Y}_n$ konvergiert und im konvergenten Fall die Grenzwerte übereinstimmen.\newline
						Es gilt:
						\[ |\E(X_n)-\E(Y_n)|=|\E(X_n\cdot[|X_n|>n])|=|\E(X_1[|X_1|>n])\le\E(|X_1|\cdot[|X_1|>n])\to 0.  \]
						\[ \Rightarrow \lim_{n\to\infty}\E(Y_n)=\E(X_1). \]
						Nach dem Satz von Cesaro gilt nun
						\[ \sum_{n\in \N} \frac{\V(Y_n)}{n^2}\le \sum_{n\in\N}\frac{\E(Y_n^2)}{n^2}=\sum_{n\in\N}\frac{\E(X_n^2\cdot[|X_n|\le n])}{n^2}=\sum_{n=1}^\infty\int \frac{x^2\cdot[|x|\le n]}{n^2}d\P_X(x) \]
						\[ =\int x^2\sum_{n\in\N}\frac{[|x|\le n]}{n^2}d\P_X(x)=\int x^2\sum_{n\ge |x|} \frac{1}{n^2}d\P_X(x). \]
						Nun ist
						\[ \sum_{n=m}\frac{1}{m^2}\le \frac{1}{m^2}+\int_m^\infty\frac{dx}{x^2}=\frac{1}{m^2}+\frac{1}{m}\int x^2\sum_{n=1}^\infty\frac{[|x|\le n]}{n^2}d\P_X(x)=\int x^2\int_{n\ge|x|}\frac{1}{n^2}d\P_X(x)\]
						\[\le \int x^2(\frac{1}{x^2}+\frac{1}{|x|}d\P_X(x)=1+\E(|X|)<\infty \]
						womit die Behauptung gezeigt ist. 
				\end{enumerate}
			\end{bew}
			
			\begin{satz}[Umkehrung des vorigen Satzes]
				Seien $(X_n)$ (paarweise) unabhängig und identisch verteilt. Ist nun $\lim\limits_{n\to\infty}\overline{X}_n$ endlich (mit positiver Wahrscheinlichkeit), so folgt
				\[ \E(X_n)\text{ ist endlich.} \]
				Ist $\E(|X_n|)=\infty$, so konvergiert $\overline{X}_n$ mit Wahrscheinlichkeit 1 nicht.
			\end{satz}
			
			\begin{bew}
				Ist $\lim\overline{X}_n(\omega)$ endlich ($=a$) so ist
				\[ a=\lim_{n\to a}\overline{X}_n(\omega)=\lim_{n\to\infty}\frac{n-1}{n}\overline{X}_{n-1}(\omega) \]
				\[ \Rightarrow 0=\lim_{n\to\infty}\left(\overline{X}_n-\frac{n-1}{n}\overline{X}_{n-1}\right)=\lim_{n\to\infty}\frac{X_n}{n}\]
				\[ \Rightarrow \sum_{n\in\N}\P(|X_n|>n)=\sum_{n\in\N}\P(|X_n|>n)=\infty \]
				also ist mit Wahrscheinlichkeit 1 für unendlich viele $n$ $|X_n|>n$ und somit $\frac{|X_n|}{n}>1$, womit der Widerspruch folgt. 
			\end{bew}
			
			
		\section{Riemann- und Lebesgueintegral}
			\begin{bem}
			Wann ist $f:[a,b]\to\R$ Riemann-integrierbar? Wir wissen aus der Analysis:
			\begin{enumerate} [(1)]
				\item $f$ muss beschränkt sein.
				\item Ober/Untersumme:
					\[ O_n=\sum_{i=1}^{N_n} (t_{n,i}-t_{n,i-1})\cdot\sup f([t_{n,i-1},t_{n,i}]) \]
					\[ U_n=\sum_{i=1}^{N_n} (t_{n,i}-t_{n,i-1})\cdot\inf f([t_{n,i-1},t_{n,i}]) \]
					Wir basteln nun Funktionen 
					\[N_n=2^n, t_{n,i}=a+\frac{b-a}{2^n}\cdot i, i=0,...,2^n\]
					\[ \overline{f}_n=\sum_{i=1}^{2^n} ]t_{n,i-1},t_{n,i}]()\cdot \sup f([t_{n,i-1}, t_{n,i}]) \]
					\[ \underline{f}_n=\sum_{i=1}^{2^n} ]t_{n,i-1},t_{n,i}]()\cdot \inf f([t_{n,i-1}, t_{n,i}]) \]
					Nun ist 
					\[ \int \overline{f}_nd\lambda= O_n, \int\underline{f}_nd\lambda=U_n. \]
					Außerdem ist $\overline{f}_n$ fallend und $\underleftarrow{f}_n$ steigend, also 
					\[\overline{f}_n\ge \overline{f}_{n+1}\ge f\]
					\[ \underline{f}_n\le \underline{f}_{n+1}\le f.\]
					Aus den Sätzen von der monotonen Konvergenz wissen wir, dass 
					\[ \lim_{n\in\N} \overline{f}_n=\overline{f} \text{   und   }\lim_{n\in\N} \underline{f}_n=\underline{f}\]
					existieren. Außerdem ist 
					\[ \int \overline{f}d\lambda=\lim_{n\in\N} \int\overline{f}_nd\lambda=\lim_{n\in\N} O_n \]
					\[ \int \underline{f}d\lambda=\lim_{n\in\N}\int \underline{f}_nd\lambda=\lim_{n\in\N} U_n. \]
					Damit ist $f$ genau dann Riemann-integrierbar, falls
					\[ \int \overline{f}d\lambda=\int \overline{f}d\lambda. \]
					Es gilt jedenfalls
					\[ \underline{f}\le\overline{f}. \]
					Damit ist obiges auch äquivalent zu 
					\[ \overline{f}=\underline{f} \qquad \lambda-\text{fast überall.} \]
					Wann stimmen sie nun überein? Dazu nehmen wir $x\in(a,b]$. Ist $f$ stetig in $x$, so folgt aus der $\varepsilon-\delta$ Definition von Stetigkeit, dass $\lim_{n\in\N} \overline{f}_n(x)=\lim_{n\in\N}\underline{f}_n(x)$. Da die Endpunkte der Invervalle abzählbar sind, haben sie Lebesgue-Maß 0. Damit erhalten wir insgesamt den folgenden Satz:
			\end{enumerate}
		\end{bem}
		
		\begin{satz}
			Eine Funktion $f$ ist im Intervall $[a,b]$ genau dann Riemann-integrierbar, falls die Menge $\mathcal{D}f$ der Unstetigkeitsstellen Lebesguemaß 0 hat,
			\[ f\text{ Riemann-integrierbar }\Leftrightarrow \lambda(\mathcal{D}f)=0. \]
			In diesem Fall gilt
			\[ \int_a^bf(x)dx=\int_{]a,b]} fd\lambda. \]
		\end{satz}
		
		\begin{defi}[Stieltjes-Integral]
			Das Integral
			\[ \int fdF \]
			wird als (Riemann-)Stieltjes-Integral bezeichnet, dabei ist
			\[ \int fdF=\lim \sum_{i=1}^n (F(t_{n,i})-F(t_{n,i-1}))\cdot f(x_{n,i}). \]
		\end{defi}
		
		\begin{bsp}
			Wir betrachten
			\[ \int_0^\infty \frac{1}{x}\sin(x)dx. \]
			Dieses konvergiert nicht absolut, und es gilt
			\[ \int_0^\infty \frac{1}{x}\sin(x)dx\neq\int_{[0,\infty[}\frac{1}{x}\sin(x)d\lambda(x). \]
			Wir können uns allerdings durch Grenzwertbildung retten:
			\[ \int_0^\infty \frac{1}{x}\sin(x)dx=\lim_{M\to\infty}\int_{]0,M]}\frac{1}{x}\sin(x)d\lambda(x) \]
		\end{bsp}
		
	\chapter{Signierte Maße}
		\begin{defi}
			Sei $\mu$ eine Mengenfunktion auf einer Sigmaalgebra $\mf{S}$. $\mu$ heißt signiertes Maß, falls
			\[ \mu:\mf{S}\to\overline{\R}\text{ und }\mu \text{ sigmaadditiv}. \]
		\end{defi}
		
		\begin{bem}
			Für die Sigmaadditivität verlangen wir, dass für disjunkte $(A_n)$
			\[ \mu(\bigcup_{n\in\N}A_n)=\sum_{n\in\N}\mu(A_n) \]
			die rechte Seite definiert sein muss, also unbedingt konvergiert. Außerdem muss für $A,B\in\mf{S}$
			\[ A\cap B=\varnothing\Rightarrow \mu(A\cup B)=\mu(A)+\mu(B). \]
			Somit dürfen nicht $-\infty$ und $\infty$ beide also Funktionswerte auftreten. Nämlich seien $A,B\in\mf{S}$ mit $\mu(A)=+\infty, \mu(B)=-\infty$, dann gilt
			\[ \mu(B)=\mu(B\setminus A)+\mu(B\cap A) \]
			\[ \mu(A)=\mu(A\setminus B)+\mu(B\cap A), \]
			was einen Widerspruch ergibt. Damit darf nur eine Unendlichkeit als Wert auftreten. 
		\end{bem}
		
		\begin{bsp}
			Für beliebiges Maß $\mu$ ist
			\[ \nu(A)=\int_A fd\mu \]
			sigmaadditiv, außerdem gilt
			\[ \int_Afd\mu=\int_A f_+d\mu-\int_A f_-d\mu, \]
			wir können also so $\nu$ auch aufteilen in
			\[ \nu(A)=\nu_+(A)-\nu_-(A), \]
			wobei aufgrund der quadiintegrierbarkeit von $f$ eines der beiden Integrale endlich sein muss. 
		\end{bsp}
		
		\begin{bem}
			Man kann jedes signierte Maß als Differenz zweier Maßfunktionen schreiben. Diese Aufteilung ist jedoch nicht eindeutig. Wir können sie jedoch eindeutig machen, indem wir $\mf{S}$ in positive und negative Mengen $P,N$ aufteilen und fordern, dass $\nu_+=0$ auf $N$ und $\nu_-=0$ auf $P$. Solche Maße heißen singulär. Diese Aufteilung ist das Ziel der nächsten Absätze.
		\end{bem}
		
		\begin{defi}
			Eine Menge $A\in\mf{S}$ heißt
			\begin{itemize}
				\item positiv, falls
				\[ \forall B\subseteq A, B\in\mf{S}: \mu(B)\ge 0 \]
				\item negativ, falls
				\[ \forall B\subseteq A, B\in\mf{S}: \mu(B)\le 0 \]
				\item Nullmenge, falls
				\[ \forall B\subseteq A, B\in\mf{S}: \mu(B)= 0 \]
			\end{itemize}
		\end{defi}
		
		\begin{defi}
			$(N,P)$ heißt Hahn-Zerlegung, falls $N\cap P=\varnothing$, $N\cup P=\Omega$, $P$ positiv, $N$ negativ. 
		\end{defi}
		
		\begin{satz}[Zerlegungssatz von Hahn]
			Zu jedem signierten Maß gibt es eine Hahn-Zerlegung. Diese ist eindeutig bestimmt bis auf Nullmengen.
		\end{satz}
		
		\begin{bem}
			Sind $(P,N), (\tilde{P},\tilde{N})$ zwei Hahn-Zerlegungen, dann ist $P\Delta \tilde{P}$ eine Nullmenge. 
		\end{bem}
		
		\begin{lemma}
			Ist $A\in\mf{S}$ und $\mu(\mf{S})\subseteq ]-\infty,\infty]$, so gibt es eine positive Menge $B\subseteq A$ so, dass 
			\[ \mu(B)\ge\mu(A). \]
		\end{lemma}
		
		\begin{bew}
			Wir schnitzen alles weg, was nicht wie eine positive Menge aussieht.
			\[ A_0=A, n_i=\inf\{n\in\N_0: \exists B\subseteq A_{i-1}, B\in\mf{S},\mu(B)<-2^{-n}\}. \]
			\[ B_i\subseteq A_{i-1}, \mu(B_i)<-2^{-{n_i}}, A_i=A_{i-1}\setminus B_i. \] 
			Bilden wir den Durchschnitt aller $A_i$, so erhalten wir im $i-$ten Schritt nur noch Mengen, die mindestens Maß $-2^{-n}$ haben, so erhalten wir die gesuchte Menge $B$. 
		\end{bew}
		
		\begin{lemma}
			Ist $A\in\mf{S}$ und $\mu(\mf{S})\subseteq ]-\infty,\infty]$, so gibt es eine negative Menge $B\subseteq A$ so, dass 
			\[ \mu(B)\le\mu(A). \]
		\end{lemma}
		
		\begin{bew}
			Wenn $\mu(A)\ge0$, so ist $B=\varnothing$ die gesuchte Menge. Sei also $\mu(A)<0$. Ist $C\subseteq A$, so ist $\mu(C)<\infty$. Nun können wir den Beweis analog zu den positiven Mengen führen. 
		\end{bew}
		
		\begin{lemma}
			Sei $(A_n)_{n\in\N}$ mit $A_n\in\mf{S}$ positiv (bzw. negativ). Dann ist 
			\[ \bigcup_{n\in\N} A_n \text{positiv (bzw. negativ)}. \]
		\end{lemma}
		
		\begin{bew}
			Wir setzen $B\subseteq \bigcup_{n\in\N} A_n=\bigcup_{n\in\N}C_n$, wobei 
			\[ C_n:=A_n\setminus\bigcup_{i<n} A_i\subseteq A_n\Rightarrow C_n\text{ positiv}. \]
			Nun ist 
			\[ \mu(B)=\mu(B\cap\bigcup_{n\in\N}C_n)=\mu(\bigcup_{n\in\N})=\sum_{n\in\N}\underbrace{\mu(B\cap\C_n)}_{\ge 0}\ge 0. \]
		\end{bew}
		
		\begin{bew}[Hahn-Zerlegung]
			Folgt aus den obigen Lemmas: Wir setzen:
			\[ c=\inf\{\mu(A): A\in\mf{S}, \text{negativ}\}. \]
			Wir zeigen: Es gibt eine negative Menge $N$ mit $\mu(N)=c$, also das Infimum ist ein Minimum. Dazu nehmen wir eine Folge $(c_n)_{n\in\N}$, $c_n>c$, $c_n\downarrow c$. Nun gibt es für jedes $n$ eine negative Menge $N_n$ mit 
			\[ \mu(N_n)<c_n. \]
			Sei nun
			\[ N:=\bigcup_{n\in\N} N_n. \]
			Dies ist nun eine negative Menge. Außerdem können wir schreiben
			\[ \mu(N)=\mu(N_n)+\underbrace{\mu(N\setminus N_n)}_{\le0}\le \mu(N_n)\le c_n \]
			\[ \Rightarrow (\mu(N)\le\lim_{n\to\infty} c_n=c\land \mu(N)\ge c (\text{lt. Def. von c}))\Rightarrow\mu(N)=c\land c>-\infty.  \]
			Behauptung: $P:=N^c$ ist positiv. Angenommen $P$ wäre nicht positiv, so 
			\[ \exists A\subseteq P: \mu(A)<0. \]
			\[ \Rightarrow \exists B\subseteq A \text{ negativ}\land\mu(B)\le\mu(A)<0  \]
			\[ \Rightarrow N\cup B\text{ negativ}\land \mu(N\cup B)=\mu(N)+\mu(B)=c+\mu(B)<c\;\;\lightning \] 
			Eindeutigkeit: Seien $(P,N), (\tilde{P},\tilde{N})$ zwei Hahn-Zerlegungen. Wir wollen zeigen
			\[ \mu(P\Delta\tilde{P})=0. \]
			Es gilt
			\[ P\Delta\tilde{P}=(P\setminus\tilde{P})\cup(\tilde{P}\setminus P)=\underbrace{(P\cap\tilde{N})}_{\text{Nullmenge}}\cup\underbrace{(\tilde{P}\cap N)}_{\text{Nullmenge}}. \]
		\end{bew}
		
		\begin{defi}[Jordan-Zerlegung]
			Sei $\mu$ ein signiertes Maß auf einem Messraum $(\Omega,\mf{S})$, dann heißt $(\mu_+,\mu_-)$ eine Jordan-Zerlegung, wenn $\mu_+, \mu_-$ Maße auf demselben Raum, die singulär zu einander sind ($\mu_+\perp\mu_-$) und
			\[ \mu=\mu_+-\mu_-. \]
		\end{defi}
		
		\begin{satz}[Zerlegungssatz von Jordan]
			Für jedes signierte Maß $\mu$ existiert genau eine Jordan-Zerlegung. 
		\end{satz}
		
		\begin{bew}
			Sei $(P,N)$ eine Hahn-Zerlegung, so setzen wir
			\[ \mu_+(A)=\mu(A\cap P), \]
			\[ \mu_-(A)=-\mu(A\cap N). \]
			Dies sind offensichtlich beides Maße und es gilt klarerweise
			\[ \mu_+(A)-\mu_-(A)=\mu(A\cap P)+\mu(A\cap N)=\mu(A), \]
			da $N,P$ disjunkt sind. Außerdem ist
			\[ \mu_+(N)=\mu_-(P)=0. \]
			Zu zeigen bleibt also noch die Eindeutigkeit. Sei dazu $(\tilde{\mu}_+, \tilde{\mu}_-)$ eine zweite Jordanzerlegung. Sei dann $\tilde{P}, \tilde{N}:=\tilde{P}^c$ mit
			\[ \tilde{\mu}_+(\tilde{N})=\tilde{\mu}_-(\tilde{P})=0, \]
			$\tilde{P}$ positiv, $\tilde{N}$ negativ. Nun gilt
			\[ \forall A\subseteq\tilde{P}\Rightarrow \mu_-(A)=0 \]
			\[ \Rightarrow \mu(A)=\tilde\mu_+(A)-\tilde\mu_-(A)=\tilde\mu_+(A)\ge 0.  \]
			\[ \Rightarrow \tilde{\mu}_+(A)=\mu(\tilde{P}\cap A) \]
			\[ \Rightarrow \tilde{\mu}_-(A)=-\mu(\tilde{N}\cap A). \]
			Also:
			\[ \tilde{\mu}_+(A)=\mu(A\cap\tilde{P})=\mu(A\cap\tilde{P}\cap P)+\mu(A\cap\tilde{P}\cap N)=\mu(A\cap\tilde{P}\cap P)=\mu_+(A) \]
			\[ \tilde{\mu}_-(A)=\mu_-(A), \]
			also ist die Jordan-Zerlegung eindeutig. 
		\end{bew}
		
		\begin{bsp}
			Für welche Zahlen $\alpha$ gibt es ein signiertes Maß $\mu$ auf $(\N,2^\N)$ mit $\mu(\{n\})=(-1)^nn^\alpha$?\newline
			Nun ist klarerweise
			\[ P=2\odot \N \]
			\[ N=2\odot \N\ominus 1. \]
			Damit es so ein $\mu$ gibt, muss entweder
			\[ \sum_{n=1}^\infty (2n)^\alpha<\infty\text{ oder }\sum_{n=1}^\infty (2n-1)^\alpha <\infty. \]
			Dies ist offensichtlich dann der Fall, wenn $\alpha<-1$. 
		\end{bsp}
		
		\begin{bsp}
			Für welche Zahlen $\alpha$ gibt es auf $([0,1],\mf{B})$ ein $\mu$ mit 
			\[ \mu([0,x])=x^\alpha\sin(\frac{1}{x})? \]
			Es gilt
			\[ \mu(]a,x])=x^\alpha\sin(\frac{1}{x})-a^\alpha\sin(\frac{1}{a}), \]
			also:
			\[ \mu(]a,x])=\int_a^x(u^\alpha\sin(\frac{1}{u}))'du=\int_{]a,x]}(\alpha u^{\alpha-1}\sin(\frac{1}{u})-u^{\alpha-2}\cos(\frac{1}{u}))d\lambda(x). \]
			Wenn ein signiertes Maß existiert, dann ist 
			\[ \mu(A)=\int_A f'(u)du. \]
			Damit muss $f'$ integrierbar sein. Damit muss $\alpha>1$.  
		\end{bsp}
		
		\begin{defi}
			Das Maß $\mu_+$ einer Jordan-Zerlegung heißt die positive Variation, $\mu_-$ die negative Variation. $|\mu|=\mu_++\mu_-$ heißt die Totalvariation.
		\end{defi}
		
		\begin{bsp}
			Sei $\mu$ ein endliches signiertes Maß auf $(\R,\mf{B})$, bzw auf $(]a,b],\mf{B})$. Dann ist
			\[ \mu_+,\mu_-\text{ endlich}\Rightarrow \text{Es gibt Verteilungsfunktionen }F_+,F_- \]
			\[ \Rightarrow \mu_+(]a,b])=F_+(b)-F_+(a) \text{ und } \mu_-(]a,b])=F_-(b)-F_-(a). \]
			\[ \Rightarrow \mu(]a,b])=F(b)-F(a)\text{ für }F=F_+-F_-. \]
			Es gibt also so etwas wie eine Verteilungsfunktion für das signierte Maß, die wir als Differenz zweier monotoner, nicht fallender Funktionen ist. 
		\end{bsp}
		
		\begin{defi}
			$f:[a,b]\to\R$ heißt von beschränkter Variation, falls es nichtfallende Funktionen Funktionen $g,h: [a,b]\to\R$ gibt mit $f=g-h$. 
		\end{defi}
		
		\begin{defi}
			Die Totalvariation von $f$ auf $a,b$ 
			\[ V_a^bf=\sup\{\sum_{i=1}^n |f(t_i)-f(t_{i-1})|, n\in\N, a=t_0<t_1<...<t_n=b\}. \]
		\end{defi}
		
		\begin{satz}
			Es gilt 
			\[f=f_1-f_2\text{ mit }f_1,f_2\uparrow \Leftrightarrow V_a^bf<\infty. \]
		\end{satz}
		
		\begin{bew}
			Siehe Übungsaufgabe Analysis. Aber wir machens trotzdem:\newline
			"`$\Rightarrow$"': Wächst $f$ monoton, so sind in der Definition der Variation alle Summen positiv, also gilt
			\[ V_a^b f=f(b)-f(a). \]
			Außerdem
			\[ V_a^b(f+g)\le V_a^b f+V_a^b g. \]
			Damit
			\[ V_a^b f\le V_a^bf_1+V_a^b f_2=f_1(b)-f_1(a)+f_2(b)-f_2(a)<\infty. \]
			"`$\Leftarrow$"': Umgekehrt setzen wir $f_1(x)=V_a^x f$. Damit gilt
			\[ a<x<y: V_a^yf=V_a^xf+V_a^y f. \]
			und
			\[ V_a^b f\ge |f(b)-f(a)|. \]
			Nun ist auch $f_2=V_a^x-f$ monoton nicht fallend: 
			\[ x<y: f_2(y)-f_2(x)=V_a^y f-f(y)=V_a^x f+f(x)=V_x^y f-(f(y)-f(x))\ge|f(y)-f(x)|-(f(y)-f(x))\ge 0. \]
		\end{bew}
		
		\begin{bem}
			Nun ist 
			\[ |\mu|(]\tilde{a},\tilde{b}])=V_a^b F, \text{ wenn }\mu(]\tilde{a},\tilde{b}])=F(\tilde{b})-F(\tilde{a}). \]
		\end{bem}
